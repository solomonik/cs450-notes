%%\documentclass[handout]{beamer}
%\documentclass[aspectratio=169,13pt]{beamer}

%\newcommand{\dee}{\partial}

\newcommand{\bmat}[1]{\begin{bmatrix} \B{#1}_{11} & \B{#1}_{12} \\ \B{#1}_{21} & \B{#1}_{22}\end{bmatrix}}
\newcommand{\blmat}[1]{\begin{bmatrix} \B{#1}_{11} &  \\ \B{#1}_{21} & \B{#1}_{22}\end{bmatrix}}
\newcommand{\brmat}[1]{\begin{bmatrix} \B{#1}_{11} & \B{#1}_{12} \\ & \B{#1}_{22}\end{bmatrix}}
\newcommand{\bomat}[1]{\begin{bmatrix} {#1}_{11} & \B{#1}_{12} \\ \B{#1}_{21} & \B{\uppercase{#1}}_{22}\end{bmatrix}}
\newcommand{\blomat}[1]{\begin{bmatrix} {#1}_{11} & \\ \B{#1}_{21} & \B{\uppercase{#1}}_{22}\end{bmatrix}}
\newcommand{\bromat}[1]{\begin{bmatrix} {#1}_{11} & \B{#1}_{12} \\  & \B{\uppercase{#1}}_{22}\end{bmatrix}}

\DeclareMathOperator{\vcop}{vec}
\newcommand{\vc}[1]{\vcop\left(#1\right)}
\newcommand{\prm}{\mu}

\newcommand{\fl}{fl}
\newcommand{\asmcond}[1]{\smallskip{\it #1}\smallskip}
\newcommand{\amdcond}[1]{\smallskip{\it #1}\smallskip}
\newcommand{\algcond}[1]{\smallskip{\it #1}\smallskip}
\newcommand{\atsmcond}[1]{{\it #1}}
\newcommand{\atmdcond}[1]{{\it #1}}
\newcommand{\atlgcond}[1]{{\it #1}}
\newcommand{\afillnum}[1]{{\underline{ \ #1\ \ }}}

\mode<presentation>
{
  \usetheme{Malmoe}
}

\usepackage[english]{babel}
%
%\usepackage[latin1]{inputenc}
%
%\usepackage{times}
%\usepackage[T1]{fontenc}
% Or whatever. Note that the encoding and the font should match. If T1
% does not look nice, try deleting the line with the fontenc.

\usepackage{inconsolata}
\usepackage{listings}
\usepackage{bm}
\usefonttheme[onlymath]{serif}
\renewcommand\familydefault{\sfdefault}
\usepackage[sfdefault]{ClearSans} %% option 'sfdefault' activates Clear Sans as the default text font
%\usefonttheme{professionalfonts}

\usepackage{latexsym}
\usepackage{amsmath}
\usepackage{amssymb}
\usepackage{amsfonts}
\usepackage{graphics}
\usepackage{multicol}

\definecolor{darkred}{rgb}{0.8,0.2,0.2}
\newcommand{\coloremph}[1]{\textcolor{darkred}{\emph{#1}}}        % colored italics
\newcommand{\reshape}[2]{o_{#1}(#2)}        % colored italics
\newcommand{\ket}[1]{\lvert #1 \rangle}
\newcommand{\bra}[1]{\langle #1 \vert}
\newcommand{\transp}[2]{{#2}^{\langle  {#1} \rangle }}        % colored italics

% Miscellaneous special math symbols

% Big-oh notation - either Latex's calligraphic O or uppercase math italic O
\newcommand{\BIGOH}{\mathcal{O}}                              % big oh
%\newcommand{\BIGOH}{O}                                       % big oh
\newcommand{\BIGTHETA}{\Theta}                                % big theta

\newcommand{\sfrac}[2]{{#1}/{#2}}

% Various versions of the fraction one-half
% solidus 1/2 for superscripts:
\newcommand{\SHALF}{1/2}                                      % one-half power
% small 1/2 set case in displayed equations:
\newcommand{\HALF}{\mbox{\small $\frac{1}{2}$}}               % small one-half
\newcommand{\THRD}{\mbox{\scriptsize $\frac{1}{3}$}}          % small one-third
\newcommand{\TWOTH}{\mbox{\scriptsize $\frac{2}{3}$}}         % small two-thirds

% Machine epsilon - note that placement of subscript may need adjustment.
%\newcommand{\emach}{\epsilon_{\textrm{\scriptsize mach}}} % machine prec.
\newcommand{\emach}{\epsilon}

\newcommand{\lsapprox}{\cong}                             % least squares approx

% Real numbers - Prefer \mathbb{R} but it requires amsfonts.
%                Latex's calligraphic R will do if amsfonts are unavailable.
%                DO NOT use \Re, which gives old German fraktur R.
\newcommand{\Real}{\mathbb{R}}                                 % real numbers
\newcommand{\Cplx}{\mathbb{C}}                                 % complex numbers
\newcommand{\Poly}{\mathbb{P}}                                 % polynomials
\newcommand{\Float}{\mathbb{F}}                                % fl pt system
% Bold math fonts for vectors and matrices
\renewcommand{\Vec}[1]{\ensuremath{\bm{#1}}}                   % vector
\newcommand{\Mat}[1]{\ensuremath{\bm{#1}}}                     % matrix
\newcommand{\Op}[1]{#1}                              % operator
\newcommand{\loc}[1]{x_{#1}}                              % operator

% Loglike functions set in regular type
\newcommand{\diag}{\mathrm{diag}}                             % diagonal matrix
\newcommand{\cond}{\mathrm{cond}}                             % condition number
\newcommand{\sign}{\mathrm{sign}}                             % sign function
\newcommand{\Span}{\text{span}}                             % span of matrix
%\newcommand{\trace}{\mathrm{trace}}                           % trace of matrix
\DeclareMathOperator*{\trace}{trace}
\newcommand{\tentrace}[3]{\trace_{#1,#2}(#3)}
\renewcommand{\Re}{\mathrm{Re}}                               % real part
\renewcommand{\Im}{\mathrm{Im}}                               % imaginary part

% Fractions appearing in matrices - allows setting them case or solidus
%\newcommand{\mf}[2]{{#1 \over #2}}        % matrix fraction - set case
\newcommand{\mf}[2]{#1/#2}               % matrix fraction - set solidus


% Keywords for algorithm statements

\newcommand{\FOR}{\textbf{for}\ }
\newcommand{\TO}{\textbf{to}\ }
\newcommand{\IF}{\textbf{if}\ }
\newcommand{\THEN}{\textbf{then}\ }
\newcommand{\ELSE}{\textbf{else}\ }
\newcommand{\WHILE}{\textbf{while}\ }
\newcommand{\DO}{\textbf{do}\ }
\newcommand{\BEGIN}{\textbf{begin}\ }
\newcommand{\END}{\textbf{end}\ }
\newcommand{\STOP}{\textbf{stop}\ }
\newcommand{\AND}{\textbf{and}\ }
\newcommand{\OR}{\textbf{or}\ }

\newcommand{\comm}{\mathrm{comm}}
\newcommand{\comp}{\mathrm{comp}}
\newcommand{\idle}{\mathrm{idle}}
\newcommand{\msg}{\mathrm{msg}}
\newcommand{\len}{\mathrm{s}}
\newcommand{\route}{\mathrm{route}}
\newcommand{\mitem}{\medskip\item}
\newcommand{\sitem}{\smallskip\item}
\DeclareMathOperator*{\argmin}{argmin}

%Gets rid of headline

\setbeamertemplate{footline}{}

\setbeamertemplate{headline}{}
\beamertemplatenavigationsymbolsempty



\definecolor{mygreen}{rgb}{0,0.2,0}
\definecolor{mygray}{rgb}{0.5,0.5,0.5}
\definecolor{mymauve}{rgb}{0.58,0,0.82}
\definecolor{mypurple}{rgb}{0.38,0,0.32}
\definecolor{myblue}{rgb}{0.2,0,0.5}
\definecolor{darkgreen}{rgb}{0.2,0.6,0.2}
\definecolor{Brown}{rgb}{0.39, 0.09, 0.0}
\definecolor{Violet}{rgb}{0.38, 0.0, 0.63}
\definecolor{myvlgray}{rgb}{0.9,0.9,1.0}
\definecolor{mylgray}{rgb}{0.85,0.85,0.9}
\definecolor{darkgreen}{rgb}{0,0.4,0}
\definecolor{brown}{rgb}{.6,.1,.1}

\newcommand{\bemph}[1]{{\color{blue} \emph{#1}}}

\usepackage{mathtools}
\newcommand{\defeq}{\coloneqq}

\newcommand{\inti}[2]{[{#1},{#2}]}
\newcommand{\BF}{\mathbf}
\newcommand{\CF}{\mathcal}
\newcommand{\B}[1]{\bm{#1}}
\newcommand{\E}[1]{\bm{#1}}
\newcommand{\dis}{\displaystyle}
\newcommand{\lt}{\left}
\newcommand{\rt}{\right}

\newcommand{\cs}{H}

\DeclareMathOperator*{\rank}{rank}
\DeclareMathOperator*{\vecn}{vec}
\DeclareMathOperator*{\vecs}{vech}

\usepackage{eso-pic}
\newcommand{\cornertext}[1]{
  \AddToShipoutPictureFG*{
    \AtPageUpperLeft{\put(0,-10){\makebox[\paperwidth][r]{#1}}}  
   }%
}
\newcommand{\cornertexttwo}[2]{
  \AddToShipoutPictureFG*{
    \AtPageUpperLeft{\put(0,-10){\makebox[\paperwidth][r]{#1}}}  
    \AtPageUpperLeft{\put(0,-20){\makebox[\paperwidth][r]{#2}}}  
   }%
}
\newcommand{\linkdemo}[2]{{\footnotesize\href{https://relate.cs.illinois.edu/course/cs450-f18/f/demos/upload/#1/#2.html}{\color{Violet}{\it\textbf{Demo:} #2}\ }}}
\newcommand{\linkinclass}[2]{{\footnotesize\href{https://relate.cs.illinois.edu/course/cs450-f18/flow/#1/start/}{\color{darkgreen}{\it\textbf{Activity:} #2}\ }}}
\newcommand{\urcornerlinkinclass}[2]{\cornertext{\linkinclass{#1}{#2}}}
\newcommand{\dblurcornerlinkinclass}[4]{\cornertexttwo{\linkinclass{#1}{#2}}{\linkinclass{#3}{#4}}}
\newcommand{\urcornerlinkdemo}[2]{\cornertext{\linkdemo{#1}{#2}}}
\newcommand{\dblurcornerlinkdemo}[4]{\cornertexttwo{\linkdemo{#1}{#2}}{\linkdemo{#3}{#4}}}
\newcommand{\urcornerlinkdemoinclass}[4]{\cornertexttwo{\linkdemo{#1}{#2}}{\linkinclass{#3}{#4}}}



% CS 554 specific:

\newcommand{\tsync}{\alpha}
\newcommand{\tword}{\beta}
\newcommand{\tflop}{\gamma}

\newcommand{\bw}{w}

\newcommand{\tpl}[2]{{\bm{#1}}}
\newcommand{\vtpl}[3]{{\bm{#1}}_#3}
\newcommand{\perm}[2]{\lt[#1\rt]_{#2}}
\newcommand{\costyle}{\ttfamily\bfseries}
\newcommand{\cotext}[1]{{\costyle{#1}}}
\newcommand{\kwstyle}{\costyle\textcolor{mypurple}}
\newcommand{\kwtext}[1]{{\kwstyle{#1}}}
\newcommand{\emstyle}{\costyle\textcolor{myblue}}
\newcommand{\emtext}[1]{{\emstyle{#1}}}
\newcommand{\const}{}
\newcommand{\work}{Q}
\newcommand{\depth}{D}
\newcommand{\flops}{F}
\newcommand{\words}{W}
\newcommand{\syncs}{S}
\newcommand{\mem}{M}
\newcommand{\eff}{E}
\newcommand{\isofun}[1]{\tilde{\work}(p)}
\newcommand{\isomem}[1]{\tilde{\mem}(p)}
\newcommand{\pstrong}{p_s}
\newcommand{\pweak}{p_w}
\newcommand{\ALL}{\star }
\newcommand{\rmn}[2]{\mathbb{R}^{#1\times #2}}
\newcommand{\rn}[1]{\mathbb{R}^{#1}}
\newcommand{\err}{\varepsilon}
\newcommand{\mat}[1]{\begin{bmatrix} #1 \end{bmatrix}}
\newcommand{\mc}[1]{\mathcal{#1}}
\newcommand{\h}[2]{\mc{H}_{#1}(#2)}
\newcommand{\T}{T}%{\mathsf{T}}
\newcommand{\dn}[2]{\mc{M}_{#1}^{\uparrow}(#2)}
%\usepackage[colorlinks=false,urlbordercolor={1.0 1.0 1.0}]{hyperref}
%
%\lstset{ %
%  postbreak=false,
%%  backgroundcolor=\color{white},   % choose the background color; you must add \usepackage{color} or \usepackage{xcolor}
%  basicstyle=\costyle,        % the size of the fonts that are used for the code
%%  breakatwhitespace=false,         % sets if automatic breaks should only happen at whitespace
%%  breaklines=false,                 % sets automatic line breaking
%  captionpos=n,                    % sets the caption-position to bottom
%  commentstyle=\color{mygreen},    % comment style
%  deletekeywords={...},            % if you want to delete keywords from the given language
%  escapeinside={\%*}{*)},          % if you want to add LaTeX within your code
%  extendedchars=true,              % lets you use non-ASCII characters; for 8-bits encodings only, does not work with UTF-8
%  frame=none,                    % adds a frame around the code
%  keepspaces=true,                 % keeps spaces in text, useful for keeping indentation of code (possibly needs columns=flexible)
%  keywordstyle=\color{mypurple},       % keyword style
%  language=C++,                 % the language of the code
%  otherkeywords={*,...},            % if you want to add more keywords to the set
%  numbers=none,                    % where to put the line-numbers; possible values are (none, left, right)
%  numbersep=5pt,                   % how far the line-numbers are from the code
%  numberstyle=\tiny\color{mygray}, % the style that is used for the line-numbers
%  rulecolor=\color{black},         % if not set, the frame-color may be changed on line-breaks within not-black text (e.g. comments (green here))
%  showspaces=false,                % show spaces everywhere adding particular underscores; it overrides 'showstringspaces'
%  showstringspaces=false,          % underline spaces within strings only
%  showtabs=false,                  % show tabs within strings adding particular underscores
%  stepnumber=2,                    % the step between two line-numbers. If it's 1, each line will be numbered
%  stringstyle=\color{mymauve},     % string literal style
%  tabsize=2,                     % sets default tabsize to 2 spaces
%  title=\lstname,                   % show the filename of files included with \lstinputlisting; also try caption instead of title
%  keywordstyle=\emstyle
%}

\lstset{ %
  postbreak=false,
%  backgroundcolor=\color{white},   % choose the background color; you must add \usepackage{color} or \usepackage{xcolor}
  basicstyle=\footnotesize\costyle,        % the size of the fonts that are used for the code
%  breakatwhitespace=false,         % sets if automatic breaks should only happen at whitespace
%  breaklines=false,                 % sets automatic line breaking
  captionpos=n,                    % sets the caption-position to bottom
  commentstyle=\color{mygreen},    % comment style
  deletekeywords={privatei,shared},            % if you want to delete keywords from the given language
  escapeinside={\%*}{*)},          % if you want to add LaTeX within your code
  extendedchars=true,              % lets you use non-ASCII characters; for 8-bits encodings only, does not work with UTF-8
  frame=none,                    % adds a frame around the code
  keepspaces=true,                 % keeps spaces in text, useful for keeping indentation of code (possibly needs columns=flexible)
  keywordstyle=\footnotesize\color{mypurple},       % keyword style
  language=C++,                 % the language of the code
  otherkeywords={*,pragma},            % if you want to add more keywords to the set
  numbers=none,                    % where to put the line-numbers; possible values are (none, left, right)
  numbersep=5pt,                   % how far the line-numbers are from the code
  numberstyle=\footnotesize\color{mygray}, % the style that is used for the line-numbers
  rulecolor=\color{black},         % if not set, the frame-color may be changed on line-breaks within not-black text (e.g. comments (green here))
  showspaces=false,                % show spaces everywhere adding particular underscores; it overrides 'showstringspaces'
  showstringspaces=false,          % underline spaces within strings only
  showtabs=false,                  % show tabs within strings adding particular underscores
  stepnumber=2,                    % the step between two line-numbers. If it's 1, each line will be numbered
  stringstyle=\footnotesize\color{mymauve},     % string literal style
  tabsize=2,                     % sets default tabsize to 2 spaces
  title=\lstname,                   % show the filename of files included with \lstinputlisting; also try caption instead of title
  emph={MPI_Send,MPI_Recv,MPI_Comm,MPI_Comm_rank,MPI_Send,MPI_Comm_split,MPI_Init,MPI_Finalize,MPI_Status,MPI_COMM_WORLD,MPI_Datatype,MPI_Comm_size,MPI_FLOAT,mpi.h,omp,parallel,shared,private,default,reduction},
  emphstyle=\footnotesize\emstyle
}

\title{CS 450: Numerical Anlaysis\footnote{{\it These slides have been drafted by Edgar Solomonik as lecture templates and supplementary material for the book ``Scientific Computing: An Introductory Survey'' by Michael T. Heath (\href{http://heath.cs.illinois.edu/scicomp/notes/index.html}{{\color{blue} slides}}).}}}
\author{}
\institute{University of Illinois at Urbana-Champaign}




\subtitle{Introduction to Scientific Computing}
\date{}
\begin{document}

\begin{frame}
  \titlepage
\end{frame}

\section{Motivation and Course Overview}

\begin{frame}[fragile]{Scientific Computing Applications and Context}

\begin{itemize}

\item {\bf Mathematical modelling for computational science}
\algcond{
Typical scientific computing problems are numerical solutions to PDEs
\begin{itemize}
\sitem Newtonian dynamics: simulating particle systems in time
\sitem Fluid and air flow models for engineering
\sitem PDE-constrained numerical optimization: finding optimal configurations (used in engineering of control systems)
\sitem Quantum chemistry (electronic structure calculations): many-electron Schr\"odinger equation
\end{itemize}
}

\item {\bf Linear algebra and computation}

\algcond{
\begin{itemize}
\item Linear algebra and numerical optimization are building blocks for machine learning methods and data analysis
\sitem Computer architecture, compilers, and parallel computing use numerical algorithms (matrix multiplication, Gaussian elimination) as benchmarks
\end{itemize}
}

\end{itemize}
\end{frame}


\begin{frame}[fragile]{Example: Mechanics\footnote{{\it Variational Principles of Mechanics}, Cornelius Lanczos, Dover Books on Physics, 1949.}}

\begin{itemize}

\mitem Newton's laws provide incomplete particle-centric picture

\mitem Physical systems can be described in terms of \coloremph{degrees of freedom} (DoFs)

\begin{itemize}

\sitem A piston moving up and down requires \fillnum{1} DoFs

\sitem 1-particle system requires \fillnum{3} DoFs

\sitem 2-particle system requires \fillnum{6} DoFs

\sitem 2-particles at a fixed distance require \fillnum{5} DoFs

\end{itemize}

\mitem $N$-particle system \coloremph{configuration} described by $3N$ DoFs

\lgcond{
\begin{itemize}

\mitem Trajectories in \coloremph{configuration space} ($\mathbb{R}^{3N}$) describe free energy configuration

\mitem Various choice of \coloremph{basis functions} (i.e. coordinate system) for configuration space are possible

\end{itemize}
}
\end{itemize}

\end{frame}

\begin{frame}[fragile]{Course Structure}

\begin{itemize}
\item {\bf Complex numerical problems are generally reduced to simpler problems}

  \lgcond{
  \begin{itemize}
  \sitem \coloremph{Discretization} corresponds to representing a continuous function/model by a discrete set of points
  \sitem Nonlinear problems are mapped to linear problems
  \sitem Complicated functions are mapped to polynomials
  \sitem Differential equations are mapped to algebraic equations
  \end{itemize}
  }

\item {\bf The course topics will follow this hierarchical structure}

  \lgcond{
  \begin{itemize}
  \sitem Error, conditioning, and floating point are the starting point for representation and evaluation of algorithms for any numerical problem
  \sitem Linear systems provide the simplest and most important building block for solving linear algebra problems
  \sitem Least squares and eigenvalue problems provide basic technology for matrices
  \sitem Nonlinear equations and optimization make use of matrix algebra to solve more general modelling problems
  \sitem Numerical interpolation, differentiation, and quadrature provide the building blocks to reduce numerical PDE problems to matrix algebra
  \end{itemize}
  }
\end{itemize}
\end{frame}

\begin{frame}{Numerical Analysis}

\begin{itemize}

\item {\bf Numerical Problems involving Continuous Phenomena}:

  \lgcond{
    Given input $\Vec{x}\in\mathbb{R}^n$, approximate output $\Vec{y} = f(\Vec{x})$
    \begin{itemize}
    \item Problem is \coloremph{well-posed} if $f$ is a smoothly varying function, $f(\Vec{\hat{x}})\to f(\Vec{x})$ as $\Vec{\hat{x}} \to \Vec{x}$.
    \item Otherwise, problem is \coloremph{ill-posed}
    \end{itemize}

  }

\item {\bf Error Analysis}: 

  \lgcond{
    Quality of approximation is quantified by distance to the solution
    \begin{itemize}

    \item If solution $y=f(\Vec x)$ is a scalar, distance from computed solution $\hat{y}$ to correct answer is the \coloremph{absolute error}
      \[\Delta y = \hat{y} - y,\]
          while the normalized distance is the \coloremph{relative error}
      \[\Delta y/y = \frac{\hat{y} - y}{y}\]
    
    \item More generally, we are interested in the error 
      \[\Delta \Vec{y} = \Vec{\hat{y}} - \Vec{y} \]
          the magnitude of which is measured by a given \coloremph{vector norm}

    \end{itemize}
  }

\end{itemize}

\end{frame}

\section{Numerical Error}

\begin{frame}{Sources of Error}

\urcornerlinkdemo{01-error-and-fp}{Floating Point vs Program Logic}

\begin{itemize}

\item {\bf Representation of Numbers}: 
  
  \lgcond{

  \begin{itemize}
  \sitem We cannot represent arbitrary real numbers in a finite amount of space, e.g. a computer cannot exactly represent $\pi$
  \sitem Moreover, hardware architectures are only well-fit to work with fixed-length (32-bit or 64-bit) representations
  \sitem As we will see, the best we can do is represent a wide range of numbers with a relatively uniform \coloremph{precision}, which corresponds to \coloremph{scientific notation}
  \sitem With scientific notation, we seek to store the most significant digits of each number, so that the magnitude of the relative error in the floating point representation $\fl(x)$ for most real numbers $x$ will be $|\fl(x) - x|/|x| \leq \epsilon$
  \end{itemize}

  }

\item {\bf Propagated Data Error}:
  \mdcond{ 
  error due approximations in the input, $f(\hat{x})-f(x)$
  }

\item {\bf Computational Error = $\hat{f}(x)-f(x)$ = Truncation Error + Rounding Error}

  \lgcond{ 
    \begin{itemize}
    \sitem \coloremph{Truncation error} is the error made due to approximations made by the algorithm (simplified models used in our approximation)
    \sitem \coloremph{Rounding error} is the error made due to inexact representation of quantities computed by the algorithm
    \end{itemize}
  }

\end{itemize}

\end{frame}

\begin{frame}{Error Analysis}

\begin{itemize}

\item {\bf Forward Error}: 

  \lgcond{ 
    \coloremph{Forward error} is the computational error of an algorithm
    \begin{itemize}
    \item Absolute: $\hat{f}(x) - f(x)$
    \item Relative: $(\hat{f}(x) - f(x))/f(x)$ 
    \item Usually, we care about the \coloremph{magnitude} of the final error,
          but carrying through signs is important when analyzing error
    \end{itemize}
  }
  
\item {\bf Backward Error}:

  \lgcond{ 
    It can be hard to tell what a `good' forward error is, but
    \coloremph{backward error analysis} enables us to measure computational error
    with respect to data propagation error
    \begin{itemize}
    \sitem An algorithm is \coloremph{backward stable} if its a solution to a nearby problem
    \sitem If the computed solution $\hat{f}(x)=f(\hat{x})$ then
           $$\text{backward error} = \hat{x}-x$$
    \item More precisely, we want the nearest $\hat{x}$ to $x$ with $\hat{f}(x)=f(\hat{x})$
    \sitem If the backward error is smaller than the propagated data error, the solution computed by the algorithm is as good as possible
    \end{itemize}
    
  }

\end{itemize}

\end{frame}

\begin{frame}{Conditioning}

\urcornerlinkdemo{01-error-and-fp}{Conditioning of Evaluating tan}

\begin{itemize}


\item {\bf Absolute Condition Number}:

  \lgcond{
    The \coloremph{absolute condition number is a property of the problem}, which measures its sensitivity to perturbations in input
    \[\kappa_\text{abs}(f) = \lim_{\text{size of input perturbation} \to 0} \quad \max_{\text{inputs}} \quad \max_{\text{perturbations in input}}\left|\frac{\text{perturbation in output}}{\text{perturbation in input}}\right|\]
    For problem $f$ at input $x$ it is simply the derivative of $f$ at $x$,
    \[\kappa_\text{abs}(f) =\lim_{\Delta x \to 0}  \left|\frac{f(x+\Delta x)-f(x)}{\Delta x}\right| = \left|\frac{df}{dx}(x)\right|\]
    When considering a space of inputs $\mathcal{X}$ it is $\kappa_\text{abs} = \max_{x\in \mathcal{X}}\left|\frac{df}{dx}(x)\right|$
  }

\item {\bf (Relative) Condition Number}:

  \lgcond{ 
    The relative condition number considers relative perturbations in input and output, so that 
    \[\kappa(f) = \kappa_\text{rel}(f) =  \max_{x\in \mathcal{X}}\lim_{\Delta x \to 0} \left|\frac{(f(x+\Delta x)-f(x))/f(x)}{\Delta x/x}\right| = \frac{\kappa_\text{abs}(f) |x|}{|f(x)|}\]
  }

\end{itemize}

\end{frame}

\section{Conditioning}

\begin{frame}{Posedness and Conditioning}

\begin{itemize}

\item {\bf What is the condition number of an ill-posed problem?}

  \lgcond{ 

  \begin{itemize} 
  \mitem If the condition number is bounded and the solution is unique, the problem is \coloremph{well-posed}
  \mitem An \coloremph{ill-posed} problem $f$ either has no unique solution or has a (relative) condition 
         number of $\kappa(f) = \infty$
  \mitem This condition implies that the solutions to problem $f$ are continuous and differentiable 
         in the given space of possible inputs to $f$
  \mitem Sometimes well-posedness is defined to only require continuity
  \mitem Generally, $\kappa(f)$ can be thought of as the reciprocal of the \coloremph{distance} (in an appropriate geometric 
         embedding of problem configurations) from $f$ \coloremph{to the nearest ill-posed problem}
  \end{itemize}


  }
  \lgcond{ }

\end{itemize}

\end{frame}

\begin{frame}{Stability and Accuracy}

\begin{itemize}

\item {\bf Accuracy}:

  \lgcond{ 
     An algorithm is \coloremph{accurate} if $\hat{f}(x)=f(x)$ for all inputs $x$ when $\hat{f}(x)$ is computed in infinite precision
    \begin{itemize}
    \sitem In other words, the truncation error is zero (rounding error is ignored)
    \sitem More generally, an algorithm is accurate if its truncation error is negligible in the desired context
    \sitem Yet more generally, the \coloremph{accuracy} of an algorithm is expressed in terms of bounds on the magnitude of its truncation error
    \end{itemize}
  }

\item {\bf Stability}:

  \lgcond{ 
  An algorithm is \coloremph{stable} if its output in finite precision (floating point arithmetic)
  is always near its output in exact precision 
    \begin{itemize}
    \sitem \coloremph{Stability} measures the sensitivity of an algorithm to \coloremph{roundoff error}
    \sitem In some cases, such as the approximation of a derivative using a finite difference formula,
           there is a trade-off between stability and accuracy
    \end{itemize}
  }
  \lgcond{} 

\end{itemize}

\end{frame}

%\begin{frame}{Floating Point Numbers}
%
%\begin{itemize}
%
%\item {\bf Scientific Notation}
%
%  \lgcond{ }
%
%\item {\bf Significand (Mantissa) and Exponent}
%
%  \lgcond{ }
%
%\end{itemize}
%
%\end{frame}
%
%\begin{frame}{Rounding Error}
%
%\begin{itemize}
%
%\item {\bf Machine Epsilon}
%
%  \lgcond{ }
%
%\item {\bf Rounding Error Analysis}
%
%  \lgcond{ }
%
%\end{itemize}
%
%\end{frame}
%
%\begin{frame}{Rounding Error in Operations}
%
%\begin{itemize}
%
%\item {\bf Addition and Subtraction}
%
%  \lgcond{ }
%
%\item {\bf Multiplication and Division}
%
%  \lgcond{ }
%
%\end{itemize}
%
%\end{frame}
%
%\begin{frame}{Subnormal Numbers}
%
%\begin{itemize}
%
%\item {\bf Subnormal (Denormal) Number Range}
%
%  \lgcond{ }
%
%\item {\bf Gradual Underflow: Avoiding underflow in addition}
%
%  \lgcond{ }
%
%\end{itemize}
%
%\end{frame}
%
%


%\end{document}

\begin{frame}{Error and Conditioning}

\urcornerlinkdemo{01-error-and-fp}{Truncation vs Rounding}

\begin{itemize}
\item Two major sources of error: \coloremph{roundoff} and \coloremph{truncation} error.
\begin{itemize}
\sitem roundoff error concerns floating point error due to finite precision
\sitem truncation error concerns error incurred due to algorithmic approximation, e.g. the representation of a function by a finite Taylor series
  \lgcond{
    \[f(x+h) \approx g(h) = \sum_{i=0}^{k} \frac{f^{(i)}(x)}{i!}h^i\]
    The absolute truncation error of this approximation is
    \[f(x+h) - g(h) = \sum_{i=k+1}^{\infty} \frac{f^{(i)}(x)}{i!}h^i = O(h^{k+1}) \ \text{as}\ h\to 0\]
  }
\vspace{-.1in}
\end{itemize}
\item To study the propagation of roundoff error in arithmetic we can use the notion of conditioning.
  \lgcond{
    The condition number tells use the worst-case amplification of output error with respect to input error
    \[\kappa(f) =  \max_{x\in \mathcal{X}}\lim_{\Delta x \to 0} \left|\frac{(f(x+\Delta x)-f(x))/f(x)}{\Delta x/x}\right| = \frac{|f'(x) x|}{|f(x)|}\]
  }
\end{itemize}

\end{frame}

\section{Floating Point}

\begin{frame}{Floating Point Numbers}

\dblurcornerlinkdemo{01-error-and-fp}{Picking apart a floating point number}{01-error-and-fp}{Density of Floating Point Numbers}

\begin{itemize}

\item {\bf Scientific Notation}

  \lgcond{ 
    Floating-point numbers are a computational realization of scientific notation, 
    \[4.12165 \times 10^6, 2.145 \times 10^{-3}\]
    \begin{itemize}
    \item Scientific-notation provides a unique representation of any real number
          for a given amount of `precision' (number of significant digits)
    \sitem Normalized floating-point numbers are just a binary form of scientific notation, 
    \[1.01001 \times 2^5, 1.0110 \times 2^{-3}\]
    \end{itemize}
  }

\item {\bf Significand (Mantissa) and Exponent} Given $x$ with $s$ leading bits $x_0,\ldots, x_{s-1}$
  \lgcond{ 
    $$\mathit{fl}(x)=\sum_{i=0}^{s-1} x_i 2^{k-i}=\underbrace{x_0.x_1\ldots x_{s-1}}_\text{significand/mantissa}\times 2^{\underbrace{k}_\text{exponent}}$$
  A floating point number's binary representation has $s-1$ significand
  bits (excluding $x_0=1$), some bits to represent the exponent, and a sign bit
  }

\end{itemize}

\end{frame}

\begin{frame}{Rounding Error}

\dblurcornerlinkdemo{01-error-and-fp}{Floating point and the Harmonic Series}{01-error-and-fp}{Floating Point and the Series for the Exponential Function}

\begin{itemize}

\item {\bf Maximum Relative Representation Error (Machine Epsilon)}

  \lgcond{ 
    \begin{itemize}
    \item If we have $s$ significant digits in scientific notation, our error is bounded to variations of $1$ in least significant digit, whose magnitude relative to the number we are trying to represent is $10^{1-s}$ in decimal and $2^{1-s}$ in binary
    \mitem Formally, with $s$ significant binary digits the \coloremph{relative representation error} of positive real number $x$ is (with $k=\lfloor\log_{2}(|x|)\rfloor$ and each $x_i\in\{0,1\}$)
          \[x=\sum_{i=0}^\infty x_i 2^{k-i} = x_\text{rem} + \sum_{i=0}^{s-1} x_i 2^{k-i}, \quad \text{where} \quad 
           |x_\text{rem}/x| \leq 2^{1-s}
          \]
    \item The maximum such error, $2^{1-s}$, is called \coloremph{machine epsilon},
\         \[\emach=\argmin_{\emach>0}(\mathit{fl}(1+\emach)=1+\emach)\]
    \end{itemize}
  }
  \lgcond{}
\end{itemize}

\end{frame}

\begin{frame}{Rounding Error in Operations (I)}

\urcornerlinkdemoinclass{01-error-and-fp}{Catastrophic Cancellation}{inclass-fp}{Cancellation in Standard Deviation Computation}

\begin{itemize}

\item {\bf Addition and Subtraction}

  \lgcond{ 
      \begin{itemize}
      \item Subtraction is just negation of a sign bit followed by addition
      \sitem \coloremph{Catastrophic cancellation} occurs when the magnitude of the result is much smaller than the magnitude of both operands
      \sitem Cancellation corresponds to losing significant digits, e.g.
        \[3.1423\times 10^{5} - 3.1403\times 10^{5} = 2.0 \times 10^{2}\]
      \item Generally, we can bound the error incurred during addition of two real numbers $x,y$ 
            in floating point (ignoring final rounding, which has relative error $\emach$) as
        \[\frac{|(x+y)-(\fl(x)+fl(y)|}{|x+y|} \leq \frac{\emach(|x|+|y|)}{|x+y|}\]
      by this we can also observe that the condition number of addition of $x,y$ i.e. $f(x,y)=x+y$, is $\kappa(f(x,y)) = (|x|+|y|)/|x+y|$
      \sitem Consequently, when $x+y=0$ and $x,y\neq0$ addition is \coloremph{ill-posed}
%so the number of significant digits is only roughly maintained when $|x+y|\approx |x|+|y|$
      \end{itemize}
  }
  \lgcond{}
\end{itemize}

\end{frame}

\begin{frame}{Rounding Error in Operations (II)}

\urcornerlinkdemo{01-error-and-fp}{Polynomial Evaluation Floating Point}

\begin{itemize}

\item {\bf Multiplication and Division}

  \lgcond{ 
    \begin{itemize}
    \item Multiplication is a lot safer than addition in floating point
    \sitem To analyze its error, we use a 2-term \coloremph{Taylor series approximation} typical in relative error analysis
      \[f(\epsilon)= (1+n\epsilon)^k\approx f(0)+\frac{df}{d\epsilon}(0)\epsilon = 1+ kn\epsilon\]
    since $\epsilon$ is small, this linear approximation is accurate (to within $O(\emach^2)$)
    \sitem Aside from final rounding, we can bound the error in multiplication as
        \[\frac{|xy-\fl(\fl(x)fl(y))|}{|xy|} \leq \frac{|xy-(x(1+\emach)y(1+\emach))(1+\emach)|}{|xy|} \approx 3\emach\]
    \item Consequently, multiplication $f(x,y)=xy$ is always \coloremph{well-conditioned}, $\kappa(f)\approx 3$
    \sitem Division is multiplication by the reciprocal, and reciprocation is also well-conditioned
    \end{itemize}
  }
  \lgcond{}

\end{itemize}

\end{frame}

\begin{frame}{Exceptional and Subnormal Numbers}

\begin{itemize}

\item {\bf Exceptional Numbers}
      
  \lgcond{
      We had mentioned that the leading bit in normalized floating point numbers is assumed to be $1$, but how do represent $0$?
      \begin{itemize}
      \item \coloremph{Exceptional} floating point numbers are $0,-0,\infty,-\infty,$ and $\text{NaN}=0/0=\infty-\infty$
      \end{itemize}
  }

\item {\bf Subnormal (Denormal) Number Range}

  \lgcond{
  \begin{itemize}
  \item The range of magnitudes of normalized floating point numbers with an exponent range $[-e,e]$ is $[2^{-e},2^{e+1}(1-\emach/2)]$
  \sitem For numbers of magnitude $<2^{-e}$, the relative representation error is unbounded

  \sitem \coloremph{Subnormal numbers} are evenly spaced in $[-2^{-e},2^{-e}]$ with gaps of $\emach 2^{-e}$
  \sitem Consequently, the absolute representation error in $[-2^{-e},2^{-e}]$ is at most $\emach 2^{-e}$
  \end{itemize} 
  }

\item {\bf Gradual Underflow: Avoiding underflow in addition}

  \lgcond{ 
    The main benefit of subnormal numbers is that for any \coloremph{machine numbers} (floating-point numbers) $x$ and $y$, 
  \(\fl(x-y)=0\) if and only if $x=y$, since the gap between any two representable numbers is $|x-y|\geq \emach 2^{-e}$
  }

\end{itemize}

\end{frame}


%\begin{frame}{Vector Norms}
%
%\begin{itemize}
%\item {\bf Properties of vector norms}
%
%  \lgcond{ }
%
%\item {\bf $p$-norms}
%
%  \lgcond{ }
%
%\end{itemize}
%
%\end{frame}
%
%\begin{frame}{Inner-Product Spaces}
%
%\begin{itemize}
%\item {\bf Properties of inner-product spaces}
%
%  \lgcond{ }
%
%\item {\bf Inner-product-based vector norms}
%
%  \lgcond{ }
%
%\end{itemize}
%\end{frame}



%\end{document}
