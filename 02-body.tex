%%\documentclass[handout]{beamer}
%\documentclass[aspectratio=169,13pt]{beamer}

%\newcommand{\dee}{\partial}

\newcommand{\bmat}[1]{\begin{bmatrix} \B{#1}_{11} & \B{#1}_{12} \\ \B{#1}_{21} & \B{#1}_{22}\end{bmatrix}}
\newcommand{\blmat}[1]{\begin{bmatrix} \B{#1}_{11} &  \\ \B{#1}_{21} & \B{#1}_{22}\end{bmatrix}}
\newcommand{\brmat}[1]{\begin{bmatrix} \B{#1}_{11} & \B{#1}_{12} \\ & \B{#1}_{22}\end{bmatrix}}
\newcommand{\bomat}[1]{\begin{bmatrix} {#1}_{11} & \B{#1}_{12} \\ \B{#1}_{21} & \B{\uppercase{#1}}_{22}\end{bmatrix}}
\newcommand{\blomat}[1]{\begin{bmatrix} {#1}_{11} & \\ \B{#1}_{21} & \B{\uppercase{#1}}_{22}\end{bmatrix}}
\newcommand{\bromat}[1]{\begin{bmatrix} {#1}_{11} & \B{#1}_{12} \\  & \B{\uppercase{#1}}_{22}\end{bmatrix}}

\DeclareMathOperator{\vcop}{vec}
\newcommand{\vc}[1]{\vcop\left(#1\right)}
\newcommand{\prm}{\mu}

\newcommand{\fl}{fl}
\newcommand{\asmcond}[1]{\smallskip{\it #1}\smallskip}
\newcommand{\amdcond}[1]{\smallskip{\it #1}\smallskip}
\newcommand{\algcond}[1]{\smallskip{\it #1}\smallskip}
\newcommand{\atsmcond}[1]{{\it #1}}
\newcommand{\atmdcond}[1]{{\it #1}}
\newcommand{\atlgcond}[1]{{\it #1}}
\newcommand{\afillnum}[1]{{\underline{ \ #1\ \ }}}

\mode<presentation>
{
  \usetheme{Malmoe}
}

\usepackage[english]{babel}
%
%\usepackage[latin1]{inputenc}
%
%\usepackage{times}
%\usepackage[T1]{fontenc}
% Or whatever. Note that the encoding and the font should match. If T1
% does not look nice, try deleting the line with the fontenc.

\usepackage{inconsolata}
\usepackage{listings}
\usepackage{bm}
\usefonttheme[onlymath]{serif}
\renewcommand\familydefault{\sfdefault}
\usepackage[sfdefault]{ClearSans} %% option 'sfdefault' activates Clear Sans as the default text font
%\usefonttheme{professionalfonts}

\usepackage{latexsym}
\usepackage{amsmath}
\usepackage{amssymb}
\usepackage{amsfonts}
\usepackage{graphics}
\usepackage{multicol}

\definecolor{darkred}{rgb}{0.8,0.2,0.2}
\newcommand{\coloremph}[1]{\textcolor{darkred}{\emph{#1}}}        % colored italics
\newcommand{\reshape}[2]{o_{#1}(#2)}        % colored italics
\newcommand{\ket}[1]{\lvert #1 \rangle}
\newcommand{\bra}[1]{\langle #1 \vert}
\newcommand{\transp}[2]{{#2}^{\langle  {#1} \rangle }}        % colored italics

% Miscellaneous special math symbols

% Big-oh notation - either Latex's calligraphic O or uppercase math italic O
\newcommand{\BIGOH}{\mathcal{O}}                              % big oh
%\newcommand{\BIGOH}{O}                                       % big oh
\newcommand{\BIGTHETA}{\Theta}                                % big theta

\newcommand{\sfrac}[2]{{#1}/{#2}}

% Various versions of the fraction one-half
% solidus 1/2 for superscripts:
\newcommand{\SHALF}{1/2}                                      % one-half power
% small 1/2 set case in displayed equations:
\newcommand{\HALF}{\mbox{\small $\frac{1}{2}$}}               % small one-half
\newcommand{\THRD}{\mbox{\scriptsize $\frac{1}{3}$}}          % small one-third
\newcommand{\TWOTH}{\mbox{\scriptsize $\frac{2}{3}$}}         % small two-thirds

% Machine epsilon - note that placement of subscript may need adjustment.
%\newcommand{\emach}{\epsilon_{\textrm{\scriptsize mach}}} % machine prec.
\newcommand{\emach}{\epsilon}

\newcommand{\lsapprox}{\cong}                             % least squares approx

% Real numbers - Prefer \mathbb{R} but it requires amsfonts.
%                Latex's calligraphic R will do if amsfonts are unavailable.
%                DO NOT use \Re, which gives old German fraktur R.
\newcommand{\Real}{\mathbb{R}}                                 % real numbers
\newcommand{\Cplx}{\mathbb{C}}                                 % complex numbers
\newcommand{\Poly}{\mathbb{P}}                                 % polynomials
\newcommand{\Float}{\mathbb{F}}                                % fl pt system
% Bold math fonts for vectors and matrices
\renewcommand{\Vec}[1]{\ensuremath{\bm{#1}}}                   % vector
\newcommand{\Mat}[1]{\ensuremath{\bm{#1}}}                     % matrix
\newcommand{\Op}[1]{#1}                              % operator
\newcommand{\loc}[1]{x_{#1}}                              % operator

% Loglike functions set in regular type
\newcommand{\diag}{\mathrm{diag}}                             % diagonal matrix
\newcommand{\cond}{\mathrm{cond}}                             % condition number
\newcommand{\sign}{\mathrm{sign}}                             % sign function
\newcommand{\Span}{\text{span}}                             % span of matrix
%\newcommand{\trace}{\mathrm{trace}}                           % trace of matrix
\DeclareMathOperator*{\trace}{trace}
\newcommand{\tentrace}[3]{\trace_{#1,#2}(#3)}
\renewcommand{\Re}{\mathrm{Re}}                               % real part
\renewcommand{\Im}{\mathrm{Im}}                               % imaginary part

% Fractions appearing in matrices - allows setting them case or solidus
%\newcommand{\mf}[2]{{#1 \over #2}}        % matrix fraction - set case
\newcommand{\mf}[2]{#1/#2}               % matrix fraction - set solidus


% Keywords for algorithm statements

\newcommand{\FOR}{\textbf{for}\ }
\newcommand{\TO}{\textbf{to}\ }
\newcommand{\IF}{\textbf{if}\ }
\newcommand{\THEN}{\textbf{then}\ }
\newcommand{\ELSE}{\textbf{else}\ }
\newcommand{\WHILE}{\textbf{while}\ }
\newcommand{\DO}{\textbf{do}\ }
\newcommand{\BEGIN}{\textbf{begin}\ }
\newcommand{\END}{\textbf{end}\ }
\newcommand{\STOP}{\textbf{stop}\ }
\newcommand{\AND}{\textbf{and}\ }
\newcommand{\OR}{\textbf{or}\ }

\newcommand{\comm}{\mathrm{comm}}
\newcommand{\comp}{\mathrm{comp}}
\newcommand{\idle}{\mathrm{idle}}
\newcommand{\msg}{\mathrm{msg}}
\newcommand{\len}{\mathrm{s}}
\newcommand{\route}{\mathrm{route}}
\newcommand{\mitem}{\medskip\item}
\newcommand{\sitem}{\smallskip\item}
\DeclareMathOperator*{\argmin}{argmin}

%Gets rid of headline

\setbeamertemplate{footline}{}

\setbeamertemplate{headline}{}
\beamertemplatenavigationsymbolsempty



\definecolor{mygreen}{rgb}{0,0.2,0}
\definecolor{mygray}{rgb}{0.5,0.5,0.5}
\definecolor{mymauve}{rgb}{0.58,0,0.82}
\definecolor{mypurple}{rgb}{0.38,0,0.32}
\definecolor{myblue}{rgb}{0.2,0,0.5}
\definecolor{darkgreen}{rgb}{0.2,0.6,0.2}
\definecolor{Brown}{rgb}{0.39, 0.09, 0.0}
\definecolor{Violet}{rgb}{0.38, 0.0, 0.63}
\definecolor{myvlgray}{rgb}{0.9,0.9,1.0}
\definecolor{mylgray}{rgb}{0.85,0.85,0.9}
\definecolor{darkgreen}{rgb}{0,0.4,0}
\definecolor{brown}{rgb}{.6,.1,.1}

\newcommand{\bemph}[1]{{\color{blue} \emph{#1}}}

\usepackage{mathtools}
\newcommand{\defeq}{\coloneqq}

\newcommand{\inti}[2]{[{#1},{#2}]}
\newcommand{\BF}{\mathbf}
\newcommand{\CF}{\mathcal}
\newcommand{\B}[1]{\bm{#1}}
\newcommand{\E}[1]{\bm{#1}}
\newcommand{\dis}{\displaystyle}
\newcommand{\lt}{\left}
\newcommand{\rt}{\right}

\newcommand{\cs}{H}

\DeclareMathOperator*{\rank}{rank}
\DeclareMathOperator*{\vecn}{vec}
\DeclareMathOperator*{\vecs}{vech}

\usepackage{eso-pic}
\newcommand{\cornertext}[1]{
  \AddToShipoutPictureFG*{
    \AtPageUpperLeft{\put(0,-10){\makebox[\paperwidth][r]{#1}}}  
   }%
}
\newcommand{\cornertexttwo}[2]{
  \AddToShipoutPictureFG*{
    \AtPageUpperLeft{\put(0,-10){\makebox[\paperwidth][r]{#1}}}  
    \AtPageUpperLeft{\put(0,-20){\makebox[\paperwidth][r]{#2}}}  
   }%
}
\newcommand{\linkdemo}[2]{{\footnotesize\href{https://relate.cs.illinois.edu/course/cs450-f18/f/demos/upload/#1/#2.html}{\color{Violet}{\it\textbf{Demo:} #2}\ }}}
\newcommand{\linkinclass}[2]{{\footnotesize\href{https://relate.cs.illinois.edu/course/cs450-f18/flow/#1/start/}{\color{darkgreen}{\it\textbf{Activity:} #2}\ }}}
\newcommand{\urcornerlinkinclass}[2]{\cornertext{\linkinclass{#1}{#2}}}
\newcommand{\dblurcornerlinkinclass}[4]{\cornertexttwo{\linkinclass{#1}{#2}}{\linkinclass{#3}{#4}}}
\newcommand{\urcornerlinkdemo}[2]{\cornertext{\linkdemo{#1}{#2}}}
\newcommand{\dblurcornerlinkdemo}[4]{\cornertexttwo{\linkdemo{#1}{#2}}{\linkdemo{#3}{#4}}}
\newcommand{\urcornerlinkdemoinclass}[4]{\cornertexttwo{\linkdemo{#1}{#2}}{\linkinclass{#3}{#4}}}



% CS 554 specific:

\newcommand{\tsync}{\alpha}
\newcommand{\tword}{\beta}
\newcommand{\tflop}{\gamma}

\newcommand{\bw}{w}

\newcommand{\tpl}[2]{{\bm{#1}}}
\newcommand{\vtpl}[3]{{\bm{#1}}_#3}
\newcommand{\perm}[2]{\lt[#1\rt]_{#2}}
\newcommand{\costyle}{\ttfamily\bfseries}
\newcommand{\cotext}[1]{{\costyle{#1}}}
\newcommand{\kwstyle}{\costyle\textcolor{mypurple}}
\newcommand{\kwtext}[1]{{\kwstyle{#1}}}
\newcommand{\emstyle}{\costyle\textcolor{myblue}}
\newcommand{\emtext}[1]{{\emstyle{#1}}}
\newcommand{\const}{}
\newcommand{\work}{Q}
\newcommand{\depth}{D}
\newcommand{\flops}{F}
\newcommand{\words}{W}
\newcommand{\syncs}{S}
\newcommand{\mem}{M}
\newcommand{\eff}{E}
\newcommand{\isofun}[1]{\tilde{\work}(p)}
\newcommand{\isomem}[1]{\tilde{\mem}(p)}
\newcommand{\pstrong}{p_s}
\newcommand{\pweak}{p_w}
\newcommand{\ALL}{\star }
\newcommand{\rmn}[2]{\mathbb{R}^{#1\times #2}}
\newcommand{\rn}[1]{\mathbb{R}^{#1}}
\newcommand{\err}{\varepsilon}
\newcommand{\mat}[1]{\begin{bmatrix} #1 \end{bmatrix}}
\newcommand{\mc}[1]{\mathcal{#1}}
\newcommand{\h}[2]{\mc{H}_{#1}(#2)}
\newcommand{\T}{T}%{\mathsf{T}}
\newcommand{\dn}[2]{\mc{M}_{#1}^{\uparrow}(#2)}
%\usepackage[colorlinks=false,urlbordercolor={1.0 1.0 1.0}]{hyperref}
%
%\lstset{ %
%  postbreak=false,
%%  backgroundcolor=\color{white},   % choose the background color; you must add \usepackage{color} or \usepackage{xcolor}
%  basicstyle=\costyle,        % the size of the fonts that are used for the code
%%  breakatwhitespace=false,         % sets if automatic breaks should only happen at whitespace
%%  breaklines=false,                 % sets automatic line breaking
%  captionpos=n,                    % sets the caption-position to bottom
%  commentstyle=\color{mygreen},    % comment style
%  deletekeywords={...},            % if you want to delete keywords from the given language
%  escapeinside={\%*}{*)},          % if you want to add LaTeX within your code
%  extendedchars=true,              % lets you use non-ASCII characters; for 8-bits encodings only, does not work with UTF-8
%  frame=none,                    % adds a frame around the code
%  keepspaces=true,                 % keeps spaces in text, useful for keeping indentation of code (possibly needs columns=flexible)
%  keywordstyle=\color{mypurple},       % keyword style
%  language=C++,                 % the language of the code
%  otherkeywords={*,...},            % if you want to add more keywords to the set
%  numbers=none,                    % where to put the line-numbers; possible values are (none, left, right)
%  numbersep=5pt,                   % how far the line-numbers are from the code
%  numberstyle=\tiny\color{mygray}, % the style that is used for the line-numbers
%  rulecolor=\color{black},         % if not set, the frame-color may be changed on line-breaks within not-black text (e.g. comments (green here))
%  showspaces=false,                % show spaces everywhere adding particular underscores; it overrides 'showstringspaces'
%  showstringspaces=false,          % underline spaces within strings only
%  showtabs=false,                  % show tabs within strings adding particular underscores
%  stepnumber=2,                    % the step between two line-numbers. If it's 1, each line will be numbered
%  stringstyle=\color{mymauve},     % string literal style
%  tabsize=2,                     % sets default tabsize to 2 spaces
%  title=\lstname,                   % show the filename of files included with \lstinputlisting; also try caption instead of title
%  keywordstyle=\emstyle
%}

\lstset{ %
  postbreak=false,
%  backgroundcolor=\color{white},   % choose the background color; you must add \usepackage{color} or \usepackage{xcolor}
  basicstyle=\footnotesize\costyle,        % the size of the fonts that are used for the code
%  breakatwhitespace=false,         % sets if automatic breaks should only happen at whitespace
%  breaklines=false,                 % sets automatic line breaking
  captionpos=n,                    % sets the caption-position to bottom
  commentstyle=\color{mygreen},    % comment style
  deletekeywords={privatei,shared},            % if you want to delete keywords from the given language
  escapeinside={\%*}{*)},          % if you want to add LaTeX within your code
  extendedchars=true,              % lets you use non-ASCII characters; for 8-bits encodings only, does not work with UTF-8
  frame=none,                    % adds a frame around the code
  keepspaces=true,                 % keeps spaces in text, useful for keeping indentation of code (possibly needs columns=flexible)
  keywordstyle=\footnotesize\color{mypurple},       % keyword style
  language=C++,                 % the language of the code
  otherkeywords={*,pragma},            % if you want to add more keywords to the set
  numbers=none,                    % where to put the line-numbers; possible values are (none, left, right)
  numbersep=5pt,                   % how far the line-numbers are from the code
  numberstyle=\footnotesize\color{mygray}, % the style that is used for the line-numbers
  rulecolor=\color{black},         % if not set, the frame-color may be changed on line-breaks within not-black text (e.g. comments (green here))
  showspaces=false,                % show spaces everywhere adding particular underscores; it overrides 'showstringspaces'
  showstringspaces=false,          % underline spaces within strings only
  showtabs=false,                  % show tabs within strings adding particular underscores
  stepnumber=2,                    % the step between two line-numbers. If it's 1, each line will be numbered
  stringstyle=\footnotesize\color{mymauve},     % string literal style
  tabsize=2,                     % sets default tabsize to 2 spaces
  title=\lstname,                   % show the filename of files included with \lstinputlisting; also try caption instead of title
  emph={MPI_Send,MPI_Recv,MPI_Comm,MPI_Comm_rank,MPI_Send,MPI_Comm_split,MPI_Init,MPI_Finalize,MPI_Status,MPI_COMM_WORLD,MPI_Datatype,MPI_Comm_size,MPI_FLOAT,mpi.h,omp,parallel,shared,private,default,reduction},
  emphstyle=\footnotesize\emstyle
}

\title{CS 450: Numerical Anlaysis\footnote{{\it These slides have been drafted by Edgar Solomonik as lecture templates and supplementary material for the book ``Scientific Computing: An Introductory Survey'' by Michael T. Heath (\href{http://heath.cs.illinois.edu/scicomp/notes/index.html}{{\color{blue} slides}}).}}}
\author{}
\institute{University of Illinois at Urbana-Champaign}




\subtitle{Linear Systems}
\date{}
\begin{document}

\begin{frame}
  \titlepage
\end{frame}

\section{Vector and Matrix Norms}

\begin{frame}{Vector Norms}

\urcornerlinkdemo{01-error-and-fp}{Vector Norms}

\begin{itemize}
\item {\bf Properties of vector norms}
  \mdcond{ 
    \begin{align*}
    ||\B x|| &\geq 0 \\
    ||\B x|| &= 0  \quad\Leftrightarrow \quad\B x = \B 0 \\
    ||\alpha \B x|| &= |\alpha|\cdot ||\B x|| \\
    ||\B x + \B y|| &\leq ||\B x|| + ||\B y||\quad \text{(\coloremph{triangle inequality}) implies continuity}
    \end{align*}
    }
\item {\bf A norm is uniquely defined by its unit sphere:}
  \mdcond{
    Surface defined by space of vectors $\mathbb{V}\subset \mathbb{R}^n$ such that $\forall \B x\in\mathbb{V}, ||\B x||=1$
  }
\item {\bf $p$-norms}
  \mdcond{ 
    $||\B x||_p = \bigg(\sum_{i} |x_i|^p\bigg)^{1/p}$
    \begin{itemize}
    \sitem $p=1$ gives sum of absolute values of entry (unit sphere is diamond-like)
    \sitem $p=\infty$ gives maximum entry in absolute value (unit sphere is box-like)
    \sitem $p=2$ gives Euclidean distance metric (unit sphere is spherical)
    \end{itemize}
  }
  

\end{itemize}

\end{frame}

\begin{frame}{Inner-Product Spaces}

\begin{itemize}
\item {\bf Properties of inner-product spaces}:
%
%  \lgcond{ 
    Inner products $\langle \B x, \B y \rangle$ must satisfy
    \begin{align*}
    \langle \B x, \B x\rangle  &\geq 0 \\
    \langle \B x, \B x\rangle &= 0  \quad\Leftrightarrow \quad\B x = \B 0 \\
    \langle \B x, \B y \rangle & =\langle \B y, \B x \rangle \\
    \langle \B x, \B y + \B z \rangle & =\langle \B x, \B y \rangle + \langle \B x, \B z \rangle\\
    \langle \alpha \B x, \B y \rangle & =\alpha \langle \B x, \B y \rangle 
    \end{align*}
%  }

\item {\bf Inner-product-based vector norms}

\lgcond{
The $p=2$ vector norm is the Eucledian inner-product norm,
  $$||\B x||_2=\sqrt{\B x^\T \B x}$$
and due to Cauchy-Schwartz inequality $|\langle \B x, \B y\rangle|\leq \sqrt{\langle \B x, \B x\rangle\cdot \langle \B y, \B y\rangle}$,
  $$|\B x^T\B y| \leq ||\B x||_2||\B y||_2.$$
  Other inner-products can be expressed as $\langle \B x, \B y \rangle=\B x^\T\B A\B y$ where $\B A$
  is symmetric positive definite, yielding norms $||\B x||_{\B A} = \sqrt{\B x^\T \B A \B x}$
}


\end{itemize}
\end{frame}


\begin{frame}{Matrix Norms}

\urcornerlinkdemo{02-linear-systems}{Matrix norms}

\begin{itemize}
\item {\bf Properties of matrix norms}:
  %\lgcond{ 
    \begin{align*}
    ||\B A|| &\geq 0 \\
    ||\B A|| &= 0  \quad\Leftrightarrow \quad\B A = \B 0 \\
    ||\alpha \B A|| &= |\alpha|\cdot ||\B A|| \\
    ||\B A + \B B|| &\leq ||\B A|| + ||\B B||\quad \text{(\coloremph{triangle inequality})}
    \end{align*}
   % }
\item {\bf Frobenius norm}:
    \smcond{
      $$||\mathbf{A}||_F=\bigg(\sum_{i,j} a_{ij}^2\bigg)^{1/2}$$
    }
\item {\bf Operator/induced/subordinate matrix norms}:

\lgcond{
  For any vector norm $||\cdot ||$, the induced matrix norm is
  \[||\B A|| = \max_{\B x\neq \B 0} ||\B A \B x||/||\B x|| = \max_{||\B x||=1} ||\B A \B x||\]

}

\end{itemize}
\end{frame}

\begin{frame}{Induced Matrix Norms}

\begin{itemize}
\item {\bf Interpreting induced matrix norms}:
  \lgcond{
  A matrix is uniquely defined with respect to a norm by a unit-ball, which is the space of vectors $\B y = \B A\B x$ for all $\B x$ on the unit-sphere of the norm.

  $$||\B A||_{p}=\max_{||\B x||_p=1} ||\B A \B x||_p$$ is the maximum possible $p$-norm
  \coloremph{amplification} due to application of $\B A$ 
%(radius of unit-sphere enclosing unit-ball)

  
  $$1/||\B A^{-1}||_{p}=\min_{||\B x||_p=1} ||\B A \B x||_p$$ is the maximum possible $p$-norm
  \coloremph{reduction} due to application of $\B A$ 
%(radius of unit-sphere enclosed by unit-ball)
  }
\item {\bf General induced matrix norms}:
  \lgcond{
  $$||\B A||_{mp}=\max_{||\B x||_p=1} ||\B A \B x||_m$$
  typically $m=p$ so we write $||\B A||_{p}$ and almost always we have $p\in\{1,2,\infty\}$.
  (Computing the matrix norm for certain choices of $m\neq p$ is NP-complete.)
  }


\end{itemize}
\end{frame}

\section{Matrix Condition Number}

\begin{frame}{Matrix Condition Number}

\dblurcornerlinkdemo{02-linear-systems}{Conditioning of 2x2 Matrices}{02-linear-systems}{Condition number visualized}

\begin{itemize}
\item {\bf Definition}: $\kappa(\B A) = ||\B A|| \cdot||\B A^{-1}||$ is the ratio between the shortest/longest distances from the unit-ball center to any point on the surface.
\mitem {\bf Intuitive derivation}:
\begin{align*}
\kappa(\B A) &= \max_{\text{inputs}} \quad \max_{\text{perturbations in input}}\left|\frac{\text{relative perturbation in output}}{\text{relative perturbation in input}}\right| 
\end{align*}
since a matrix is a linear operator, we can decouple its action on
the input $\B x$ and the perturbation $\B{\delta x}$ since $\B A(\B x +\B{\delta x}) = \B A \B x + \B A \B{\delta x}$, so
%&= \left|\frac{\max_{\text{perturbations in input}}\text{relative perturbation growth}}{\max_{\text{inputs}}\text{relative input reduction}}\right| \\
\begin{align*}
\kappa(\B A) &= \left|\frac{\overbrace{\max_{\text{perturbations in input}}\text{relative perturbation growth}}^{||\B A||}}{\underbrace{\max_{\text{inputs}} \text{relative input reduction}}_{1/||\B A^{-1}||}}\right| 
\end{align*}
\end{itemize}
\end{frame}



\begin{frame}{Matrix Conditioning}

\begin{itemize}
\item The matrix condition number $\kappa(\B A)$ is the ratio between the max and min distance from the surface to the center of the unit ball transformed by $\kappa(\B A)$:
\lgcond{
\begin{itemize} 
\item The max distance to center is given by the vector maximizing $\max_{||\B x||=1} ||\B A\B x||_2$.
\sitem The min distance to center is given by the vector minimizing $\min_{||\B x||=1} ||\B A\B x||_2 = 1/(\max_{||\B x||=1} ||\B A^{-1}\B x||_2)$.
\sitem Thus, we have that $\kappa(\B A) = ||\B A||_2||\B A^{-1}||_2$
\end{itemize}
}
\item The matrix condition number bounds the worst-case amplification of error in a matrix-vector product:
  \lgcond{
  Consider $\B y + \B{\delta y}= \B A (\B x + \B{\delta x})$, assume $||\B x||_2=1$
  \begin{itemize}
  \mitem In the worst case, $||\B y||_2$ is minimized, that is $||\B y||_2 = 1/||\B A^{-1}||_2$
  \mitem In the worst case, $||\B{\delta y}||_2$ is maximized, that is $||\B{\delta y}||_2 = ||\B A||_2||\B{\delta y}||_2$
  \mitem So $||\B{\delta y}||_2/||\B{y}||_2$ is at most $\kappa(\B A)||\B{\delta x}||_2/||\B{x}||_2$
  \end{itemize}
  }

\end{itemize}

\end{frame}



\begin{frame}{Norms and Conditioning of Orthogonal Matrices}

\begin{itemize}
\item {\bf Orthogonal matrices}:
  \mdcond{
  A matrix $\B Q$ is orthogonal, if its square and its columns are orthonormal,
  %\[ \B Q =\begin{bmatrix}\B {q_1} & \cdots & \B{q_2}\end{bmatrix}, \quad \langle \B{q_i},\B{q_j}\rangle =\delta_{i,j} = \begin{cases} 1 & : i=j \\ 0 & :i\neq j \end{cases}\]
  or equivalently $\B Q^\T= \B Q^{-1}$.
  }
\item {\bf Norm and condition number of orthogonal matrices}:
  \smcond{}
  \lgcond{
  For any $||\B v||_2=1$,
  \begin{align*}
    ||\B Q\B v||_2 &= \bigg( \Big\langle\B v^\T \B Q^\T,  \B{Q} \B v\Big\rangle\bigg)^{1/2}
                   = \bigg( \B v^\T \B Q^\T  \B{Q} \B v\bigg)^{1/2}  =\bigg( \B v^\T \B v\bigg)^{1/2} \\
                   &= ||\B v||_2
  \end{align*}
  Consequently, $||\B Q||_2=||\B Q^{-1}||_2 = \kappa(\B Q) = 1$.

  \smallskip
  $\B Q \B v$ expresses $\B v$ in a coordinate system whose axes
  are columns of $\B Q^T$
  }

\end{itemize}

\end{frame}

\section{Singular Value Decomposition}

\begin{frame}{Singular Value Decomposition}

\begin{itemize}
\item {\bf The singular value decomposition (SVD)}:

  \lgcond{
    We can express \coloremph{any} matrix $\B A$ as
    $$\B A = \B U \B \Sigma \B V^\T$$
    where $\B U$ and $\B V$ are orthogonal, and $\B \Sigma$ is square nonnegative and diagonal,
    \[\B \Sigma = \begin{bmatrix} \sigma_\text{max} & & \\ & \ddots &\\& & \sigma_\text{min}\end{bmatrix}\]
    \coloremph{Any matrix is diagonal when expressed as an operator mapping vectors from a coordinate
    system given by $\B V$ to a coordinate system given by $\B U^\T$}.
  }
  \lgcond{}
    
\end{itemize}
\end{frame}

\begin{frame}{Norms and Conditioning via SVD}

\urcornerlinkinclass{inclass-svdnorm}{Singular Value Decomposition and Norms}

\begin{itemize}
\item {\bf Norm and condition number in terms of singular values}:

  \lgcond{
    When multiplying a vector by matrix $\B A = \B U \B \Sigma \B V^\T$
    \begin{itemize}
    \mitem Multiplication by $\B V^\T$ changes coordinate systems, leaving the norm unchanged
    \mitem Multiplication by $\B U$ changes coordinate systems, leaving the norm unchanged
    \end{itemize}
    \medskip
    so, only multiplication by $\B \Sigma$ has an effect on the vector norm
    \begin{itemize}
    \mitem Note that $||\B \Sigma||_2=\sigma_\text{max}$, $||\B \Sigma^{-1}||_2=1/\sigma_\text{min}$, so 
    \[\kappa(\B A) =\kappa(\B \Sigma)=\frac{\sigma_\text{max}}{\sigma_\text{min}}\]
    \end{itemize}
  }
  \lgcond{}
\end{itemize}
\end{frame}

\section{Perturbation Analysis}

\begin{frame}{Conditioning of Linear Systems}

\begin{itemize}
\item {\bf Lets now return to formally deriving the conditioning of solving $\B A \B x = \B b$}:

  \lgcond{
  Consider a perturbation to the right-hand side (input) $\B {\hat{b}} = \B b + \B{\delta b}$
  \begin{align*}
  \B A \B{\hat{x}} &= \B{\hat{b}} \\
  \B A (\B x  +\B{\delta x}) &= \B{b} + \B{\delta b} \\
  \B A\B{\delta x} &= \B{\delta b}
  \end{align*}
  we wish to bound the size of the relative perturbation to the output $||\B{\delta x}||/||\B{x}||$ with respect to the size of the relative perturbation the the input $||\B{\delta b}||/||\B{b}||$
  \begin{align*}
  \B{\delta x} &= \B A^{-1}\B{\delta b} \\
  \frac{||\B{\delta x}||}{||\B{x}||} &= \frac{||\B A^{-1}\B{\delta b}||}{||\B{x}||} 
  \leq  \frac{||\B A^{-1}||\cdot ||\B{\delta b}||}{||\B{x}||}
%  \frac{||\B{\delta b}||}{||\B{b}||} =  \frac{||\B A\B{\delta x}||}{||\B{b}||} \leq  \frac{||\B A||\cdot ||\B{\delta x}||}{||\B{b}||}
%  \frac{||\B{\delta b}||}{||\B{b}||} =  \frac{||\B A\B{\delta x}||}{||\B{b}||} \leq  \frac{||\B A||\cdot ||\B{\delta x}||}{||\B{b}||}
  \end{align*}
  we can use that $||\B{x}|| \geq ||\B{b}||/\sigma_\text{max}= ||\B{b}||/||\B{A}||$ so
  \begin{align*}
  \frac{||\B{\delta x}||}{||\B{x}||} \leq  \underbrace{||\B A||\cdot ||\B A^{-1}||}_{\kappa(\B A)}\cdot \frac{||\B{\delta b}||}{||\B{b}||} = \frac{\sigma_\text{max} ||\B{\delta b}||}{\sigma_\text{min} ||\B{b}||} 
  \end{align*}
  }
  \lgcond{}
\end{itemize}
\end{frame}

\begin{frame}{Conditioning of Linear Systems II}

\begin{itemize}
\item {\bf Consider perturbations to the input coefficients $\B {\hat{A}} = \B{A} + \B{\delta A}$}:

  \lgcond{
  In this case we solve the perturbed system 
  \begin{align*}
  \B{\hat{A}} \B{\hat{x}} &= \B{b} \\
  \B A \B x  +\B{\delta \B A} \B x &= \B{b} - \B{\hat{A}} \B{\delta x} \\
  \B{\delta A}\B x &=  -\B{\hat{A}} \B{\delta x} \approx - \B{A} \B{\delta x}
  \end{align*}
  we wish to bound the size of the relative perturbation to the output $||\B{\delta x}||/||\B{x}||$ with respect to the size of the relative perturbation the the input $||\B{\delta A}||/||\B{A}||$
  \begin{align*}
  \B{\delta x} &= -\B A^{-1}\B{\delta A}\B{x} \\
  ||\B{\delta x}|| &= ||\B A^{-1}\B{\delta A}\B{x}|| \leq ||\B A^{-1}||\cdot ||\B{\delta A}|| \cdot ||\B{x}|| \\
  \frac{||\B{\delta x}||}{||\B{x}||} &\leq \underbrace{||\B A^{-1}||\cdot ||\B {A}||}_{\kappa(\B A)} \cdot \frac{||\B{\delta A}||}{||\B A||}
  \end{align*}
  }
  \lgcond{
  }
\end{itemize}
\end{frame}

\section{Algorithms for Simple Linear Systems}

\begin{frame}[fragile]{Solving Basic Linear Systems}

\begin{itemize}
\item Solve $\B D \B x = \B b$ if $\B D$ is diagonal 

    \smcond{$x_i = b_i/d_{ii}$ with total cost $O(n)$}

\item Solve $\B Q \B x = \B b$ if $\B Q$ is orthogonal

    \smcond{$\B x= \B Q^\T \B b$ with total cost $O(n^2)$}

\item Given SVD $\B A = \B U \B \Sigma \B V^T$, solve $\B A \B x = \B b$

    \mdcond{
    \begin{itemize}
    \item Compute $\B z = \B U^T\B b$
    \item Solve $\B \Sigma \B y = \B z$ (diagonal)
    \item Compute $\B x = \B V \B x$
    \end{itemize}
    }

\end{itemize}
\end{frame}


\begin{frame}{Solving Triangular Systems}

\urcornerlinkdemo{02-linear-systems}{Coding back-substitution}

\begin{itemize}
\item {\bf $\B L \B x = \B b$ if $\B L$ is lower-triangular is solved by forward substitution:}
    \filufil{
    \begin{alignat*}{4}
    l_{11}x_1 &= b_1 && &&x_{1} &&= b_1/l_{11} \\
    l_{21}x_1 + l_{22}x_2 &=b_2 && \quad\Rightarrow \quad&& x_{2} &&= (b_2-l_{21}x_1)/l_{22} \\
    l_{31}x_1 + l_{32}x_2 + l_{33}x_3 &=b_3 && && x_{3} &&= (b_3-l_{31}x_1-l_{32}x_2)/l_{33} \\
          &\ \ \vdots && &&  && \ \ \vdots
    \end{alignat*}
    }{
    \begin{alignat*}{4}
    l_{11}x_1 &= b_1 && &&x_{1} &&= \quad\quad\quad  \quad\quad\quad\quad\quad\quad\\
    l_{21}x_1 + l_{22}x_2 &=b_2 && \quad\Rightarrow \quad&& x_{2} &&= \hfill \\
    l_{31}x_1 + l_{32}x_2 + l_{33}x_3 &=b_3 && && x_{3} &&= \hfill  \\
          &\ \ \vdots && &&  && \ \ \vdots
    \end{alignat*}
  }
\item {\bf Algorithm can also be formulated recursively by blocks:}
   
  \lgcond{ 
    \[\blomat{l} \begin{bmatrix} x_1 \\ \B{x_2} \end{bmatrix} = \begin{bmatrix} b_1 \\ \B{b_2}\end{bmatrix}\]
    $x_1=b_1/l_{11}$, then solve recursively for $\B{x_2}$ in $\B{L_{22}}\B{x_2} = \B{b_2}-\B{l_{21}}x_1$.
  }
\end{itemize}
\end{frame}

\begin{frame}{Solving Triangular Systems}

\begin{itemize}

\item {\bf Existence of solution to $\B L \B x =\B b$:}

    \mdcond{
      If some $l_{ii}=0$, the solution may not exist, and $\B L^{-1}$ does not exist.
    }
\item {\bf Uniqueness of solution:}
    \mdcond{ 
      Even if some $l_{ii}=0$ and $\B L^{-1}$ does not exist, the system may have a solution.
      The solution will not be unique since columns of $\B{L}$ are necessarily linearly dependent if a diagonal element is zero.
      May want to select solution minimizing norm of $\B x$.
    }
\item {\bf Computational complexity of forward/backward substitution:}

  \mdcond{
  The recursive algorithm has the cost recurrence,
  $$T(n)=T(n-1)+n=\sum_{i=1}^{n} i = n(n+1)/2.$$
  The total cost is $n^2/2$ multiplications and $n^2/2$ additions to leading order. 
  }

\end{itemize}
\end{frame}

\begin{frame}{Properties of Triangular Matrices}

\begin{itemize}
\item {\bf $\B Z = \B X \B Y$ is lower triangular is $\B X$ and $\B Y$ are both lower triangular:}

  \lgcond{
    \[\bomat{z}=\blomat{x}\blomat{y}.\]
    Clearly, $z_{11}=x_{11}y_{11}$ and $\B z_{12}=0$, then we proceed by the same argument for the triangular matrix product 
    $\B{Z}_{22}=\B{X}_{22}\B{Y}_{22}$.
  }

\item {\bf $\B{L^{-1}}$ is lower triangular if it exists:}
    
  \lgcond{
    We give a constructive proof by providing an algorithm for triangular matrix inversion.
    We need $\B{Y}=\B{X}^{-1}$ so 
    \[\blmat{Y}\blmat{X} = \begin{bmatrix} \B I & \\ & & \B I \end{bmatrix},\]
    from which we can deduce
    \[\B{Y}_{11} = \B{X}_{11}^{-1},\quad \B{Y}_{22}=\B{X}_{22}^{-1},\quad \B{Y}_{21} = -\B{Y}_{22}\B{X}_{21}\B{Y}_{11}.\]
  }


\end{itemize}
\end{frame}



\section{Gaussian Elimination}

\subsection{LU Decomposition}

\begin{frame}{LU Factorization}

\begin{itemize}
\item {\bf An \coloremph{LU factorization} consists of a unit-diagonal lower-triangular \coloremph{factor} $\B L$ and upper-triangular factor $\B U$ such that $\B  A = \B L \B U$:}

\lgcond{
\begin{itemize}
\mitem Unit-diagonal implies each $l_{ii}=1$, leaving $n(n-1)/2$ unknowns in $\B L$ and $n(n+1)/2$ unknowns in $\B U$, for a total of $n^2$, the same as the size of $\B A$.
%\mitem The latter is done by forward substitution, the former involves an upper-triangular matrix and requires backward substitution
%\mitem It suffices to have a forward substitution routine and reversing the ordering of vector elements, 
%      done by taking the product with $\B P$, $p_{ij} = \delta(i,n+1-j)$,
%    \[\B L \B U = \B L \B P \underbrace{\B P \B U \B P}_{\B{\tilde L}} \B P = \B L \B P \B{\tilde L} \B P\]
\mitem For rectangular matrices $\B{A}\in \mathbb{R}^{m\times n}$, one can consider a full LU factorization, 
      with $\B{L}\in \mathbb{R}^{m \times \max(m,n)}$ and $\B{U} \in \mathbb{R}^{\max(m,n) \times n}$, 
      but it is fully described by a reduced LU factorization, with lower-trapezoidal
      $\B{L}\in \mathbb{R}^{m \times \min(m,n)}$ and upper-trapezoidal $\B{U} \in \mathbb{R}^{\min(m,n) \times n}$.
\end{itemize}
}
\item {\bf Given an LU factorization of $\B A$, we can solve the linear system $\B A \B x = \B b$:} 
\lgcond{
    \begin{itemize} 
    \sitem using forward substitution $\B L \B y = \B b$
    \sitem using backward substitution to solve $\B U \B x = \B y$
    \end{itemize}
\smallskip
    Backward substitution is the same as forward substitution with a reversal of the ordering of the elements of the vectors and the ordering of the rows/columns of the matrix.
}

\lgcond{}

\end{itemize}
\end{frame}

\subsection{Naive Algorithm}

\begin{frame}{Gaussian Elimination Algorithm}

\urcornerlinkdemo{02-linear-systems}{LU factorization}

\begin{itemize}
\item {\bf Algorithm for factorization is derived from equations given by $\B A = \B L \B U$:}

\lgcond{
%\[\bmat{A} = \blmat{L}\brmat{U}\]
\[\begin{bmatrix} a_{11} & \B{a}_{12} \\ \B{a}_{21} & \B{A}_{22} \end{bmatrix}=
\begin{bmatrix} 1 &  \\ \B{l}_{21} & \B{L}_{22} \end{bmatrix}
\begin{bmatrix} u_{11} & \B{u}_{12} \\  & \B{U}_{22} \end{bmatrix}
 = \blmat{L}\brmat{U}\]
\begin{itemize}
%\item Obtain LU factorization of the leading minor \(\B{A}_{11}=\B L_{11}\B{U}_{11}\) by recursion
\sitem First, observe $\begin{bmatrix} u_{11} & \B{u}_{12}\end{bmatrix} = \begin{bmatrix}{a}_{11} & \B{a}_{12}\end{bmatrix}$
\sitem To obtain $\B{l}_{21}$ compute $\B{l}_{21} = \B{a}_{21}/u_{11}$
%\item Solve  sets of triangular linear systems can be solved to obtain $\B L_{21}$ and $\B{U}_{12}$ from $\B{A}_{21}=\B{L}_{21}\B{U}_{11}$ and $\B{A}_{12}=\B{L}_{11}\B{U}_{12}$
\sitem 
Obtain $\B{L}_{22}$ and $\B{U}_{22}$ by recursively computing LU of the \coloremph{Schur complement} \[\B{S}= \B{A}_{22}-\B{l}_{21}\B{u}_{12}\] 
\end{itemize}
}
\item {\bf The computational complexity of LU is $O(n^3)$:}

\lgcond{
Computing $\B{l}_{21} = \B{a}_{21}/u_{11}$ requires $O(n)$ operations, finding $\B S$ requires $2n^2$, so to leading order the complexity of LU is
$$T(n)=T(n-1)+2n^2=\sum_{i=1}^n 2i^2 \approx 2n^3/3$$ 
}
%\item {\bf The $k$th column of $\B L$ is given by the $k$th \coloremph{elementary matrix} $\B{M}_k$:}
%
%\lgcond{
%\[\B{M}_k\begin{bmatrix} v_1 & \cdots & v_k & 0 & \cdots & 0 \end{bmatrix}^\T= \B{v} \]
%}

\end{itemize}
\end{frame}

\subsection{Existence of LU}

\begin{frame}{Existence of LU Factorization}

\begin{itemize}
\item {\bf The LU factorization may not exist:}
Consider matrix $\begin{bmatrix} 3 & 2 \\ 6 & 4 \\ 0 & 3\end{bmatrix}$.

\lgcond{
%We can infer what the first row of $\B L$ and column of $\B U$ directly from the matrix,
Proceeding with Gaussian elimination we obtain
\[\begin{bmatrix} 3 & 2 \\ 6 & 4 \\ 0 & 3\end{bmatrix} =\begin{bmatrix} 1 & 0 \\ 2 & 1 \\ 0 & l_{32} \end{bmatrix} \begin{bmatrix} 3 & 2 \\ 0 & u_{21} \end{bmatrix}.\]
Then we need that $4 = 4+u_{21}$ so $u_{21}=0$, but at the same time $l_{32}u_{21}=3$.

More generally, if and only if for any partitioning $\bmat{A}$ the leading minor is singular ($\det(\B{A}_{11})=0$), $\B{A}$ has no LU factorization.
}

\item {\bf Permutation of rows enables us to transform the matrix so the LU factorization does exist:}

\lgcond{
Gaussian elimination can only fail if dividing by zero. 
At every recursive step of Gaussian elimination, if the leading entry of the first row is zero, we permute it with a row with an leading nonzero (if $\B{a}_{21}=\B 0$, we set $u_{11}=0$ and $\B{l}_{21}=\B 0$).
}

\end{itemize}
\end{frame}
%\begin{frame}{Elimination Matrices}
%
%\begin{itemize}
%\item {\bf An elimination matrix $\B{M}_k$ satisfies the following properties:}
%
%\lgcond{
%\begin{itemize}
%\item It is a rank-$1$ perturbation of the identity that is unit-diagonal and lower-triangular,
%\[ \B{M}_k = \B{I} - \B{m}_k \B{e}_k^\T=\B{I}-\begin{bmatrix} \B{\tilde{m}}_k \\ \B{0}\end{bmatrix} \B{e}_k^\T\]
%\item It reduces a given vector to its first $k$ elements \[\B{M}_k\begin{bmatrix} a_1 \\ \vdots \\ a_k \\ 0 \\ \vdots\end{bmatrix}= \B{a} \]
%\item $\B{M}_k^{-1}= \B{I}+\B{m}_k \B{e}_k^\T = 2\B{I} - \B{M}_k$
%\item $\B{M}_j\B{M}_k=(\B{I} - \B{m}_j \B{e}_j^\T)(\B{I} - \B{m}_k \B{e}_k^\T)=(\B{I} - \begin{bmatrix} \B{m}_j & \B{m}_k \end{bmatrix}\begin{bmatrix} \B{e}_j & \B{e}_k\end{bmatrix}^\T)=\B{M}_j+\B{M}_k-\B I$
%\end{itemize}
%}
%
%\lgcond{}
%
%\end{itemize}
%\end{frame}
%

\subsection{Pivoting}

\begin{frame}{Gaussian Elimination with Partial Pivoting}

\urcornerlinkdemo{02-linear-systems}{LU with Partial Pivoting}

\begin{itemize}
\item {\bf \coloremph{Partial pivoting} permutes rows to make divisor $u_{ii}$ is maximal at each step:}

\lgcond{
Based on our argument above, for any matrix $\B A$ there exists a permutation matrix $\B P$ that
can permute the rows of $\B A$ to permit an LU factorization,
\[\B P \B A = \B L \B U. \]
Partial pivoting finds such a permutation matrix $\B P$ one row at a time.
The $i$th row is selected to maximize the magnitude of the leading element (over elements in the first column),
which becomes the entry $u_{ii}$.
This selection ensures that we are never forced to divide by zero during Gaussian elimination and that the magnitude
of any element in $\B L$ is at most $1$.
}

\item {\bf A row permutation corresponds to an application of a \coloremph{row permutation matrix} $\B{P}_{jk} = \B{I} -(\B{e}_j-\B{e}_k)(\B{e}_j-\B{e}_k)^\T$:}

\lgcond{
If we permute row $i_j$ .o be the leading ($i$th) row at the $i$th step, the overall permutation matrix is given by
\(\B P^T =\prod_{i=1}^{n-1} \B{P}_{ii_j}.\)
}

\end{itemize}
\end{frame}

\begin{frame}{Partial Pivoting Example}

\begin{itemize}
\item Lets consider again the matrix $\B A = \begin{bmatrix} 3 & 2 \\ 6 & 4 \\ 0 & 3\end{bmatrix}$.

\lgcond{
\begin{itemize}
\item The largest magnitude element in the first column is $6$, so we select this as our pivot and perform the first step of LU
\[\underbrace{\begin{bmatrix} & 1 & \\ 1 & & \\ & & 1 \end{bmatrix}}_{\B P_1} \begin{bmatrix} 6 & 4 \\ 3 & 2 \\ 0 & 3\end{bmatrix} = \begin{bmatrix} 1 \\ 1/2 \\ 0 \end{bmatrix} \begin{bmatrix} 6 & 4 \end{bmatrix} + \begin{bmatrix} 0 & 0  \\ 0 & 2 - (1/2)\cdot 4 \\ 0 & 3 - 0\cdot 4 \end{bmatrix}\]
\item The Schur complement is $\begin{bmatrix} 0 & 3 \end{bmatrix}^T$ and we proceed with pivoted LU, %to obtain its (reduced) LU factorization, we pivot again
\[\underbrace{\begin{bmatrix} & 1 \\ 1 & \end{bmatrix}}_{\B P_2} \begin{bmatrix} 0 \\ 3\end{bmatrix} = \begin{bmatrix} 1 \\ 0 \end{bmatrix} \begin{bmatrix} 3 \end{bmatrix} \]
\item The overall LU factorization is then given by
\(\B P_1 \begin{bmatrix} 1 & \\ & \B P_2 \end{bmatrix} \B A = \begin{bmatrix} 1 & \\ 0 & 1 \\ 1/2 & 0\end{bmatrix}\begin{bmatrix} 6 & 4 \\ & 3 \end{bmatrix}\)
\end{itemize}
}
\lgcond{
}
\end{itemize}
\end{frame}


\begin{frame}{Complete Pivoting}

\begin{itemize}
\item {\bf \coloremph{Complete pivoting} permutes rows and columns to make divisor $u_{ii}$ is maximal at each step:}

\lgcond{
\begin{itemize}
\sitem Partial pivoting ensures that the magnitude of the \coloremph{multipliers} satisfies $|\B l_{21}|= |\B a_{21}|/|u_{11}|\leq \B 1$
\sitem Complete pivoting also gives $||\B u_{12}||_\infty\leq |u_{11}|$ and consequently $|\B l_{21}|\cdot ||\B u_{12}||_\infty = |\B a_{21}|\cdot ||\B u_{12}||_\infty/|u_{11}|\leq |\B a_{21}|$
\sitem Complete pivoting yields a factorization of the form $\B L \B U = \B P \B A \B Q$ where $\B P$ and $\B Q$ are permutation matrices
\end{itemize}
}

\item {\bf Complete pivoting is noticeably more expensive than partial pivoting:}
\lgcond{
\begin{itemize}
\sitem Partial pivoting requires just $O(n)$ comparison operations and a row permutation
\sitem Complete pivoting requires $O(n^2)$ comparison operations, which somewhat increases the leading order cost of LU overall
\end{itemize}
}
\end{itemize}
\end{frame}

\begin{frame}{Round-off Error in LU}

\begin{itemize}
\item {\bf Lets consider factorization of $\begin{bmatrix} \epsilon & 1 \\ 1 & 1 \end{bmatrix}$ where $\epsilon<\epsilon_\text{mach}$:}

\lgcond{
\begin{itemize}
\item Without pivoting we would compute 
$\B L = \begin{bmatrix}1  & 0 \\ 1/\epsilon & 1 \end{bmatrix}$,
$\B U = \begin{bmatrix}\epsilon  & 1 \\ 0 & 1-1/\epsilon \end{bmatrix}$
\sitem Rounding yields $\fl(\B U) = \begin{bmatrix}\epsilon  & 1 \\ 0 & -1/\epsilon \end{bmatrix}$
\sitem This leads to $\B L\fl(\B U) = \begin{bmatrix} \epsilon & 1 \\ 1 & 0 \end{bmatrix}$, a backward error of $\begin{bmatrix} 0 & 0 \\ 0 & 1 \end{bmatrix}$
\end{itemize}
}
\item {\bf Permuting the rows of $\B A$ in partial pivoting gives $\B P \B A = \begin{bmatrix} 1 & 1 \\ \epsilon & 1 \end{bmatrix} $}
\lgcond {
\begin{itemize}
\item We now compute
$\B L = \begin{bmatrix}1  & 0 \\ \epsilon & 1 \end{bmatrix}$,
$\B U = \begin{bmatrix}1  & 1 \\ 0 & 1-\epsilon \end{bmatrix}$, so $\fl(\B U)=\begin{bmatrix}1  & 1 \\ 0 & 1 \end{bmatrix}$
\sitem This leads to $\B L\fl(\B U) = \begin{bmatrix} 1 & 1 \\ \epsilon & 1+\epsilon \end{bmatrix}$, a backward error of $\begin{bmatrix} 0 & 0 \\ 0 & \epsilon \end{bmatrix}$
\end{itemize}
}


\end{itemize}

\end{frame}

\subsection{Error Analysis}

\begin{frame}{Error Analysis of LU}

\begin{itemize}

\item {\bf The main source of round-off error in LU is in the computation of the Schur complement:}

\lgcond{
\begin{itemize}
\sitem Recall that division is well-conditioned, while addition can be ill-conditioned
\sitem After $k$ steps of LU, we are working on Schur complement $\B{A}_{22}-\B{L}_{21}\B{U}_{12}$ where $\B{A}_{22}$ is $(n-k)\times (n-k)$, $\B{L}_{21}$ and $\B{U}_{12}^T$ are $(n-k)\times k$
\sitem Partial pivoting and complete pivoting improve stability by making sure $\B{L}_{21}\B{U}_{12}$ is small in norm
\end{itemize}
}

\item {\bf When computed in floating point, absolute backward error $\B{\delta A}$ in LU (so $\B{\hat L}\B{\hat U}=\B{A}+\B{\delta A}$) is} $|\delta a_{ij}| \leq  \epsilon_\text{mach} (|\B{\hat L}|\cdot |\B{\hat{U}}|)_{ij}$

\lgcond{
%The major source of error is the Schur complement updates.
For any $a_{ij}$ with $j\geq i$ (lower-triangle is similar), we compute 
\[a_{ij}-\sum_{k=1}^i\hat{l}_{ik}\hat{u}_{kj} = a_{ij} - \langle \B{\hat{l}}_i,\B{\hat{u}}_j\rangle,\]
which in floating point incurs round-off error at most $\epsilon_\text{mach} \langle|\B{\hat{l}}_i|,|\B{\hat{u}}_j|\rangle$.
Using this, for complete pivoting, we can show $|\delta a_{ij}| \leq \epsilon_\text{mach} n^2||\B A||_\infty.$
%For an arbitrary $a_{ij}$, consider the partitioning $\bmat{A}$ where $\B{A}_{11}$ is of dimension $\min(i,j)-1$. 
%After the Schur complement update $\B{S}= \fl(\B{A}_{22}-\B{\hat L}_{21}\B{\hat U}_{12})$,
%the entry in $\B S$ corresponding to $a_{ij}$ (an entry in $\B{A}_{22}$) will become an entry of $\B U$ or an entry of $\B{\hat L}$ (after a division that can only shrink the error).
%Thus the $\B{\hat L}$ and $\B{\hat{U}}$ are factors of a matrix $\B{A}+\B{\delta A}$ where the perturbation is bounded by the error of the inner product necessary to compute any Schur complement entry,
%so $|\delta a_{ij}| \leq \epsilon_\text{mach}(|\B{\hat{L}}|\cdot |\B{\hat{U}}|)_{ij}$.
}

\end{itemize}
\end{frame}


\section{Linear System Problem Variants}

\begin{frame}{Helpful Matrix Properties}

\begin{itemize}
\item {\bf  Matrix is \coloremph{diagonally dominant}, so $\sum_{i\neq j} |a_{ij}| \leq |a_{ii}|$:}

\mdcond{
Pivoting is not required if matrix is strictly diagonally dominant $\sum_{i\neq j} |a_{ij}| < |a_{ii}|$.
}
\item {\bf Matrix is \coloremph{symmetric positive definite (SPD)}, so $\forall_{\B x\neq 0}, \B x^T\B A \B x>0$:}

\mdcond{
$\B L = \B U$ and pivoting is not required, \coloremph{Cholesky} algorithm $\B A = \B L \B L^T$ can be used ($\B L$ in Cholesky is not unit-diagonal).
}
\item {\bf Matrix is symmetric but indefinite}:

\mdcond{
Compute pivoted \coloremph{LDL factorization} $\B P \B A \B P^T = \B L \B D \B L^T$ (where $\B L$ is lower-triangular and unit-diagonal, while $\B D$ is diagonal)
}
\item {\bf Matrix is \coloremph{banded}, $a_{ij}= 0$ if $|i-j|>b$}:

\mdcond{
LU without pivoting and Cholesky preserve banded structure and require only $O(nb^2)$ work.
}
\end{itemize}

\end{frame}

\begin{frame}{Solving Many Linear Systems}

\urcornerlinkdemoinclass{02-linear-systems}{Sherman-Morrison}{inclass-sherman-morrison-woodbury}{Sherman-Morrison-Woodbury Formula}

\begin{itemize}
\item {\bf Suppose we have computed $\B A = \B L \B U$ and want to solve $\B A \B X = \B B$ where $\B B$ is $n\times k$ with $k<n$:}

\mdcond{
Cost is $O(n^2k)$ for solving the $k$ independent linear systems
}

\item {\bf Supposed  we have computed $\B A = \B L \B U$ and now want to solve a perturbed system $(\B A - \B u \B v^T)\B x = \B b$:}

Can use the \coloremph{Sherman-Morrison-Woodbury} formula
\[(\B A - \B u \B v^T)^{-1} = \B A^{-1} + \frac{\B{A}^{-1}\B u\B v^T\B A^{-1}}{1- \B v^T \B{A}^{-1}\B u}\] 
\lgcond{
\begin{itemize}
\item Consequently we have $\B A \B x = \B b + \frac{\B u\B v^T\B A^{-1}}{1- \B v^T \B{A}^{-1}\B u}\B b =\B b + \frac{\B v^T\B A^{-1}\B b}{1- \B v^T \B{A}^{-1}\B u}\B u  $
\sitem Need not form $\B A^{-1}$ or $\B L^{-1}$ or $\B U^{-1}$, suffices to use backward/forward substitution to solve $\B w^T \B A = \B v^T$, i.e. solve $\B U^T \B L^T \B w = \B v$ and then solve
\[\B L \B U \B x = \B b + \underbrace{\bigg(\frac{\B w^T\B b}{1- \B w^T \B u}\bigg)}_\text{scalar}\B u \]
\end{itemize}
}
\end{itemize}

\end{frame}


%\end{document}
