%%\documentclass[handout]{beamer}
%\documentclass[aspectratio=169,13pt]{beamer}

%\newcommand{\dee}{\partial}

\newcommand{\bmat}[1]{\begin{bmatrix} \B{#1}_{11} & \B{#1}_{12} \\ \B{#1}_{21} & \B{#1}_{22}\end{bmatrix}}
\newcommand{\blmat}[1]{\begin{bmatrix} \B{#1}_{11} &  \\ \B{#1}_{21} & \B{#1}_{22}\end{bmatrix}}
\newcommand{\brmat}[1]{\begin{bmatrix} \B{#1}_{11} & \B{#1}_{12} \\ & \B{#1}_{22}\end{bmatrix}}
\newcommand{\bomat}[1]{\begin{bmatrix} {#1}_{11} & \B{#1}_{12} \\ \B{#1}_{21} & \B{\uppercase{#1}}_{22}\end{bmatrix}}
\newcommand{\blomat}[1]{\begin{bmatrix} {#1}_{11} & \\ \B{#1}_{21} & \B{\uppercase{#1}}_{22}\end{bmatrix}}
\newcommand{\bromat}[1]{\begin{bmatrix} {#1}_{11} & \B{#1}_{12} \\  & \B{\uppercase{#1}}_{22}\end{bmatrix}}

\DeclareMathOperator{\vcop}{vec}
\newcommand{\vc}[1]{\vcop\left(#1\right)}
\newcommand{\prm}{\mu}

\newcommand{\fl}{fl}
\newcommand{\asmcond}[1]{\smallskip{\it #1}\smallskip}
\newcommand{\amdcond}[1]{\smallskip{\it #1}\smallskip}
\newcommand{\algcond}[1]{\smallskip{\it #1}\smallskip}
\newcommand{\atsmcond}[1]{{\it #1}}
\newcommand{\atmdcond}[1]{{\it #1}}
\newcommand{\atlgcond}[1]{{\it #1}}
\newcommand{\afillnum}[1]{{\underline{ \ #1\ \ }}}

\mode<presentation>
{
  \usetheme{Malmoe}
}

\usepackage[english]{babel}
%
%\usepackage[latin1]{inputenc}
%
%\usepackage{times}
%\usepackage[T1]{fontenc}
% Or whatever. Note that the encoding and the font should match. If T1
% does not look nice, try deleting the line with the fontenc.

\usepackage{inconsolata}
\usepackage{listings}
\usepackage{bm}
\usefonttheme[onlymath]{serif}
\renewcommand\familydefault{\sfdefault}
\usepackage[sfdefault]{ClearSans} %% option 'sfdefault' activates Clear Sans as the default text font
%\usefonttheme{professionalfonts}

\usepackage{latexsym}
\usepackage{amsmath}
\usepackage{amssymb}
\usepackage{amsfonts}
\usepackage{graphics}
\usepackage{multicol}

\definecolor{darkred}{rgb}{0.8,0.2,0.2}
\newcommand{\coloremph}[1]{\textcolor{darkred}{\emph{#1}}}        % colored italics
\newcommand{\reshape}[2]{o_{#1}(#2)}        % colored italics
\newcommand{\ket}[1]{\lvert #1 \rangle}
\newcommand{\bra}[1]{\langle #1 \vert}
\newcommand{\transp}[2]{{#2}^{\langle  {#1} \rangle }}        % colored italics

% Miscellaneous special math symbols

% Big-oh notation - either Latex's calligraphic O or uppercase math italic O
\newcommand{\BIGOH}{\mathcal{O}}                              % big oh
%\newcommand{\BIGOH}{O}                                       % big oh
\newcommand{\BIGTHETA}{\Theta}                                % big theta

\newcommand{\sfrac}[2]{{#1}/{#2}}

% Various versions of the fraction one-half
% solidus 1/2 for superscripts:
\newcommand{\SHALF}{1/2}                                      % one-half power
% small 1/2 set case in displayed equations:
\newcommand{\HALF}{\mbox{\small $\frac{1}{2}$}}               % small one-half
\newcommand{\THRD}{\mbox{\scriptsize $\frac{1}{3}$}}          % small one-third
\newcommand{\TWOTH}{\mbox{\scriptsize $\frac{2}{3}$}}         % small two-thirds

% Machine epsilon - note that placement of subscript may need adjustment.
%\newcommand{\emach}{\epsilon_{\textrm{\scriptsize mach}}} % machine prec.
\newcommand{\emach}{\epsilon}

\newcommand{\lsapprox}{\cong}                             % least squares approx

% Real numbers - Prefer \mathbb{R} but it requires amsfonts.
%                Latex's calligraphic R will do if amsfonts are unavailable.
%                DO NOT use \Re, which gives old German fraktur R.
\newcommand{\Real}{\mathbb{R}}                                 % real numbers
\newcommand{\Cplx}{\mathbb{C}}                                 % complex numbers
\newcommand{\Poly}{\mathbb{P}}                                 % polynomials
\newcommand{\Float}{\mathbb{F}}                                % fl pt system
% Bold math fonts for vectors and matrices
\renewcommand{\Vec}[1]{\ensuremath{\bm{#1}}}                   % vector
\newcommand{\Mat}[1]{\ensuremath{\bm{#1}}}                     % matrix
\newcommand{\Op}[1]{#1}                              % operator
\newcommand{\loc}[1]{x_{#1}}                              % operator

% Loglike functions set in regular type
\newcommand{\diag}{\mathrm{diag}}                             % diagonal matrix
\newcommand{\cond}{\mathrm{cond}}                             % condition number
\newcommand{\sign}{\mathrm{sign}}                             % sign function
\newcommand{\Span}{\text{span}}                             % span of matrix
%\newcommand{\trace}{\mathrm{trace}}                           % trace of matrix
\DeclareMathOperator*{\trace}{trace}
\newcommand{\tentrace}[3]{\trace_{#1,#2}(#3)}
\renewcommand{\Re}{\mathrm{Re}}                               % real part
\renewcommand{\Im}{\mathrm{Im}}                               % imaginary part

% Fractions appearing in matrices - allows setting them case or solidus
%\newcommand{\mf}[2]{{#1 \over #2}}        % matrix fraction - set case
\newcommand{\mf}[2]{#1/#2}               % matrix fraction - set solidus


% Keywords for algorithm statements

\newcommand{\FOR}{\textbf{for}\ }
\newcommand{\TO}{\textbf{to}\ }
\newcommand{\IF}{\textbf{if}\ }
\newcommand{\THEN}{\textbf{then}\ }
\newcommand{\ELSE}{\textbf{else}\ }
\newcommand{\WHILE}{\textbf{while}\ }
\newcommand{\DO}{\textbf{do}\ }
\newcommand{\BEGIN}{\textbf{begin}\ }
\newcommand{\END}{\textbf{end}\ }
\newcommand{\STOP}{\textbf{stop}\ }
\newcommand{\AND}{\textbf{and}\ }
\newcommand{\OR}{\textbf{or}\ }

\newcommand{\comm}{\mathrm{comm}}
\newcommand{\comp}{\mathrm{comp}}
\newcommand{\idle}{\mathrm{idle}}
\newcommand{\msg}{\mathrm{msg}}
\newcommand{\len}{\mathrm{s}}
\newcommand{\route}{\mathrm{route}}
\newcommand{\mitem}{\medskip\item}
\newcommand{\sitem}{\smallskip\item}
\DeclareMathOperator*{\argmin}{argmin}

%Gets rid of headline

\setbeamertemplate{footline}{}

\setbeamertemplate{headline}{}
\beamertemplatenavigationsymbolsempty



\definecolor{mygreen}{rgb}{0,0.2,0}
\definecolor{mygray}{rgb}{0.5,0.5,0.5}
\definecolor{mymauve}{rgb}{0.58,0,0.82}
\definecolor{mypurple}{rgb}{0.38,0,0.32}
\definecolor{myblue}{rgb}{0.2,0,0.5}
\definecolor{darkgreen}{rgb}{0.2,0.6,0.2}
\definecolor{Brown}{rgb}{0.39, 0.09, 0.0}
\definecolor{Violet}{rgb}{0.38, 0.0, 0.63}
\definecolor{myvlgray}{rgb}{0.9,0.9,1.0}
\definecolor{mylgray}{rgb}{0.85,0.85,0.9}
\definecolor{darkgreen}{rgb}{0,0.4,0}
\definecolor{brown}{rgb}{.6,.1,.1}

\newcommand{\bemph}[1]{{\color{blue} \emph{#1}}}

\usepackage{mathtools}
\newcommand{\defeq}{\coloneqq}

\newcommand{\inti}[2]{[{#1},{#2}]}
\newcommand{\BF}{\mathbf}
\newcommand{\CF}{\mathcal}
\newcommand{\B}[1]{\bm{#1}}
\newcommand{\E}[1]{\bm{#1}}
\newcommand{\dis}{\displaystyle}
\newcommand{\lt}{\left}
\newcommand{\rt}{\right}

\newcommand{\cs}{H}

\DeclareMathOperator*{\rank}{rank}
\DeclareMathOperator*{\vecn}{vec}
\DeclareMathOperator*{\vecs}{vech}

\usepackage{eso-pic}
\newcommand{\cornertext}[1]{
  \AddToShipoutPictureFG*{
    \AtPageUpperLeft{\put(0,-10){\makebox[\paperwidth][r]{#1}}}  
   }%
}
\newcommand{\cornertexttwo}[2]{
  \AddToShipoutPictureFG*{
    \AtPageUpperLeft{\put(0,-10){\makebox[\paperwidth][r]{#1}}}  
    \AtPageUpperLeft{\put(0,-20){\makebox[\paperwidth][r]{#2}}}  
   }%
}
\newcommand{\linkdemo}[2]{{\footnotesize\href{https://relate.cs.illinois.edu/course/cs450-f18/f/demos/upload/#1/#2.html}{\color{Violet}{\it\textbf{Demo:} #2}\ }}}
\newcommand{\linkinclass}[2]{{\footnotesize\href{https://relate.cs.illinois.edu/course/cs450-f18/flow/#1/start/}{\color{darkgreen}{\it\textbf{Activity:} #2}\ }}}
\newcommand{\urcornerlinkinclass}[2]{\cornertext{\linkinclass{#1}{#2}}}
\newcommand{\dblurcornerlinkinclass}[4]{\cornertexttwo{\linkinclass{#1}{#2}}{\linkinclass{#3}{#4}}}
\newcommand{\urcornerlinkdemo}[2]{\cornertext{\linkdemo{#1}{#2}}}
\newcommand{\dblurcornerlinkdemo}[4]{\cornertexttwo{\linkdemo{#1}{#2}}{\linkdemo{#3}{#4}}}
\newcommand{\urcornerlinkdemoinclass}[4]{\cornertexttwo{\linkdemo{#1}{#2}}{\linkinclass{#3}{#4}}}



% CS 554 specific:

\newcommand{\tsync}{\alpha}
\newcommand{\tword}{\beta}
\newcommand{\tflop}{\gamma}

\newcommand{\bw}{w}

\newcommand{\tpl}[2]{{\bm{#1}}}
\newcommand{\vtpl}[3]{{\bm{#1}}_#3}
\newcommand{\perm}[2]{\lt[#1\rt]_{#2}}
\newcommand{\costyle}{\ttfamily\bfseries}
\newcommand{\cotext}[1]{{\costyle{#1}}}
\newcommand{\kwstyle}{\costyle\textcolor{mypurple}}
\newcommand{\kwtext}[1]{{\kwstyle{#1}}}
\newcommand{\emstyle}{\costyle\textcolor{myblue}}
\newcommand{\emtext}[1]{{\emstyle{#1}}}
\newcommand{\const}{}
\newcommand{\work}{Q}
\newcommand{\depth}{D}
\newcommand{\flops}{F}
\newcommand{\words}{W}
\newcommand{\syncs}{S}
\newcommand{\mem}{M}
\newcommand{\eff}{E}
\newcommand{\isofun}[1]{\tilde{\work}(p)}
\newcommand{\isomem}[1]{\tilde{\mem}(p)}
\newcommand{\pstrong}{p_s}
\newcommand{\pweak}{p_w}
\newcommand{\ALL}{\star }
\newcommand{\rmn}[2]{\mathbb{R}^{#1\times #2}}
\newcommand{\rn}[1]{\mathbb{R}^{#1}}
\newcommand{\err}{\varepsilon}
\newcommand{\mat}[1]{\begin{bmatrix} #1 \end{bmatrix}}
\newcommand{\mc}[1]{\mathcal{#1}}
\newcommand{\h}[2]{\mc{H}_{#1}(#2)}
\newcommand{\T}{T}%{\mathsf{T}}
\newcommand{\dn}[2]{\mc{M}_{#1}^{\uparrow}(#2)}
%\usepackage[colorlinks=false,urlbordercolor={1.0 1.0 1.0}]{hyperref}
%
%\lstset{ %
%  postbreak=false,
%%  backgroundcolor=\color{white},   % choose the background color; you must add \usepackage{color} or \usepackage{xcolor}
%  basicstyle=\costyle,        % the size of the fonts that are used for the code
%%  breakatwhitespace=false,         % sets if automatic breaks should only happen at whitespace
%%  breaklines=false,                 % sets automatic line breaking
%  captionpos=n,                    % sets the caption-position to bottom
%  commentstyle=\color{mygreen},    % comment style
%  deletekeywords={...},            % if you want to delete keywords from the given language
%  escapeinside={\%*}{*)},          % if you want to add LaTeX within your code
%  extendedchars=true,              % lets you use non-ASCII characters; for 8-bits encodings only, does not work with UTF-8
%  frame=none,                    % adds a frame around the code
%  keepspaces=true,                 % keeps spaces in text, useful for keeping indentation of code (possibly needs columns=flexible)
%  keywordstyle=\color{mypurple},       % keyword style
%  language=C++,                 % the language of the code
%  otherkeywords={*,...},            % if you want to add more keywords to the set
%  numbers=none,                    % where to put the line-numbers; possible values are (none, left, right)
%  numbersep=5pt,                   % how far the line-numbers are from the code
%  numberstyle=\tiny\color{mygray}, % the style that is used for the line-numbers
%  rulecolor=\color{black},         % if not set, the frame-color may be changed on line-breaks within not-black text (e.g. comments (green here))
%  showspaces=false,                % show spaces everywhere adding particular underscores; it overrides 'showstringspaces'
%  showstringspaces=false,          % underline spaces within strings only
%  showtabs=false,                  % show tabs within strings adding particular underscores
%  stepnumber=2,                    % the step between two line-numbers. If it's 1, each line will be numbered
%  stringstyle=\color{mymauve},     % string literal style
%  tabsize=2,                     % sets default tabsize to 2 spaces
%  title=\lstname,                   % show the filename of files included with \lstinputlisting; also try caption instead of title
%  keywordstyle=\emstyle
%}

\lstset{ %
  postbreak=false,
%  backgroundcolor=\color{white},   % choose the background color; you must add \usepackage{color} or \usepackage{xcolor}
  basicstyle=\footnotesize\costyle,        % the size of the fonts that are used for the code
%  breakatwhitespace=false,         % sets if automatic breaks should only happen at whitespace
%  breaklines=false,                 % sets automatic line breaking
  captionpos=n,                    % sets the caption-position to bottom
  commentstyle=\color{mygreen},    % comment style
  deletekeywords={privatei,shared},            % if you want to delete keywords from the given language
  escapeinside={\%*}{*)},          % if you want to add LaTeX within your code
  extendedchars=true,              % lets you use non-ASCII characters; for 8-bits encodings only, does not work with UTF-8
  frame=none,                    % adds a frame around the code
  keepspaces=true,                 % keeps spaces in text, useful for keeping indentation of code (possibly needs columns=flexible)
  keywordstyle=\footnotesize\color{mypurple},       % keyword style
  language=C++,                 % the language of the code
  otherkeywords={*,pragma},            % if you want to add more keywords to the set
  numbers=none,                    % where to put the line-numbers; possible values are (none, left, right)
  numbersep=5pt,                   % how far the line-numbers are from the code
  numberstyle=\footnotesize\color{mygray}, % the style that is used for the line-numbers
  rulecolor=\color{black},         % if not set, the frame-color may be changed on line-breaks within not-black text (e.g. comments (green here))
  showspaces=false,                % show spaces everywhere adding particular underscores; it overrides 'showstringspaces'
  showstringspaces=false,          % underline spaces within strings only
  showtabs=false,                  % show tabs within strings adding particular underscores
  stepnumber=2,                    % the step between two line-numbers. If it's 1, each line will be numbered
  stringstyle=\footnotesize\color{mymauve},     % string literal style
  tabsize=2,                     % sets default tabsize to 2 spaces
  title=\lstname,                   % show the filename of files included with \lstinputlisting; also try caption instead of title
  emph={MPI_Send,MPI_Recv,MPI_Comm,MPI_Comm_rank,MPI_Send,MPI_Comm_split,MPI_Init,MPI_Finalize,MPI_Status,MPI_COMM_WORLD,MPI_Datatype,MPI_Comm_size,MPI_FLOAT,mpi.h,omp,parallel,shared,private,default,reduction},
  emphstyle=\footnotesize\emstyle
}

\title{CS 450: Numerical Anlaysis\footnote{{\it These slides have been drafted by Edgar Solomonik as lecture templates and supplementary material for the book ``Scientific Computing: An Introductory Survey'' by Michael T. Heath (\href{http://heath.cs.illinois.edu/scicomp/notes/index.html}{{\color{blue} slides}}).}}}
\author{}
\institute{University of Illinois at Urbana-Champaign}




\subtitle{Numerical Optimization}

\date{}
\begin{document}

\begin{frame}
  \titlepage
\end{frame}

\section{Numerical Optimization Problems}

\subsection{Constrained and Unconstrained Optimization}

\begin{frame}{Numerical Optimization}

\begin{itemize}
\item Our focus will be on \coloremph{continuous} rather than \coloremph{combinatorial} optimization:
\[\min_{\B x} f(\B x)\quad \text{subject to} \quad \B g(\B x) = \B 0 \quad \text{and} \quad \B{h}(\B x) \leq \B 0\]

\lgcond{
where $f\in \mathbb{R}^n\to \mathbb{R}$ is assumed to be differentiable.
\begin{itemize}
\sitem Without the constraints, i.e. with $\B g = \B 0$ and $\B h = \B 0$, the problem is \coloremph{unconstrained}.
\sitem With constraints, the \coloremph{constrained} optimization problem restricts the solution to elements of the \coloremph{feasible region}: $\{\B x:\B g(\B x) = \B 0 \ \text{and} \ \B{h}(\B x) \leq \B 0\}$.
\end{itemize}
}

\item We consider linear, quadratic, and general nonlinear optimization problems:

\lgcond{
\begin{itemize}
\item If $f$, $\B g$, and $\B h$ are affine (linear and constant terms only) then we have \coloremph{linear programming} problem.
\sitem If $f$ is quadratic while $\B g$ and $\B h$ are linear, then we have a \coloremph{quadratic programming} problem, for which specialized methods exist.
\sitem Generally, we have a \coloremph{nonlinear programming} problem.
\end{itemize}
}

\end{itemize}

\end{frame}


\subsection{Existence and Uniqueness of Solutions}

\begin{frame}{Local Minima and Convexity}

\begin{itemize}
\item Without knowledge of the analytical form of the function, numerical optimization methods at best achieve convergence to a \coloremph{local} rather than \coloremph{global} minimum:

\lgcond{
If the input domain is infinite or the global minimum is in an infinitesimally narrow trough, it may be impossible to find the global minimum in finite time.
}

\item A set is \coloremph{convex} if it includes all points on any line, while a function is (strictly) convex if its (unique) local minimum is always a global minimum:

\lgcond{
\begin{itemize}
\item Set $S$ is convex if \[\forall \B x,\B y \in S,\quad \alpha \in [0,1], \quad \alpha \B x + (1-\alpha)\B y \in S.\]
\item Function $f$ is convex if \[f(\alpha \B x + (1-\alpha)\B y)\leq \alpha f(\B x) + (1-\alpha)f(\B y).\]
\item A function may have a unique global minima but not be convex.
\end{itemize}
}

\end{itemize}

\end{frame}

\begin{frame}{Existence of Local Minima}

\begin{itemize}
\item \coloremph{Level sets} are all points for which $f$ has a given value, \coloremph{sublevel sets} are all points for which the value of $f$ is less than a given value:

\lgcond{
\[L(z) = \{ \B x: f(\B x)=z\}\]
\[S(z) = \{ \B x: f(\B x)\leq z\}\]
}

\item If there exists a closed and bounded sublevel set in the domain of feasible points, then $f$ has has a global minimum in that set:

\lgcond{
Need a value $z$ such that $S(z)$ has finite size, is contiguous, and includes its own boundary.
}

\end{itemize}


\end{frame}

\subsection{Properties of Local Minima}

\begin{frame}{Optimality Conditions}

\begin{itemize}
\item If $\B x$ is an interior point in the feasible domain and is a local minima, 
\[\nabla f(\B x) = \begin{bmatrix} \frac{df}{dx_1}(\B x) & \cdots \frac{df}{dx_n}(\B x)\end{bmatrix}^T = \B 0:\]

\lgcond{
\begin{itemize}
\item If $\frac{df}{dx_i}(\B x) < 0$ an infinitesimal increment to $x_i$ improves the solution, 
\sitem if $\frac{df}{dx_i}(\B x) > 0$
an infinitesimal decrement to $x_i$ improves the solution.
\end{itemize}
}
\item \coloremph{Critical points} $\B x$ satisfy $\nabla f(\B x) = \B 0$ and can be minima, maxima, or saddle points:

\lgcond{
For scalar function $f$, can distinguish the three by considering sign of $f''(x)$.
}


\end{itemize}


\end{frame}

\begin{frame}{Hessian Matrix}

\begin{itemize}
\item To ascertain whether a critical point $\B x$, for which $\nabla f(\B x) = \B 0$, is a local minima, consider the \coloremph{Hessian matrix}:
\mdcond{
\[\B{H}_f(\B x) = \B{J}_{\nabla f}(\B x) = \begin{bmatrix} 
                                  \frac{d^2 f}{dx_1^2}(\B x) & \cdots & \frac{d^2 f}{dx_1dx_n}(\B x) \\
                                  \vdots & \ddots & \vdots  \\
                                  \frac{d^2 f}{dx_ndx_1}(\B x) & \cdots & \frac{d^2 f}{dx_n^2}(\B x) \end{bmatrix}\]


The Hessian matrix is always symmetric.
}

\item If $\B x^*$ is a minima of $f$, then $\B{H}_f(\B x^*)$ is positive semi-definite:

\lgcond{
If $\B{H}_f(\B x^*)$ is not positive semi-definite, there exists normalized vector $\B{s}$ such that $\B s^T \B{H}_f(\B x^*)\B s < 0$, which means that
for a sufficiently small $\alpha$, $\B{\hat{x}} = \B x^* + \alpha \B s$ will have be a better solution, $f(\B{\hat{x}})<f(\B{x^*})$,
since the gradient is zero at $\B x^*$ and decreases for an infinitesimal perturbation of $\B x^*$ in the direction $\B s$.
}
\mdcond{}


\end{itemize}


\end{frame}


\begin{frame}{Optimality on Feasible Region Border}

\begin{itemize}
\item Given an equality constraint $\B g(\B x)=\B 0$, % implies that a minimum $\B x^*$ must satisfy $\B g(\B x^*)=\B 0$.
      it is no longer necessarily the case that $\nabla f(\B x^*) =\B 0$.
      Instead, it may be that directions in which the gradient decreases lead to points outside the feasible region:

\[\exists \B \lambda\in\mathbb{R}^n, \quad -\nabla f(\B x^*) = \B J^T_{\B g}(\B x^*) \B \lambda\]

\lgcond{
\begin{itemize}
\item $\B \lambda$ are referred to as the Lagrange multipliers.
\sitem This condition implies that at $\B x^*$, the direction in which $f$ decreases is in the span of directions moving along which would exit the feasible region.
\end{itemize}
}

\item Such \coloremph{constrained minima} are critical points of the Lagrangian function $\mathcal{L}(\B x, \B \lambda) = f(\B x) + \B \lambda^T\B g(\B x)$, so they satisfy: %, described by the nonlinear equation,

\[\nabla \mathcal{L}(\B x^*, \B \lambda) = \begin{bmatrix} \nabla f(\B x^*) + \B J_{\B g}^T(\B x^*) \B \lambda \\ \B g(\B x^*)\end{bmatrix} = \B 0\]

\lgcond{
Seeking $\B \lambda^*$ to obtain a function $k(\B x) = \mathcal{L}(\B x, \B \lambda^*)$ with maximum global minimum is the \coloremph{dual optimization problem}.
}


\end{itemize}


\end{frame}

\subsection{Conditioning of Minima}

\begin{frame}{Sensitivity and Conditioning}

\begin{itemize}
\item The condition number of solving a nonlinear equations is $1/f'(x^*)$, however for a minimizer $x^*$, we have $f'(x^*)=0$, so conditioning of optimization is inherently bad:

\lgcond{Consider perturbation of function values for a function that changes slowly near the minimum.}

\item To analyze worst case error, consider how far we have to move from a root $\B x^*$ to perturb the function value by $\epsilon$:

\lgcond{
\[\epsilon = f(x^*+h)-f(x^*)=  \underbrace{f'(x^*)h}_{0} + \frac{1}{2}f''(x^*)h^2 + O(h^3)\]
\begin{itemize}
\item so the function value changes by $\frac{1}{2}f''(x^*)h^2$, which implies we need $h=O(\sqrt{\epsilon})$,
\sitem  a perturbation to the function value in the $k$th significant digit, could result in the solution changing in the $k/2$th significant digit.
\end{itemize}
}
%For an absolute perturbation to the function of size $\epsilon$, we can check how far away If we perturb $x^*$ by $h^*$

\end{itemize}

\end{frame}

\subsection{Algorithms for 1D Optimization}

\begin{frame}{Golden Section Search}

\urcornerlinkdemo{06-optimization}{Golden Section Proportions}

\begin{itemize}
\item Given bracket $[a,b]$ with a unique minimum ($f$ is \coloremph{unimodal} on the interval), \coloremph{golden section search} considers consider points $f(x_1),f(x_2)$, $a<x_1<x_2<b$ and discards subinterval $[a,x_1]$ or $[x_2,b]$:

\lgcond{
\begin{itemize}
\sitem If a function is strictly convex and bounded on $[a,b]$, it is unimodal on that interval, but a unimodal function may be non-convex.
\sitem Because the function is unimodal, if we have $f(x_1)<f(x_2)$ then the unique local minima $f$ in $[a,b]$ has to be in the interval $[a,x_2]$.
\sitem So, if $f(x_1)<f(x_2)$ can restrict search to $[a,x_2]$ and otherwise to $[x_1,b]$.
\end{itemize}
}

\item Since one point remains in the interval, golden section search selects $x_1$ and $x_2$ so one of them can be effectively reused in the next iteration:

\lgcond{
\begin{itemize}
\sitem For example, when $f(x_1)>f(x_2)$,  $x_2$ is inside $[x_1,b]$ and we would like $x_2$ to serve as the $x_1$ for the next iteration.
\sitem To ensure this, and minimize resulting interval length, we pick $x_2=a+(b-a)(\sqrt{5}-1)/2$ and $x_1=b-(b-a)(\sqrt{5}-1)/2$.
\sitem Consequently, the convergence of golden secetion search is linear with constant $(\sqrt{5}-1)/2$ per function evaluation.
\end{itemize}
}

\end{itemize}

\end{frame}


\begin{frame}{Newton's Method for Optimization}

\urcornerlinkdemo{06-optimization}{Newton's Method in 1D}

\begin{itemize}
\item At each iteration, approximate function by quadratic and find minimum of quadratic function:

\lgcond{
Pick quadratic function $\hat{f}$ as first three terms of Taylor expansion of $f$ about $x_k$, matching value and first two derivatives of $f$ at $x_k$.
}

\item The new approximate guess will be given by $x_{k+1}-x_k = -f'(x_k)/f''(x_k)$:

\lgcond{
\[f(x) \approx \hat{f}(x) = f(x_k)+f'(x_k)(x-x_k) + \frac{1}{2}f''(x_k)(x-x_k)^2\]
since the function is quadratic, we can find its unique critical point to find its minima,
\[\hat{f}'(x_{k+1}) = f'(x_k) + f''(x_k)(x_{k+1}-x_k) = 0.\]
}

\end{itemize}

\end{frame}

%\begin{frame}{Successive Parabolic Interpolation}
%
%\begin{itemize}
%\item Like in secant methods, we can use previous points, rather than derivatives of $f$ to define quadratic approximation to $f$:
%
%\lgcond{
%Given three points, there is a unique quadratic function interpolating them.
%}
%
%\item The convergence rate of the resulting method is roughly $1.324$
%
%\lgcond{
%By comparison, the convergence of bisection is linear with a constant of $0.618$, while Newton's method converges quadratically.
%}
%
%\end{itemize}
%
%\end{frame}
%\end{document}


\begin{frame}{Successive Parabolic Interpolation}

\begin{itemize}
\item Interpolate $f$ with a quadratic function at each step and find its minima:

\lgcond{
Given three points, there is a unique quadratic function interpolating them.
}

\item The convergence rate of the resulting method is roughly $1.324$

\lgcond{
By comparison, the convergence of golden section search is linear with a constant of $0.618$, while Newton's method converges quadratically.
}

\end{itemize}

\end{frame}

%\begin{frame}{Newton's Method for Optimization}
%
%\begin{itemize}
%\item Instead of interpolating points, match derivatives at current approximate point:
%
%\lgcond{
%Pick quadratic function $\hat{f}$ as first three terms of Taylor expansion of $f$ about $x_k$, matching value and first two derivatives of $f$ at $x_k$.
%}
%
%\item The new approximate guess will be given by $x_{k+1}-x_k = -f'(x_k)/f''(x_k)$:
%
%\lgcond{
%\[f(x_{k+1}-x_k) \approx \hat{f}(x_{k+1}-x_k) = f(x_k)+f'(x_k)(x_{k+1}-x_k) + \frac{1}{2}f''(x_k)(x_{k+1}-x_k)^2\]
%since the function is quadratic, we can find its unique critical point to find its minima,
%\[\hat{f}'(x_{k+1}-x_k) = f'(x_k) + f''(x_k)(x_{k+1}-x_k) = 0.\]
%}
%
%\end{itemize}
%
%\end{frame}



\begin{frame}{Safeguarded 1D Optimization}

\begin{itemize}
\item Safeguarding can be done by bracketing via golden section search:

\lgcond{
Combination of Newton and golden section search 
\begin{itemize}
\sitem achieves quadratic convergence locally,
\sitem is guaranteed convergence provided unimodality of function.
\end{itemize}
}

\item Backtracking and step-size control:

\lgcond{
\begin{itemize}
\sitem Can take smaller step $x_{k+1} = x_k - \alpha_k f'(x_k)/f''(x_k)$ for some $\alpha_k<1$.
\sitem Can backtrack and choose smaller $\alpha_k$ if $f(x_{k+1})>f(x_k)$.
\end{itemize}
}

\end{itemize}

\end{frame}

\section{Multidimensional Quadratic Optimization}

\subsection{Steepest Descent}

\begin{frame}{General Multidimensional Optimization}

\urcornerlinkdemo{06-optimization}{Nelder-Mead Method}

\begin{itemize}
\item Direct search methods by simplex (\coloremph{Nelder-Mead}):

\lgcond{
Form a $n$-point polytope in $n$-dimensional space and adjust worst point (highest function value) by moving it along a line passing through the centroid of the remaining points.
}

\item Steepest descent: find the minimizer in the direction of the negative gradient:

\lgcond{
\[\B{x}_{k+1} = \B{x}_k - \alpha_k \nabla f(\B x_k)\]
such that $f(\B{x}_{k+1})=\min_{\alpha_k} f(\B{x}_k - \alpha_k \nabla f(\B x_k))$, i.e. perform a line search (solve 1D optimization problem) in the direction of the negative gradient.
}

\end{itemize}

\end{frame}

\begin{frame}{Convergence of Steepest Descent}

\urcornerlinkdemo{06-optimization}{Steepest Descent}

\begin{itemize}
\item Steepest descent converges linearly with a constant that can be arbitrarily close to $1$:

\lgcond{
%See Michael Heath's notes chapter 6 slide 35 for an example.
\begin{itemize}
\sitem Convergence is slow locally, in the worst case, and generally depends on the Hessian near the minima.
\sitem If the gradient is changing quickly, it serves as good approximation only within a small local neighborhood, so the line search may result in arbitrarily small steps.
\end{itemize}
}
\item Given quadratic optimization problem $f(\B x)= \frac{1}{2}\B x^T\B A \B x + \B c^T \B x$ where $\B A$ is symmetric positive definite, the error $\B e_k = \B x_k - \B x^{*}$ satisfies
\lgcond{
\[||\B e_{k+1}||_{\B A} = \B e_{k+1}^T \B A \B e_{k+1} = \frac{\sigma_\text{max}(\B A) -\sigma_\text{min}(\B A)}{\sigma_\text{max}(\B A) +\sigma_\text{min}(\B A)}||\B e_k||_{\B A}\]

\begin{itemize}
\item When sufficiently close to a local minima, general nonlinear optimization problems are described by such an SPD quadratic problem.
\sitem Convergence rate depends on the conditioning of $\B A$, since 
\[\frac{\sigma_\text{max}(\B A) -\sigma_\text{min}(\B A)}{\sigma_\text{max}(\B A) +\sigma_\text{min}(\B A)}=\frac{\kappa(\B A)-1}{\kappa(\B A)+1}.\]
\end{itemize}
}
\end{itemize}

\end{frame}

\begin{frame}{Gradient Methods with Extrapolation}

\begin{itemize}
\item We can improve the constant in the linear rate of convergence of steepest descent by leveraging \coloremph{extrapolation methods}, which consider two previous iterates (maintain \coloremph{momentum} in the direction $\B x_k -\B x_{k-1}$):
\lgcond{
\[\B x_{k+1} = \B x_k - \alpha_k \nabla f(\B x_k) + \beta_k(\B x_k - \B x_{k-1})\]


}

\item The \coloremph{heavy ball method}, which uses constant $\alpha_k=\alpha $ and $\beta_k=\beta$, achieves better convergence than steepest descent:

\lgcond{
\[||\B e_{k+1}||_{\B A} = \frac{\sqrt{\kappa(\B A)} -1}{\sqrt{\kappa(\B A) }+1}||\B e_k||_{\B A}\]
Nesterov's gradient optimization method is another instance of an extrapolation method that provides further improved optimality guarantees.

}

\end{itemize}

\end{frame}

\subsection{Conjugate Gradient}

\urcornerlinkdemo{06-optimization}{Conjugate Gradient Method}

\begin{frame}{Conjugate Gradient Method}

\begin{itemize}
\item The \coloremph{conjugate gradient method} is capable of making the optimal choice of $\alpha_k$ and $\beta_k$ at each iteration of an extrapolation method:

\lgcond{
\[(\alpha_k,\beta_k)=\argmin_{\alpha_k,\beta_k}\bigg[f\Big(\B x_k - \alpha_k \nabla f(\B x_k) + \beta_k(\B x_k - \B x_{k-1})\Big)\bigg]\]
\begin{itemize}
\item For SPD quadratic programming problems, conjugate gradient is an optimal 1st order method, converging in $n$ iterations.
\sitem It implicitly computes Lanczos iteration, searching along $\B A$-orthogonal directions at each step.
\end{itemize}
}

\item \coloremph{Parallel tangents} implementation of the method proceeds as follows

\lgcond{
\begin{enumerate}
\sitem Perform a step of steepest descent to generate $\B{\hat{x}}_{k}$ from $\B x_k$.
\sitem Generate $\B x_{k+1}$ by minimizing over the line passing through $\B x_{k-1}$ and $\B{\hat{x}}_{k}$.
\end{enumerate}
}

\end{itemize}

\end{frame}

\begin{frame}{Krylov Optimization}

\urcornerlinkdemo{06-optimization}{Conjugate Gradient Parallel Tangents as Krylov Subspace Method}

\begin{itemize}
\item Conjugate Gradient finds the minimizer of $f(\B x) = \frac{1}{2} \B x^T \B A \B x + \B c^T \B x $ within the Krylov subspace of $\B A$:
%Generally, Krylov subspaces describe the information available from $k$ matrix-vector products, and can be used to find an approximation $\B x_k$ to the minima of $\B c^T \B x - \B x^T \B A \B x $:

\lgcond{
\begin{itemize}
\sitem It constructs Krylov subspace $\mathcal{K}_k(\B A, \B c) = \text{span}(\B c,\B A\B c, \ldots, \B A^{r-1}\B c)$.
\sitem At the $k$th step conjugate gradient yields iterate
\[\B x_k = -||\B c||_2\B Q_k \B T_k^{-1} \B e_1 ,\]
where $\B Q_k$ are the Lanczos vectors associated with $\mathcal{K}_k(\B A, \B c)$ and $\B T_k =\B Q_k^T \B A \B Q_k$.
\sitem This choice of $\B x_k$ minimizes $f(\B x)$ since
\begin{align*}
\min_{\B x \in \mathcal{K}_k(\B A, \B c)} f(\B x) &= \min_{\B y \in \mathbb{R}^k} f(\B Q_k \B y)  \\
&=\min_{\B y \in \mathbb{R}^k} \B y^T \B Q_k^T \B A \B Q_k \B y + \B c^T \B Q_k \B y \\
&= \min_{\B y \in \mathbb{R}^k} \B y^T \B T_k \B y + ||\B c||_2\B e_1^T \B y
\end{align*}
is minimized by $\B y = -||\B c||_2\B T_k^{-1} \B e_1 $.
%\sitem MINRES/GMRES (usually used to solve least squares problems) minimize residual of optimality equation: $||\B A \B x + \B c||_2^2$.
%\sitem Our goal is to obtain $\B x_r = \argmin_{\B x \in \mathcal{K}_r}f(\B x)$:
%\sitem The $r$th iteration conjugate gradient gives 
%\item there is insufficient information to minimize $||\B x_k - \B x^*||_2^2$
%\sitem The $r$th iterate of conjugate gradient minimizes this residual in $\B A^{-1}$ norm:
%\[\B x_r = \argmin_{\B x \in \mathcal{K}_r(\B A, \B c)}||\B A \B x + \B c||_{\B A^{-1}}^2=\argmin_{\B x \in \mathcal{K}_r(\B A, \B c)} \B x^T \B A \B x + 2 \B c^T\B x = \argmin_{\B x \in \mathcal{K}_r(\B A, \B c)} f(\B x).\]
%\[ = (\B c^T - \B A \B x_k)^T\B{A}^{-1}(\B c^T - \B A \B x_k)\]
\end{itemize}
}
\lgcond{}
\end{itemize}
%\item Conjugate gradient can be derived from vectors (columns of $\B Q_r$) generated by the Lanczos algorithm, in particular if $\B T_r = \B Q_r^T \B A \B Q_r$,
%\[\B x_r = -\B Q_r \B T_r^{-1} \B e_1 ||\B c||_2.\] 
%
%\lgcond{
%This iterate minimizes the projection of the residual onto $\B Q_r=\begin{bmatrix} \frac{\B c}{||\B c||_2} & \cdots \end{bmatrix}$,
%\begin{align*}
%\B 0 = 
%\B Q^T_r (\B c + \B A \B x_r) & = \B e_1 ||\B c||_2  + \underbrace{\B Q_r^T \B A\B Q_r}_{\B T_r} \B T_r^{-1} \B e_1||\B c ||_2
%\end{align*}
%}
\end{frame}


\section{General Nonlinear Optimization Methods}

\subsection{Newton's Method}

\begin{frame}{Newton's Method}

\urcornerlinkdemo{06-optimization}{Newton's Method in n dimensions}

\begin{itemize}
\item Newton's method in $n$ dimensions is given by finding minima of $n$-dimensional quadratic approximation:
\lgcond{
\[f(\B x_k+\B s)\approx \hat{f}(\B s) = f(\B x_k) + \B s^T \nabla f(\B x_k) + \frac{1}{2}\B s^T \B H_{f}(\B x_k)\B s.\]

The minima of this function can be determined by identifying critical points
\[\B 0=\nabla \hat{f}(\B s) = \nabla f(\B x_k) + \B H_{f}(\B x_k)\B s,\]
thus to determine $\B s$ we solve the linear system,
\[\B H_{f}(\B x_k)\B s =- \nabla f(\B x_k).\]
Assuming invertibility of the Hessian, we can write the Newton's method iteration as
\[\B x_{k+1} = \B x_k - \underbrace{\B H_{f}(\B x_k)^{-1}\nabla f(\B x_k)}_{\B s}.\]
Quadratic convergence follows by equivalence to Newton's method for solving nonlinear system of optimality equations \(\nabla f(\B x) = \B 0.\)
}
\lgcond{
}

\end{itemize}

\end{frame}

%\begin{frame}{Convergence of Newton's Method}
%
%\begin{itemize}
%\item Newton's method is quadratically convergent in $n$-dimensions as well:
%
%\lgcond{
%%Let $\B g(\B x)$ be the fixed-point iteration corresponding to Newton's method, using a Taylor expansion at the minima, which is valid in the limit of increasing $k$, we obtain
%\begin{align*}
%\B e_{k+1} &= \B x_{k+1} - \B x^* \\
%           &= \B x_k - \B x^* - \B H_{f}(\B x_k)^{-1}\nabla f(\B x_k) \\
%           &= -\B H_{f}(\B x_k)^{-1}(\B H_{f}(\B x_k)(\B x_k -\B x^*)- \nabla f(\B x_k)) \\
%           &= -\B H_{f}(\B x_k)^{-1}(\B H_{f}(\B x_k)(\B x_k -\B x^*)- (\nabla f(\B x_k) - \nabla f(\B x^*))), \\
%|| \B e_{k+1}|| &\leq ||\B H_{f}(\B x_k)^{-1}|| \underbrace{|| \B H_{f}(\B x_k)(\B x_k -\B x^*)- (\nabla f(\B x_k) - \nabla f(\B x^*) ||}_{O(||\B x_k - \B x^*||^2)} \\
%           &= C ||\B e_{k}||^2
%\end{align*}
%}
%\lgcond{}
%
%
%\end{itemize}
%
%\end{frame}

\subsection{Quasi-Newton Method}

\begin{frame}{Quasi-Newton Methods}

\begin{itemize}
\item \coloremph{Quasi-Newton} methods compute approximations to the Hessian at each step:

\lgcond{
\[\B x_{k+1}  = \B{x}_k -\alpha_k\B{B}_k^{-1}\nabla f(\B x_k)\]
where $\alpha_k$ is a line search parameter.
Quasi-Newton methods can be more robust than Newton's method, as the Newton's method step can lead to a direction in which the objective function is strictly increasing.
%Let $\B g(\B x)$ be the fixed-point iteration corresponding to Newton's method, using a Taylor expansion at the minima, which is valid in the limit of increasing $k$, we obtain
}

\item   The \coloremph{BFGS} method is a secant update method, similar to Broyden's method:

\lgcond{
\begin{itemize}
\sitem At each iteration, perform a rank-2 update to $\B B_{k}$ using $\B{s}_k = \B{x}_{k+1}-\B{x}_k$ and $\B{y_k} =\nabla f(\B{x}_{k+1})-\nabla f(\B{x}_{k})$:
\[\B{B}_{k+1} = \B{B}_k + \frac{\B{y}_k\B{y}_k^T}{\B{y}_k^T\B{s}_k} - \frac{\B{B}_k\B{s}_k \B{s}_k^T\B{B}_k}{\B{s}_k^T\B{B}_k \B{s}_k}\]
\sitem Can update inverse with $O(n^2)$ work, but its more stable and efficient to update a symmetric indefinite factorization.
\sitem The BFGS method also preserves symmetry of the Hessian approximation.
\end{itemize}
}


\end{itemize}

\end{frame}

\section{Nonlinear Least Squares}

\begin{frame}{Nonlinear Least Squares}

\begin{itemize}
\item An important special case of multidimensional optimization is \coloremph{nonlinear least squares}, the problem of fitting a nonlinear function $f_{\B x}(t)$ so that $f_{\B x}(t_i)\approx y_i$:

\mdcond{
For example, consider fitting $f_{[x_1,x_2]}(t) = x_1 \sin(x_2t)$ so that 
\[\begin{bmatrix} f_{[x_1,x_2]}(1.5) \\f_{[x_1,x_2]}(1.9)\\f_{[x_1,x_2]}(3.2) \end{bmatrix} \approx \begin{bmatrix} -1.2 \\ 4.5 \\ 7.3 \end{bmatrix}.\]
}
\item We can cast nonlinear least squares as an optimization problem and solve it by Newton's method:

\lgcond{
Define residual vector function $\B r(\B x)$ so that $r_i(\B x)= y_i-f_{\B x}(t_i)$ and minimize 
\[\phi(\B x)=\frac{1}{2}||\B r(\B x)||_2^2=\frac{1}{2}\B r(\B x)^T\B r(\B x).\]
Now the gradient is $\nabla \phi(\B x) = \B J_{\B r}^T(\B x)\B r(\B x)$ and the Hessian is 
\[\B{H}_{\phi}(\B x) = \B{J}^T_{\B r}(\B x)\B{J}_{\B r}(\B x) + \sum_{i=1}^m r_i(\B x)\B{H}_{r_i}(\B x).\]
}
\mdcond{}

\end{itemize}

\end{frame}


\begin{frame}{Gauss-Newton Method}

\begin{itemize}
\item The Hessian for nonlinear least squares problems has the form:

\lgcond{
%Note that the Hessian used in Newton's method is given by
\[\B{H}_{\phi}(\B x) = \B{J}^T_{\B r}(\B x)\B{J}_{\B r}(\B x) + \sum_{i=1}^m r_i(\B x)\B{H}_{r_i}(\B x).\]
The second term is small when the residual function $\B r(\B x)$ is small, so approximate
\[\B H_{\phi}(\B x) \approx \B{\hat{H}}_{\phi}(\B x) = \B J_{\B r}^T(\B x)\B J_{\B r}(\B x).\]
%and we can observe that the error is small if $||\B r(\B x)||_2$ is small,
%\[||\B H_{\phi}(\B x) - \B{\hat{H}}_{\phi}(\B x)||_F = ||\B r(\B x)||_2 \sum_{i=1}^m ||\B H_{r_i}(\B x)||_F\]
}
\item The \coloremph{Gauss-Newton} method is Newton iteration with an approximate Hessian:
\lgcond{
  \begin{align*}
  \B x_{k+1} &= \B x_{k} - \B{\hat{H}}_{\phi}(\B x_k)^{-1} \nabla f(\B x_k) 
             = \B x_{k} - (\B J_{\B r}^T(\B x_k)\B J_{\B r}(\B x_k))^{-1} \B J_{\B r}^T(\B x_k)\B r(\B x_k).
  \end{align*}
Recognizing the normal equations, we interpret the Gauss-Newton method as solving  linear least squares problems $\B J_{\B r}(\B x_k)\B s_k  \cong \B r(\B x_k), \B x_{k+1}= \B x_k + \B s_k$.
}
\item The Levenberg-Marquardt method incorporates Tykhonov regularization into the linear least squares problems within the Gauss-Newton method.
\end{itemize}

\end{frame}
%\begin{frame}{Levenberg-Marquardt Method}
%
%\begin{itemize}
%\item The \coloremph{Levenberg-Marquardt} method modifies the Gauss-Newton method to use Tykhonov regularization:
%
%\lgcond{
%  \begin{align*}
%  \B x_{k+1} &= \B x_{k} - (\B J_{\B r}^T(\B x_k)\B J_{\B r}(\B x_k) - \mu \B I)^{-1} \B J_{\B r}(\B x_k)\B r(\B x_k).
%  \end{align*}
%}
%
%\item The scalar $\mu$ controls the step size through the least squares problem:
%
%
%\lgcond{
%  \[\begin{bmatrix} \B J_{\B r}(\B x_k) \\ \sqrt{\mu} \B I \end{bmatrix} \B s_k \cong \begin{bmatrix} \B r(\B x_k) \\ \B 0 \end{bmatrix}\]
%By increasing $\mu$ we force the step $\B s$ to be smaller.
%}
%\end{itemize}
%
%\end{frame}




\section{Constrained Optimization Methods}

\subsection{Overview of Approaches}

%\end{document}


\begin{frame}{Constrained Optimization Problems}

\begin{itemize}
\item We now return to the general case of \coloremph{constrained} optimization problems:
\lgcond{
\[\min_{\B x} f(\B x)\quad \text{subject to} \quad \B g(\B x) = \B 0 \quad \text{and} \quad \B{h}(\B x) \leq \B 0\]

When $f$ is quadratic, while $h$, $g$ is linear, this is a \coloremph{quadratic optimization} problem.
}

\item Generally, we will seek to reduce constrained optimization problems to a series of unconstrained optimization problems:

%\lgcond{
\begin{itemize}
\mitem \coloremph{sequential quadratic programming}: \filufil{solve an unconstrained quadratic optimization problem at each iteration,}{\medskip}
\mitem \coloremph{penalty-based methods}: \filufil{solve a series of more complicated (more ill-conditioned) unconstrained optimization problems,}{\medskip}
\mitem \coloremph{active set methods}: \filufil{define sequence of optimization problems with inequality constrains ignored or treated as equality constraints.}{\medskip}
\end{itemize}
%}
\end{itemize}

\end{frame}


%\begin{frame}{Constrained Optimality}
%
%\begin{itemize}
%\item In equality-constrained optimization $\B g(\B x)=\B 0$, minimizers $\B x^*$ are on the border of the feasible region (set of points satisfying constraints),
%      in which case we must ensure any direction of decrease of $f$ from $\B x^*$ leads to an infeasible point, which gives us the condition:
%
%\[\exists \B \lambda\in\mathbb{R}^n, \quad -\nabla f(\B x^*) = \B J^T_{\B g}(\B x^*) \B \lambda\]
%
%\mdcond{
%$\B \lambda$ are referred to as the Lagrange multipliers.
%}
%
%\item Seek critical points in the Lagrangian function $\mathcal{L}(\B x, \B \lambda) = f(\B x) + \B \lambda^T\B g(\B x)$, described by the nonlinear equation,
%
%\[\nabla \mathcal{L}(\B x, \B \lambda) = \begin{bmatrix} \nabla f(\B x) + \B J_{\B g}^T(\B x) \B \lambda \\ \B g(\B x)\end{bmatrix} = \B 0\]
%
%\lgcond{
%Seeking $\B \lambda$ that maximizes the global minimum of $\mathcal{L}(\B x, \B \lambda)$ defines the \coloremph{dual optimization problem}.
%}
%\mdcond{}
%
%
%\end{itemize}
%
%
%\end{frame}

\subsection{Optimization with Equality Constraints}

%\begin{frame}{Sequential Quadratic Programming}
%
%\begin{itemize}
%\item \coloremph{Sequential quadratic programming (SQP)} reduces a nonlinear equality constrained problem to a sequence of constrained quadratic programs
%via a Taylor expansion of the Lagrangian function $\mathcal{L}_f(\B x, \B \lambda)=f(\B x)+\B \lambda ^T\B g(\B x)$:
%\lgcond{
%\begin{align*} 
%q(\B x_k+ \B s,\B \lambda_k + \B \delta ) =& \mathcal{L}_f(\B x_k, \B \lambda_k) + \B s^T (\nabla f(\B x_k) + \B J_{\B g}^T(\B x_k)\B \lambda_k) + \frac{1}{2}\B s^T \B B(\B x_k, \B \lambda_k)\B s \\
%&+\B \delta^T (\B J_{\B g}(\B x_k) \B s + \B g(\B x_k))
%\end{align*}
%where $\B B(\B x, \B \lambda) = \B H_{f}(\B x) + \sum_{i=1}^m \lambda_i \B H_{g_i}(\B x)$
%}
%\item SQP ignores the constant term $\mathcal{L}_f(\B x_k, \B \lambda_k)$ and minimizes $\B s$ while treating $\B \delta$ as a Lagrange multiplier:
%
%\lgcond{
%The above unconstrained quadratic program corresponds to the Lagrangian form of the constrained quadratic program
%\begin{align*} 
%\max_{\B s} \B s^T (\nabla f(\B x_k) + \B J_{\B g}^T(\B x_k)\B \lambda_k) + \frac{1}{2}\B s^T \B B(\B x_k, \B \lambda_k)\B s
%\end{align*}
%with constraint $\B J_{\B g}(\B x_k) \B s = -\B g(\B x_k)$.
%
%%and use a linear approximation of the constraint $\B g(\B x_k + \B s)=\B 0$, 
%%subject to the linear constraint approximation $\B J_{\B g}(\B x_k) \B s = -\B g(\B x_k)$.
%}
%
%%\item We can reduce a constrained quadratic program to an unconstrained one, using the Lagrangian function of $q$:
%%\lgcond{
%%\[\mathcal{L}_q(\B x_k+ \B s,\B \lambda + \B \delta, \B \mu ) = f(\B x_k) + \B s^T (\nabla f(\B x_k) + \B J_g^T(\B x_k)\B \lambda_k) + \frac{1}{2}\B s^T \B B(\B x_k)\B s + \mu(\B J_{\B g}(\B x_k) \B s +\B g(\B x_k))\]
%%which we can solve directly as a linear system.
%%%\[\mathcal{L}_{\hat{f}}(\B x_k+ \B s,\B \lambda) = f(\B x_k) + \B s^T \nabla f(\B x_k) + \frac{1}{2}\B s^T \B H_{f}(\B x_k)\B s + \B \lambda^T(\B J_{\B g}(\B x_k) \B s + \B g(\B x_k)) \]
%%The minima of $q$ will correspond to a critical point of $\mathcal{L}_{\hat{f}}$, which may or may not be the true minima.
%%}
%%
%\end{itemize}
%
%
%\end{frame}


\begin{frame}{Sequential Quadratic Programming}

\urcornerlinkdemo{06-optimization}{Sequential Quadratic Programming}

\begin{itemize}
\item \coloremph{Sequential quadratic programming} (SQP) corresponds to using Newton's method to solve the equality constrained optimality conditions, by finding critical points of the Lagrangian function $\mathcal{L}(\B x, \B \lambda) = f(\B x) + \B \lambda^T\B g(\B x)$,
\mdcond{
\[\nabla \mathcal{L}(\B x, \B \lambda) = \begin{bmatrix} \nabla f(\B x) + \B J_{\B g}^T(\B x) \B \lambda \\ \B g(\B x)\end{bmatrix} = \B 0\]
}

\item At each iteration, SQP computes $\begin{bmatrix} \B x_{k+1} \\ \B \lambda_{k+1}\end{bmatrix} = \begin{bmatrix} \B x_{k}\\ \B \lambda_{k}\end{bmatrix} + \begin{bmatrix} \B s_k \\ \B\delta_k \end{bmatrix}$ by solving
\lgcond{
\[\B H_{\mathcal{L}}(\B x_k, \B \lambda_k)\begin{bmatrix} \B s_k \\ \B\delta_k \end{bmatrix} =  -\nabla \mathcal{L}(\B x_k, \B \lambda_k)\]
% = -\begin{bmatrix} \nabla f(\B x_k) + \B J_{\B g}^T(\B x_k) \B \lambda_k \\ \B g(\B x_k)\end{bmatrix}
where 
\[\B H_{\mathcal{L}}(\B x_k, \B \lambda_k) = \begin{bmatrix} \B B(\B x_k, \B\lambda_k) & \B J^T_{\B g}(\B x_k) \\ \B J_{\B g}(\B x_k) & \B 0\end{bmatrix} \quad \text{with} \quad \B B(\B x, \B \lambda) = \B H_{f}(\B x) + \sum_{i=1}^m \lambda_i \B H_{g_i}(\B x)\]
}

\end{itemize}


\end{frame}

%\begin{frame}{Quadratic Programming Problems}
%
%\begin{itemize}
%\item An equality-constrained quadratic programming problem has the form
%\[\min_{\B x} f(\B x), \quad f(\B x)=\frac{1}{2} \B x^T \B Q \B x + \B c^T \B x \quad \text{subject to} \quad \B A \B x = \B b\]
%where $\B Q$ is symmetric.
%
%\lgcond{
%Its first-order optimality condition in the unconstrained case is
%\[\B 0 = \nabla f(\B x)  = \B Q \B x + \B c \quad \Rightarrow \quad \B Q \B x =-\B c\]
%and in the constrained case, becomes 
%\[ \begin{bmatrix} \B Q & \B A^T \\ \B A & \B 0 \end{bmatrix} \begin{bmatrix} \B x \\ \B \lambda \end{bmatrix} = - \begin{bmatrix} \B c \\ \B b \end{bmatrix} \]
%This allows us to observe, that at the $k$th iteration SQP solves a QP with $\B Q =\B B(\B x_k, \B\lambda_k)$, $\B A =\B J_g(\B x_k)$, $\B c =\nabla f(\B x_k) + \B J_{\B g}^T(\B x_k)$, and $\B b = \B g(\B x_k)$.
%}
%\end{itemize}
%
%
%\end{frame}
%
%\begin{frame}{Steepest Descent for Quadratic Programming}
%
%\begin{itemize}
%\item Near the minima, all smooth nonlinear programming problems look like quadratic programming problems, where $\B Q$ converges to the Hessian at the minima, $\B H_f(\B x^*)$.
%
%\lgcond{}
%
%\item Consequently, we can analyze local convergence of methods by considering their convergence for a QP, e.g. for steepest descent:
%
%\lgcond{
%Assume $\B b =\B 0$ and $\B c =\B 0$, then we have $\nabla f(\B x) = \B Q \B x$.
%Consequently, steepest descent will seek a step-size $\alpha_k$ to perform the step
%\[\B x_{k+1} = \B x_{k} -\alpha_k \B Q \B x_k = (\B I - \alpha_k \B Q) \B x_k\]
%for this fixed-point iteration to converge, it suffices that $\alpha_k<2/\lambda_{\text{max}}(\B Q)$.
%The optimal value can be shown to be $\alpha^*=2/(\lambda_{\text{max}}(\B Q) + \lambda_{\text{min}}(\B Q))$, leading to convergence rate
%\[||\B e_{k+1}|| = \frac{\kappa(\B Q) -1}{\kappa(\B Q) +1}||\B e_k||\]
%}
%
%\end{itemize}
%
%\end{frame}

\subsection{Optimization with Inequality Constraints}

\begin{frame}{Inequality Constrained Optimality Conditions}

\begin{itemize}


\item The \coloremph{Karush-Kuhn-Tucker (KKT)} conditions hold for local minima of a problem with equality and inequality constraints, the key conditions are

\lgcond{
\begin{itemize}
\sitem First, any minima $\B x^*$ must be a feasible point, so $\B g(\B x^*) =\B 0$ and $\B h(\B x^*) \leq \B 0$.
\sitem We say the $i$th inequality constraint is \coloremph{active} at a minima $\B x^*$ if $h_i(\B x^*) = 0$.
\sitem The collection of equality constraints and active inequality constraints $\B q^*$, satisfies $\B q^*(\B x^*) = \B 0$.
\sitem The negative gradient of the objective function at the minima must be in the row span of the Jacobian of this collection of constraints:
\[-\nabla f(\B x^*)= \B J_{\B q^*}^T(\B x^*)\B \lambda^* \quad \text{where $\B \lambda^*$ are Lagrange multiplers of constraints in $\B q^*$} .\]
\end{itemize}
}
%for the general Lagrangian function $\mathcal{L}(\B x, \B \mu, \B \nu) = f(\B x)+\B \mu^T \B g(\B x) + \B \nu^T \B h(\B x)$ are 
%%\lgcond{
%\begin{align*}
%\nabla_{\B x}\mathcal{L} (\B x, \B \lambda) & = \B 0\\
%\B g(\B x) &= \B 0\\
%\B h(\B x) &\leq \B 0\\
%\B \nu &\geq \B 0\\
%\B \nu^T \B h (\B x)  &= \B 0
%\end{align*}
%\smcond{
%At an optimal point, we must have that for either the $i$th inequality constraint is \coloremph{active}, so $h_i(\B x)=\B 0$ or it is inactive, but its Lagrange multiplier $\nu_i=0$.
%}
\item %SQP can handle only equality constraints, 
To use SQP for an inequality constrained optimization problem, consider at each iteration an \coloremph{active set} of constraints:

\lgcond{
\begin{itemize}
\sitem Active set $\B q_k$ contains all equality constraints and all inequality constraints that are exactly satisfied or violated at $\B x_k$.
\sitem Perform one step of Newton's method to minimize $\mathcal{L}_k(\B x, \B \lambda) = f(\B x) + \B \lambda^T \B q_k(\B x)$ with respect to $\B x$ and $\B \lambda$, then update active set.
%\sitem A minima $\B x^*$ will be a feasible point and if the constraints in the active set at $x^*$ are $\B q(\B x)$ with Lagrange multipliers $\B \lambda$, then 
\end{itemize}
%A constraint is active if the current iterate is at a boundary satisfies the inequality constraint as an equality (i.e. $i$th constraint is active if $i$th component of $\B h(\B x_k)$ is zero).
}

\end{itemize}

\end{frame}


%\begin{frame}{Optimality Conditions for Inequality Constraints}
%
%\begin{itemize}
%\item The Lagrangian function with constraints $\B g(\B x) = \B 0$ and $\B{h}(\B x) \leq \B 0$ is
%\lgcond{
%\[\mathcal{L}(\B x, \B \lambda) = f(\B x) + \B \lambda^T \begin{bmatrix} \B h(\B x) \\ \B g(\B x)\end{bmatrix}\]
%
%%\[\mathcal{L}(\B x, \B \lambda) = f(\B x) +  \B \lambda_1^T \B h(\B x)  + \B \lambda_2^T \B g(\B x)\]
%}
%
%\item The Lagrangian dual problem is an unconstrained optimization problem:
%\[\max_{\B \lambda} q(\B \lambda), \quad q(\B \lambda) =   \begin{cases}  \min_{\B x} \mathcal{L}(\B x, \B \lambda) & \text{if } \B \lambda \geq \B 0 \\   -\infty & \text{otherwise}\end{cases}\]
%The unconstrained optimality condition $\nabla q(\B \lambda^*) = \B 0$, implies 
%\lgcond{
%\[\max\left(\B \lambda^*,\begin{bmatrix} \B h(\B x) \\ \B g(\B x)\end{bmatrix}\right) = \B 0\]
%when $\lambda_i^*= 0$, we say the $i$th constraint is inactive at the minimum point.
%Equality constraints are active at any feasible point.
%}
%
%
%\end{itemize}
%
%
%\end{frame}


%\begin{frame}{Merit Functions}
%
%\begin{itemize}
%\item We can reduce constrained optimization problems to unconstrained ones by modifying the objective function. \coloremph{Penalty} methods are effective for equality constraints $\B g(\B x)=0$:
%
%\lgcond{
%A simple penalty function parameterized by $\rho >0$ is
%\[\phi_\rho(\B x) = f(\B x) + \frac{1}{2} \rho \B g(\B x)^T \B g(\B x)\]
%a more robust variant is the augmented Lagrangian function
%\[\mathcal{L}_\rho(\B x,\B \lambda) = f(\B x) + \B \lambda^T \B g(\B x)+\frac{1}{2} \rho \B g(\B x)^T \B g(\B x)\]
%by increasing $\rho$, in theory convergence to an optimal solution can be expected, however, we must insure the Hessian of $\phi$ is not too ill-conditioned.
%
%}
%
%\item Barrier functions provide an effective way (\coloremph{interior point methods}) of working with inequality constraints $\B h(\B x)\leq \B 0$:
%
%\lgcond{
%To enforce inequality constraints via the objective function, we can introduce barriers, like
%\[\phi_\mu(\B x) = f(\B x) - \mu \sum_{i=1}^p\frac{1}{h_i(\B x)}\]
%which is called the inverse barrier function, or the logarithmic barrier function
%\[\phi_\mu(\B x) = f(\B x) - \mu \sum_{i=1}^p\log(-h_i(\B x))\]
%hoping to obtain convergence by decreasing $\mu$ and ensuring the step size is sufficiently small.
%
%}
%
%
%
%\end{itemize}
%
%\end{frame}



\begin{frame}{Penalty Functions}

\begin{itemize}
\item Alternatively, we can reduce constrained optimization problems to unconstrained ones by modifying the objective function. \coloremph{Penalty} functions are effective for equality constraints $\B g(\B x)=0$:

\lgcond{
\[\phi_\rho(\B x) = f(\B x) + \frac{1}{2} \rho \B g(\B x)^T \B g(\B x)\]
is a simple merit function, and its solutions $\B x^*_\rho$ satisfy $\lim_{\rho \to \infty} \B x_\rho^* = \B x^*$.
However, the Hessian of $\phi_\rho$ becomes increasingly ill-conditioned for large $\rho$, leading to slow convergence.
}

\item The augmented Lagrangian function provides a more numerically robust approach:
\lgcond{
\[\mathcal{L}_\rho(\B x, \B \lambda) = f(\B x) + \B \lambda^T\B g(\B x) + \frac{1}{2} \rho \B g(\B x)^T \B g(\B x)\]
}
\end{itemize}

\end{frame}

\begin{frame}{Barrier Functions}

\begin{itemize}

\item \coloremph{Barrier functions} (\coloremph{interior point methods}) provide an effective way of working with inequality constraints $\B h(\B x)\leq \B 0$:

\lgcond{
\begin{itemize}
\item Provided we start at a feasible point, modify objective function so it diverges to $\infty$ when approaching border of feasible region.
\mitem Inverse barrier function: \[\phi_\mu(\B x) = f(\B x) - \mu\sum_{i=1}^m\frac{1}{h_i(\B x)}.\]
\mitem Logarithmic barrier function: \[\phi_\mu(\B x) = f(\B x) - \mu\sum_{i=1}^m\log(-h_i(\B x)).\]
%The minimum $\phi_\mu$ is guaranteed to be in the feasible domain and 
\mitem When using sufficiently small steps, we have $\B x^*_\mu\to \B x^*$ as $\mu \to 0$.
\end{itemize}
}
\lgcond{}



\end{itemize}

\end{frame}

%\begin{frame}{Linear Programming}
%
%\begin{itemize}
%
%\item A linear program is a quadratic program with $\B Q = \B O$:
%
%\mdcond{}
%
%\item The simplex method is effective for linear programs, and operates by using the fact that the minimum over a convex polyhedron must occur at one of its vertices:
%
%\lgcond{}
%
%\item The sequential linear-quadratic programming algorithm provides an effective way of solving linear programs iwth inequality constraints:
%
%\mdcond{}
%
%\end{itemize}
%
%\end{frame}
%\begin{frame}{Optimization with Data}
%
%\begin{itemize}
%
%\item Lets consider the matrix completion problem, which is a nonlinear optimization problem:
%
%\lgcond{
%Given a subset of entries
%\[\Omega\subseteq \{1,\ldots,n\}\times \{1,\ldots, n\}\]
%of the entries of matrix $\Mat{A}\in \rmn nn$, seek rank-$k$ approximation
%\[\argmin_{\Mat{W}\in \rmn mk, \Mat{H}\in \rmn nk} \sum_{(i,j)\in \Omega}\Big(\underbrace{a_{ij}-\sum_l w_{il}h_{jl}}_{(\Mat{A}-\Mat{W}\Mat{H}^T)_{ij}}\Big)^2 + \lambda(||\Mat{W}||_F + ||\Mat{H}||_F)\]
%}
%\lgcond{}
%\end{itemize}
%
%\end{frame}
%
%
%\begin{frame}{Alternating Least Squares}
%
%\begin{itemize}
%\item Fix $\Mat{W}$ and solving for $\Mat{H}$ by quadratic optimization:
%%\[\argmin_{\Mat{W}\in \rmn mk} \sum_{(i,j)\in \Omega} \Big(a_{ij}-\sum_l {w}_{il}{h}_{jl}\Big)^2 + \lambda||\Mat{W}||_F\]
%%\lgcond{The optimization problem is independent for rows of $\Mat{W}$. Letting $\Vec{w}_i=w_{i\ALL}$, $\Vec{h}_i=h_{i\ALL}$, $\Omega_i =\{j : (i,j)\in\Omega\}$, seek
%\[\argmin_{\Vec{w}_i\in \rn k}  \sum_{j\in \Omega_i} \Big(a_{ij}-\Vec{w}_i\Vec{h}_j^\T\Big)^2 + \lambda||\Vec{w}_i||_2\]
%%}
%%\item
%\lgcond{
%Seek minimizer $\Vec{w}_i$ for quadratic vector equation
%\[f(\Vec{w}_i) = \sum_{j\in \Omega_i} \Big(a_{ij}-\Vec{w}_i\Vec{h}_j^\T\Big)^2 + \lambda||\Vec{w}_i||\]
%
%Differentiating with respect to $\Vec{w}_i$ gives
%\[\frac{\partial f(\Vec{w}_i)}{\partial \Vec{w}_i} = 2\sum_{j\in \Omega_i}\Vec{h}_j^\T\Big( \Vec{w}_i\Vec{h}_j^\T-a_{ij}\Big) + 2\lambda \Vec{w}_i=0\]
%Rotating $\Vec{w}_i\Vec{h}_j^\T=\Vec{h}_j\Vec{w}_i^\T$ and defining $\Mat{G}^{(i)}=\sum_{j\in\Omega_i}\Vec{h}_j^\T \Vec{h}_j$,
%%$\in \rmn{|\Omega_i|}{k}$ 
%%=[\Vec{h}_{j_1}; \Vec{h}_{j_2};...]$ $$
%\[(\Mat{G}^{(i)}+\lambda \Mat{I})\Vec{w}_i^T = \sum_{j\in \Omega_i}\Vec{h}_j^\T a_{ij}\]
%which is a $k\times k$ symmetric linear system of equations
%}
%\lgcond{}
%\end{itemize}
%
%\end{frame}
%
%\begin{frame}{Gradient Descent for Matrix Completion}
%
%\begin{itemize}
%\mitem Recall that we seek to minimize
%\[f(\Vec{w}_i) = \sum_{j\in \Omega_i} \Big(a_{ij}-\Vec{w}_i\Vec{h}_j^\T\Big)^2 + \lambda||\Vec{w}_i||\]
%ALS minimizes $\Vec{w}_i$, gradient descent methods only improve it using the partial derivative
%\[\frac{\partial f(\Vec{w}_i)}{\partial \Vec{w}_i} = 2\sum_{j\in \Omega_i}\Vec{h}_j^\T\Big( \Vec{w}_i\Vec{h}_j^\T-a_{ij}\Big) + 2\lambda \Vec{w}_i
%%\item we can use $r_{ij}=a_{ij}-\Vec{h}_j^\T \Vec{w}_i$ to write this as
%%\[\frac{\partial f(\Vec{w}_i)}{\partial \Vec{w}_i} 
%=   2\bigg(\lambda \Vec{w}_i-\sum_{j\in \Omega_i}r_{ij}\Vec{h}_j\bigg) \]
%\mdcond{}
%\item \coloremph{Gradient descent} method updates
%\[\Vec{w}_i=\Vec{w}_i-\eta \frac{\partial f(\Vec{w}_i)}{\partial \Vec{w}_i}\]
%\lgcond{
%where parameter $\eta$ is our step-size
%}
%\end{itemize}
%
%\end{frame}
%
%\begin{frame}{Stochastic Gradient Descent (SGD)}
%
%\begin{itemize}
%\item \coloremph{Stochastic gradient descent (SGD)} performs fine-grained updates based on a component of the overall gradient
%\[\frac{\partial f(\Vec{w}_i)}{\partial \Vec{w}_i}  
%=   2\bigg(\lambda \Vec{w}_i-\sum_{j\in \Omega_i}r_{ij}\Vec{h}_j\bigg) 
%=   2\sum_{j\in \Omega_i}\lambda \Vec{w}_i/|\Omega_i|-r_{ij}\Vec{h}_j \]
%where $\B R$ is the residual on entries $\Omega$.
%
%\lgcond{
%SGD selects random $(i,j)\in\Omega$ and updates $\Vec{w}_i$ using $\Vec{h}_j$
%\[\Vec{w}_i\leftarrow \Vec{w}_i-\eta(\lambda \Vec{w}_i/|\Omega_i| - r_{ij}\Vec{h}_j)\]
%}
%%\item SGD randomly selects pairs $(i,j)\in\Omega$ and updates $\Vec{w}_i$ (and $\Vec{h}_j$ in a dual fashion)
%\item SGD then updates $r_{ij}=a_{ij}-\Vec{w}_i^\T \Vec{h}_j$, for a low overall cost:
%
%\lgcond{
%Each update costs $O(k)$ operations if $k$ is the rank.
%Stochastic properties make the method very effective in practice.
%Techniques such as using momentum and extrapolation apply also in this context.
%}
%%\item $O(|\Omega|)$ yield the same total cost as CCD-based updates of $W$ and $H$
%\end{itemize}
%
%\end{frame}
%
%
%\begin{frame}{Coordinate Descent}
%
%Rather than whole rows $\Vec w_i$ solve for elements of $\Mat{\Mat{W}}$, recall
%\[\argmin_{\Mat{W}\in \rmn mk} \sum_{(i,j)\in \Omega} \Big(a_{ij}-\sum_l w_{il}h_{jl}\Big)^2 + \lambda||\Mat{W}||_F\]
%\begin{itemize}
%\item \coloremph{Coordinate descent} finds the best replacement $\prm$ for $w_{it}$
%\[\prm = \argmin_{\prm} \sum_{j\in \Omega_i} \Big(a_{ij}-\prm h_{jt}-\sum_{l\neq t} w_{il}h_{jl}\Big)^2 + \lambda \prm^2\]
%
%\mdcond{}
%
%\item The solution is given by
%%\[\prm= \frac{\sum_{j\in \Omega_i} h_{jt}\Big(a_{ij}-\sum_{l\neq t} w_{il}h_{jl}\Big)}{
%%\lambda +\sum_{j\in\Omega_i}h_{jt}^2}\]
%\[\prm= \frac{\sum_{j\in \Omega_i} h_{jt}\Big(a_{ij}-\sum_{l\neq t} w_{il}h_{jl}\Big)}{
%\lambda +\sum_{j\in\Omega_i}h_{jt}^2}
%%
%%can be computed as
%%\[\prm= 
%=\frac{\sum_{j\in \Omega_i} h_{jt}\Big(r_{ij}+ w_{it}h_{jt}\Big)}{
%\lambda +\sum_{j\in\Omega_i}h_{jt}^2}\]
%
%\mdcond{
%after which we can update $\Mat{R}$ via
%\[r_{ij}\gets r_{ij}-(\prm-w_{it})h_{jt}\quad \forall j\in \Omega_i\]
%}
%
%
%\end{itemize}
%
%\end{frame}





%\end{document}
