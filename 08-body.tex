%%\documentclass[handout]{beamer}
%\documentclass[aspectratio=169,13pt]{beamer}

%\newcommand{\dee}{\partial}

\newcommand{\bmat}[1]{\begin{bmatrix} \B{#1}_{11} & \B{#1}_{12} \\ \B{#1}_{21} & \B{#1}_{22}\end{bmatrix}}
\newcommand{\blmat}[1]{\begin{bmatrix} \B{#1}_{11} &  \\ \B{#1}_{21} & \B{#1}_{22}\end{bmatrix}}
\newcommand{\brmat}[1]{\begin{bmatrix} \B{#1}_{11} & \B{#1}_{12} \\ & \B{#1}_{22}\end{bmatrix}}
\newcommand{\bomat}[1]{\begin{bmatrix} {#1}_{11} & \B{#1}_{12} \\ \B{#1}_{21} & \B{\uppercase{#1}}_{22}\end{bmatrix}}
\newcommand{\blomat}[1]{\begin{bmatrix} {#1}_{11} & \\ \B{#1}_{21} & \B{\uppercase{#1}}_{22}\end{bmatrix}}
\newcommand{\bromat}[1]{\begin{bmatrix} {#1}_{11} & \B{#1}_{12} \\  & \B{\uppercase{#1}}_{22}\end{bmatrix}}

\DeclareMathOperator{\vcop}{vec}
\newcommand{\vc}[1]{\vcop\left(#1\right)}
\newcommand{\prm}{\mu}

\newcommand{\fl}{fl}
\newcommand{\asmcond}[1]{\smallskip{\it #1}\smallskip}
\newcommand{\amdcond}[1]{\smallskip{\it #1}\smallskip}
\newcommand{\algcond}[1]{\smallskip{\it #1}\smallskip}
\newcommand{\atsmcond}[1]{{\it #1}}
\newcommand{\atmdcond}[1]{{\it #1}}
\newcommand{\atlgcond}[1]{{\it #1}}
\newcommand{\afillnum}[1]{{\underline{ \ #1\ \ }}}

\mode<presentation>
{
  \usetheme{Malmoe}
}

\usepackage[english]{babel}
%
%\usepackage[latin1]{inputenc}
%
%\usepackage{times}
%\usepackage[T1]{fontenc}
% Or whatever. Note that the encoding and the font should match. If T1
% does not look nice, try deleting the line with the fontenc.

\usepackage{inconsolata}
\usepackage{listings}
\usepackage{bm}
\usefonttheme[onlymath]{serif}
\renewcommand\familydefault{\sfdefault}
\usepackage[sfdefault]{ClearSans} %% option 'sfdefault' activates Clear Sans as the default text font
%\usefonttheme{professionalfonts}

\usepackage{latexsym}
\usepackage{amsmath}
\usepackage{amssymb}
\usepackage{amsfonts}
\usepackage{graphics}
\usepackage{multicol}

\definecolor{darkred}{rgb}{0.8,0.2,0.2}
\newcommand{\coloremph}[1]{\textcolor{darkred}{\emph{#1}}}        % colored italics
\newcommand{\reshape}[2]{o_{#1}(#2)}        % colored italics
\newcommand{\ket}[1]{\lvert #1 \rangle}
\newcommand{\bra}[1]{\langle #1 \vert}
\newcommand{\transp}[2]{{#2}^{\langle  {#1} \rangle }}        % colored italics

% Miscellaneous special math symbols

% Big-oh notation - either Latex's calligraphic O or uppercase math italic O
\newcommand{\BIGOH}{\mathcal{O}}                              % big oh
%\newcommand{\BIGOH}{O}                                       % big oh
\newcommand{\BIGTHETA}{\Theta}                                % big theta

\newcommand{\sfrac}[2]{{#1}/{#2}}

% Various versions of the fraction one-half
% solidus 1/2 for superscripts:
\newcommand{\SHALF}{1/2}                                      % one-half power
% small 1/2 set case in displayed equations:
\newcommand{\HALF}{\mbox{\small $\frac{1}{2}$}}               % small one-half
\newcommand{\THRD}{\mbox{\scriptsize $\frac{1}{3}$}}          % small one-third
\newcommand{\TWOTH}{\mbox{\scriptsize $\frac{2}{3}$}}         % small two-thirds

% Machine epsilon - note that placement of subscript may need adjustment.
%\newcommand{\emach}{\epsilon_{\textrm{\scriptsize mach}}} % machine prec.
\newcommand{\emach}{\epsilon}

\newcommand{\lsapprox}{\cong}                             % least squares approx

% Real numbers - Prefer \mathbb{R} but it requires amsfonts.
%                Latex's calligraphic R will do if amsfonts are unavailable.
%                DO NOT use \Re, which gives old German fraktur R.
\newcommand{\Real}{\mathbb{R}}                                 % real numbers
\newcommand{\Cplx}{\mathbb{C}}                                 % complex numbers
\newcommand{\Poly}{\mathbb{P}}                                 % polynomials
\newcommand{\Float}{\mathbb{F}}                                % fl pt system
% Bold math fonts for vectors and matrices
\renewcommand{\Vec}[1]{\ensuremath{\bm{#1}}}                   % vector
\newcommand{\Mat}[1]{\ensuremath{\bm{#1}}}                     % matrix
\newcommand{\Op}[1]{#1}                              % operator
\newcommand{\loc}[1]{x_{#1}}                              % operator

% Loglike functions set in regular type
\newcommand{\diag}{\mathrm{diag}}                             % diagonal matrix
\newcommand{\cond}{\mathrm{cond}}                             % condition number
\newcommand{\sign}{\mathrm{sign}}                             % sign function
\newcommand{\Span}{\text{span}}                             % span of matrix
%\newcommand{\trace}{\mathrm{trace}}                           % trace of matrix
\DeclareMathOperator*{\trace}{trace}
\newcommand{\tentrace}[3]{\trace_{#1,#2}(#3)}
\renewcommand{\Re}{\mathrm{Re}}                               % real part
\renewcommand{\Im}{\mathrm{Im}}                               % imaginary part

% Fractions appearing in matrices - allows setting them case or solidus
%\newcommand{\mf}[2]{{#1 \over #2}}        % matrix fraction - set case
\newcommand{\mf}[2]{#1/#2}               % matrix fraction - set solidus


% Keywords for algorithm statements

\newcommand{\FOR}{\textbf{for}\ }
\newcommand{\TO}{\textbf{to}\ }
\newcommand{\IF}{\textbf{if}\ }
\newcommand{\THEN}{\textbf{then}\ }
\newcommand{\ELSE}{\textbf{else}\ }
\newcommand{\WHILE}{\textbf{while}\ }
\newcommand{\DO}{\textbf{do}\ }
\newcommand{\BEGIN}{\textbf{begin}\ }
\newcommand{\END}{\textbf{end}\ }
\newcommand{\STOP}{\textbf{stop}\ }
\newcommand{\AND}{\textbf{and}\ }
\newcommand{\OR}{\textbf{or}\ }

\newcommand{\comm}{\mathrm{comm}}
\newcommand{\comp}{\mathrm{comp}}
\newcommand{\idle}{\mathrm{idle}}
\newcommand{\msg}{\mathrm{msg}}
\newcommand{\len}{\mathrm{s}}
\newcommand{\route}{\mathrm{route}}
\newcommand{\mitem}{\medskip\item}
\newcommand{\sitem}{\smallskip\item}
\DeclareMathOperator*{\argmin}{argmin}

%Gets rid of headline

\setbeamertemplate{footline}{}

\setbeamertemplate{headline}{}
\beamertemplatenavigationsymbolsempty



\definecolor{mygreen}{rgb}{0,0.2,0}
\definecolor{mygray}{rgb}{0.5,0.5,0.5}
\definecolor{mymauve}{rgb}{0.58,0,0.82}
\definecolor{mypurple}{rgb}{0.38,0,0.32}
\definecolor{myblue}{rgb}{0.2,0,0.5}
\definecolor{darkgreen}{rgb}{0.2,0.6,0.2}
\definecolor{Brown}{rgb}{0.39, 0.09, 0.0}
\definecolor{Violet}{rgb}{0.38, 0.0, 0.63}
\definecolor{myvlgray}{rgb}{0.9,0.9,1.0}
\definecolor{mylgray}{rgb}{0.85,0.85,0.9}
\definecolor{darkgreen}{rgb}{0,0.4,0}
\definecolor{brown}{rgb}{.6,.1,.1}

\newcommand{\bemph}[1]{{\color{blue} \emph{#1}}}

\usepackage{mathtools}
\newcommand{\defeq}{\coloneqq}

\newcommand{\inti}[2]{[{#1},{#2}]}
\newcommand{\BF}{\mathbf}
\newcommand{\CF}{\mathcal}
\newcommand{\B}[1]{\bm{#1}}
\newcommand{\E}[1]{\bm{#1}}
\newcommand{\dis}{\displaystyle}
\newcommand{\lt}{\left}
\newcommand{\rt}{\right}

\newcommand{\cs}{H}

\DeclareMathOperator*{\rank}{rank}
\DeclareMathOperator*{\vecn}{vec}
\DeclareMathOperator*{\vecs}{vech}

\usepackage{eso-pic}
\newcommand{\cornertext}[1]{
  \AddToShipoutPictureFG*{
    \AtPageUpperLeft{\put(0,-10){\makebox[\paperwidth][r]{#1}}}  
   }%
}
\newcommand{\cornertexttwo}[2]{
  \AddToShipoutPictureFG*{
    \AtPageUpperLeft{\put(0,-10){\makebox[\paperwidth][r]{#1}}}  
    \AtPageUpperLeft{\put(0,-20){\makebox[\paperwidth][r]{#2}}}  
   }%
}
\newcommand{\linkdemo}[2]{{\footnotesize\href{https://relate.cs.illinois.edu/course/cs450-f18/f/demos/upload/#1/#2.html}{\color{Violet}{\it\textbf{Demo:} #2}\ }}}
\newcommand{\linkinclass}[2]{{\footnotesize\href{https://relate.cs.illinois.edu/course/cs450-f18/flow/#1/start/}{\color{darkgreen}{\it\textbf{Activity:} #2}\ }}}
\newcommand{\urcornerlinkinclass}[2]{\cornertext{\linkinclass{#1}{#2}}}
\newcommand{\dblurcornerlinkinclass}[4]{\cornertexttwo{\linkinclass{#1}{#2}}{\linkinclass{#3}{#4}}}
\newcommand{\urcornerlinkdemo}[2]{\cornertext{\linkdemo{#1}{#2}}}
\newcommand{\dblurcornerlinkdemo}[4]{\cornertexttwo{\linkdemo{#1}{#2}}{\linkdemo{#3}{#4}}}
\newcommand{\urcornerlinkdemoinclass}[4]{\cornertexttwo{\linkdemo{#1}{#2}}{\linkinclass{#3}{#4}}}



% CS 554 specific:

\newcommand{\tsync}{\alpha}
\newcommand{\tword}{\beta}
\newcommand{\tflop}{\gamma}

\newcommand{\bw}{w}

\newcommand{\tpl}[2]{{\bm{#1}}}
\newcommand{\vtpl}[3]{{\bm{#1}}_#3}
\newcommand{\perm}[2]{\lt[#1\rt]_{#2}}
\newcommand{\costyle}{\ttfamily\bfseries}
\newcommand{\cotext}[1]{{\costyle{#1}}}
\newcommand{\kwstyle}{\costyle\textcolor{mypurple}}
\newcommand{\kwtext}[1]{{\kwstyle{#1}}}
\newcommand{\emstyle}{\costyle\textcolor{myblue}}
\newcommand{\emtext}[1]{{\emstyle{#1}}}
\newcommand{\const}{}
\newcommand{\work}{Q}
\newcommand{\depth}{D}
\newcommand{\flops}{F}
\newcommand{\words}{W}
\newcommand{\syncs}{S}
\newcommand{\mem}{M}
\newcommand{\eff}{E}
\newcommand{\isofun}[1]{\tilde{\work}(p)}
\newcommand{\isomem}[1]{\tilde{\mem}(p)}
\newcommand{\pstrong}{p_s}
\newcommand{\pweak}{p_w}
\newcommand{\ALL}{\star }
\newcommand{\rmn}[2]{\mathbb{R}^{#1\times #2}}
\newcommand{\rn}[1]{\mathbb{R}^{#1}}
\newcommand{\err}{\varepsilon}
\newcommand{\mat}[1]{\begin{bmatrix} #1 \end{bmatrix}}
\newcommand{\mc}[1]{\mathcal{#1}}
\newcommand{\h}[2]{\mc{H}_{#1}(#2)}
\newcommand{\T}{T}%{\mathsf{T}}
\newcommand{\dn}[2]{\mc{M}_{#1}^{\uparrow}(#2)}
%\usepackage[colorlinks=false,urlbordercolor={1.0 1.0 1.0}]{hyperref}
%
%\lstset{ %
%  postbreak=false,
%%  backgroundcolor=\color{white},   % choose the background color; you must add \usepackage{color} or \usepackage{xcolor}
%  basicstyle=\costyle,        % the size of the fonts that are used for the code
%%  breakatwhitespace=false,         % sets if automatic breaks should only happen at whitespace
%%  breaklines=false,                 % sets automatic line breaking
%  captionpos=n,                    % sets the caption-position to bottom
%  commentstyle=\color{mygreen},    % comment style
%  deletekeywords={...},            % if you want to delete keywords from the given language
%  escapeinside={\%*}{*)},          % if you want to add LaTeX within your code
%  extendedchars=true,              % lets you use non-ASCII characters; for 8-bits encodings only, does not work with UTF-8
%  frame=none,                    % adds a frame around the code
%  keepspaces=true,                 % keeps spaces in text, useful for keeping indentation of code (possibly needs columns=flexible)
%  keywordstyle=\color{mypurple},       % keyword style
%  language=C++,                 % the language of the code
%  otherkeywords={*,...},            % if you want to add more keywords to the set
%  numbers=none,                    % where to put the line-numbers; possible values are (none, left, right)
%  numbersep=5pt,                   % how far the line-numbers are from the code
%  numberstyle=\tiny\color{mygray}, % the style that is used for the line-numbers
%  rulecolor=\color{black},         % if not set, the frame-color may be changed on line-breaks within not-black text (e.g. comments (green here))
%  showspaces=false,                % show spaces everywhere adding particular underscores; it overrides 'showstringspaces'
%  showstringspaces=false,          % underline spaces within strings only
%  showtabs=false,                  % show tabs within strings adding particular underscores
%  stepnumber=2,                    % the step between two line-numbers. If it's 1, each line will be numbered
%  stringstyle=\color{mymauve},     % string literal style
%  tabsize=2,                     % sets default tabsize to 2 spaces
%  title=\lstname,                   % show the filename of files included with \lstinputlisting; also try caption instead of title
%  keywordstyle=\emstyle
%}

\lstset{ %
  postbreak=false,
%  backgroundcolor=\color{white},   % choose the background color; you must add \usepackage{color} or \usepackage{xcolor}
  basicstyle=\footnotesize\costyle,        % the size of the fonts that are used for the code
%  breakatwhitespace=false,         % sets if automatic breaks should only happen at whitespace
%  breaklines=false,                 % sets automatic line breaking
  captionpos=n,                    % sets the caption-position to bottom
  commentstyle=\color{mygreen},    % comment style
  deletekeywords={privatei,shared},            % if you want to delete keywords from the given language
  escapeinside={\%*}{*)},          % if you want to add LaTeX within your code
  extendedchars=true,              % lets you use non-ASCII characters; for 8-bits encodings only, does not work with UTF-8
  frame=none,                    % adds a frame around the code
  keepspaces=true,                 % keeps spaces in text, useful for keeping indentation of code (possibly needs columns=flexible)
  keywordstyle=\footnotesize\color{mypurple},       % keyword style
  language=C++,                 % the language of the code
  otherkeywords={*,pragma},            % if you want to add more keywords to the set
  numbers=none,                    % where to put the line-numbers; possible values are (none, left, right)
  numbersep=5pt,                   % how far the line-numbers are from the code
  numberstyle=\footnotesize\color{mygray}, % the style that is used for the line-numbers
  rulecolor=\color{black},         % if not set, the frame-color may be changed on line-breaks within not-black text (e.g. comments (green here))
  showspaces=false,                % show spaces everywhere adding particular underscores; it overrides 'showstringspaces'
  showstringspaces=false,          % underline spaces within strings only
  showtabs=false,                  % show tabs within strings adding particular underscores
  stepnumber=2,                    % the step between two line-numbers. If it's 1, each line will be numbered
  stringstyle=\footnotesize\color{mymauve},     % string literal style
  tabsize=2,                     % sets default tabsize to 2 spaces
  title=\lstname,                   % show the filename of files included with \lstinputlisting; also try caption instead of title
  emph={MPI_Send,MPI_Recv,MPI_Comm,MPI_Comm_rank,MPI_Send,MPI_Comm_split,MPI_Init,MPI_Finalize,MPI_Status,MPI_COMM_WORLD,MPI_Datatype,MPI_Comm_size,MPI_FLOAT,mpi.h,omp,parallel,shared,private,default,reduction},
  emphstyle=\footnotesize\emstyle
}

\title{CS 450: Numerical Anlaysis\footnote{{\it These slides have been drafted by Edgar Solomonik as lecture templates and supplementary material for the book ``Scientific Computing: An Introductory Survey'' by Michael T. Heath (\href{http://heath.cs.illinois.edu/scicomp/notes/index.html}{{\color{blue} slides}}).}}}
\author{}
\institute{University of Illinois at Urbana-Champaign}




\subtitle{Numerical Integration and Differentiation}

\date{}
\begin{document}

\begin{frame}
  \titlepage
\end{frame}

\section{Introduction to Numerical Integration}

\begin{frame}{Integrability and Sensitivity}

\begin{itemize}

\item Seek to compute $\mathcal{I}(f)=\int_{a}^b f(x) dx$:

\mdcond{
\begin{itemize}
\item $f$ is integrable if continuous and bounded.
\mitem Finite number of discontinuities is also often permissible.
\end{itemize}
}

\item The condition number of integration is bounded by the distance $b-a$:

\lgcond{
Suppose the input function is perturbed $\hat{f}=f+ \delta f$, then
\begin{align*}
\delta I &=  |\mathcal{I}(\hat{f}) - \mathcal{I}(f)| \\
               &\leq |\mathcal{I}(\delta f)| \\
               &\leq (b-a) ||\delta f||_\infty, \quad \text{where} \quad |f||_\infty = \max_{x\in[a,b]} | f(x)|.
%\left|\int_{a}^b \hat{f}(x) dx - \int^b_a f(x) dx\right| \\
%               &\leq |\int_{a}^b \underbrace{|\hat{f}(x) dx - \int^b_a f(x)|}_{\leq||\hat{f}-f||_\infty = \delta f}dx \\
%               &\leq (b-a) \delta f
\end{align*}
Note that this result {\it does not depend on the magnitude of $f$ or its derivatives}, which means integration is generally very well-conditioned, which makes sense since integration corresponds to averaging.
}

\end{itemize}

\end{frame}

\section{Quadrature Rules}

\begin{frame}{Quadrature Rules}

\begin{itemize}

\item Approximate the integral $\mathcal{I}(f)$ by a weighted sum of function values:
\lgcond{
\[\mathcal{I}(f) \approx Q_n(f)= \sum_{i=1}^n w_i f(x_i)\]
\begin{itemize}
\item $\{x_i\}_{i=1}^n$ are quadrature \coloremph{nodes} or \coloremph{abscissas},
%\item
 $\{w_i\}_{i=1}^n$ are quadrature \coloremph{weights}.
\mitem Quadrature rule %defined by $(\{x_i\}_{i=1}^n,\{w_i\}_{i=1}^n)$ 
is \coloremph{closed} if $x_1=a,x_n=b$ and \coloremph{open} otherwise.
\mitem Rule is \coloremph{progressive} if nodes of $Q_n$ are a subset of those of $Q_{n+1}$.
\end{itemize}
}

\sitem For a fixed set of $n$ nodes, polynomial interpolation followed by integration give \coloremph{$(n-1)$-degree quadrature rule}:

\lgcond{
\begin{itemize}
\item Accuracy depends on interpolant, is exact for all $(n-1)$-degree polynomials.
\mitem Can obtain weights by expressing the unique $(n-1)$-degree polynomial interpolant in the Lagrange basis 
$p(x)=\sum_{i=1}^{n} \phi_i(x)f(x_i)$, so that
%, the coefficients are then % of $f$ is given by the Vandermonde system,
%\[\B y = \B y, \quad \text{where} \quad y_i= f(x_i) \quad \text{and} \quad d_{ii}=\phi_i(x_i).\] 
%The quadrature rule is defined by 
\[Q_n(f)=\mathcal{I}(p)= \sum_{i=1}^{n} \underbrace{\mathcal{I}(\phi_i)}_{w_i}f(x_i).\]
\end{itemize}
}

\end{itemize}

\end{frame}

\begin{frame}{Determining Weights in a General Basis}

\begin{itemize}
\item A quadrature rule provides $\B x$ and $\B w$ so as to approximate 
\lgcond{
  \[\mathcal{I}(f)\approx Q_n(f)=\langle \B w, \B y \rangle, \quad \text{where} \quad y_i=f(x_i)\]
$Q_n$ is the integral of the polynomial interpolant $p$ of $(x_1,y_1),\ldots,(x_n,y_n)$.
%\begin{itemize}
%\item 
%\end{itemize}
}
\item \coloremph{Method of undetermined coefficients} obtains $\B y$ from \coloremph{moment equations} based on Vandermonde system:
\lgcond{
\[ 
\mathcal{I}(p) = \mathcal{I}(\langle \{\phi_i(x)\}_{i=1}^n, \underbrace{\B V(\B x, \{\phi_i\}_{i=1}^n)^{-1} \B y}_\text{interpolant coefficients}\rangle) =
 \langle \underbrace{\B V(\B x, \{\phi_i\}_{i=1}^n)^{-T}\{\mathcal{I}(\phi_i(x))\}_{i=1}^n}_{\B w}, \B y\rangle\]
%\begin{bmatrix} \int_a^b \phi_1(x) dx & \cdots &\int_a^b \phi_n(x)dx \end{bmatrix} \B V(\B x, \{\phi_i\}_{i=1}^n)^{-1} \B y.\]
%\begin{bmatrix} \int_a^b \phi_1(x) dx & \cdots &\int_a^b \phi_n(x)dx \end{bmatrix} \B V(\B x, \{\phi_i\}_{i=1}^n)^{-1} \B y.\]
\begin{itemize}
\item Thus to obtain $\B w$, we need to solve the linear system,
\[\B V(\B x, \{\phi_i\}_{i=1}^n)^T\B w = \begin{bmatrix} \int_a^b \phi_1(x) dx & \cdots &\int_a^b \phi_n(x)dx \end{bmatrix}^T,\]
\item Note that the weights $\B w$ are \coloremph{independent} of the function values $\B y$.
\end{itemize}
}
\end{itemize}

\end{frame}

\begin{frame}{Newton-Cotes Quadrature}
\urcornerlinkdemo{08-quadrature-and-differentiation}{Newton-Cotes weight finder}

\begin{itemize}
\item \coloremph{Newton-Cotes} quadrature rules are defined by equispaced nodes on $[a,b]$:

\mdcond{
open: $x_i=a+i(b-a)/(n+1)$, closed: $x_i=a+(i-1)(b-a)/(n-1)$.
}

\item The \coloremph{midpoint rule} is the $n=1$ open Newton-Cotes rule:

\mdcond{
\[M(f) = (b-a)f\left(\frac{a+b}{2}\right)\]
}

\item The \coloremph{trapezoid rule} is the $n=2$ closed Newton-Cotes rule:

\mdcond{
\[T(f) = \frac{(b-a)}{2}(f(a) + f(b))\]
}

\item \coloremph{Simpson's rule} is the $n=3$ closed Newton-Cotes rule:

\mdcond{
\[S(f)=\frac{b-a}{6}\left(f(a) + 4f\left(\frac{a+b}{2}\right) + f(b)\right)\]
}

\end{itemize}

\end{frame}


\subsection{Error and Conditioning of Quadrature Rules}
\begin{frame}[fragile]{Error in Newton-Cotes Quadrature}
\urcornerlinkdemo{08-quadrature-and-differentiation}{Accuracy of Newton-Cotes}

\begin{itemize}
\item Consider the Taylor expansion of $f$ about the midpoint of the integration interval $m=(a+b)/2$:
\mdcond{
\[f(x) = f(m) + f'(m)(x-m) + \frac{f''(m)}{2}(x-m)^2 +\ldots\]
}

Integrating the Taylor approximation of $f$, we note that the odd terms drop: 
\lgcond{
\[\mathcal{I}(f) = \underbrace{f(m)(b-a)}_{M(f)} + \underbrace{\frac{f''(m)}{24}(b-a)^3}_{E(f)} + O((b-a)^5)\]
Consequently, the midpoint rule is third-order accurate (first degree).
}
\end{itemize}

\end{frame}


\begin{frame}{Error Estimation}

\begin{itemize}

%\item Quadrature weights can be alternatively determined for a rule by solving the moment equations:
%\lgcond{
%\begin{align*}
%\B V(\B x, \{\phi_i\}_{i=1}^n) \B w = \B y(\{\phi_i\}_{i=1}^n), \quad \text{where} \quad y_i = \mathcal{I}(\phi_i)
%\end{align*}
%}
\item The trapezoid rule is also first degree, despite using higher-degree polynomial interpolant approximation, since 
\lgcond{
\begin{align*}
f(m) = \frac{1}{2}\bigg(&f(a) - f'(m)(a-m) - \frac{f''(m)}{2}(a-m)^2 +\ldots \\
+&f(b) - f'(m)(b-m) - \frac{f''(m)}{2}(b-m)^2 +\ldots \bigg) \\
\mathcal{I}(f)= T(f)  -& \underbrace{\frac{f''(m)}{12}(b-a)^3}_{2E(f)} - O((b-a)^5)
\end{align*}
}
\item The above derivation allows us to obtain an error approximation via a difference of midpoint and trapezoidal rules:
\lgcond{
\[T(f)-M(f)\approx 3E(f).\]
This approximation rapidly becomes accurate as $b-a$ decreases.
}
\end{itemize}
\end{frame}

\begin{frame}{Error in Polynomial Quadrature Rules}

\begin{itemize}


\item We can bound the error for a an arbitrary polynomial quadrature rule by
\lgcond{
\begin{align*}
|\mathcal{I}(f)-Q_n(f)| &= |\mathcal{I}(f-p)| \\
              &\leq (b-a) || f-p||_\infty \\
              &\leq \frac{b-a}{4n}h^n || f^{(n)}||_\infty 
\end{align*}
where $h=\max_i(x_{i+1}-x_i)$.
}
\lgcond{
}

\end{itemize}

\end{frame}

\begin{frame}{Conditioning of Newton-Cotes Quadrature}

\begin{itemize}
\item We can ascertain stability of quadrature rules, by considering the amplification of a perturbation $\hat{f} = f+ \delta f$:
\lgcond {
\begin{align*}
|Q_n(\hat{f})-Q_n(f)| &= |Q_n(\delta f)| \\
                      &= \sum_{i=1}^n w_i \delta f(x_i) \\
                      &\leq ||\B w||_1 ||\delta f||_\infty.
\end{align*}
Note that we always have $\sum_i w_i=b-a$, since the quadrature rule must be correct for a constant function.
So if $\B w$ is positive $||\B w||_1=b-a$, the quadrature rule is stable, i.e. it matches the conditioning of the problem.
}
\item Newton-Cotes quadrature rules have at least one negative weight for any $n\geq 11$:
\lgcond{
More generally, $||\B w||_1\to \infty$ as $n \to \infty$ for fixed $b-a$. This means that the Newton-Cotes rules can be ill-conditioned.
}
\end{itemize}

\end{frame}

\subsection{Chebyshev Quadrature}

\begin{frame}{Clenshaw-Curtis Quadrature}

\begin{itemize}
\item To obtain a more stable quadrature rule, we need to ensure the integrated interpolant is well-behaved as $n$ increases:

\lgcond{
\begin{itemize}
\mitem Chebyshev quadrature nodes ensure that interpolant polynomial has bounded coefficients so long as $f$ is bounded, since the Vandermonde system defining its coefficients is well-conditioned.
\mitem Formally, it can be shown that $w_i>0$ for the Chebyshev-node ({\color{red} Clenshaw-Curtis}) quadrature.
\mitem The weights for Clenshaw-Curtis quadrature rules can be obtained by solutions to Vandermonde systems on $[-1,1]$ with Chebyshev-spaced nodes, then translating to a desired integration interval.
\end{itemize}
}
\end{itemize}

\end{frame}

%\begin{frame}{Gaussian Quadrature}
%
%\begin{itemize}
%\item So far, we have only considered quadrature rules based on a fixed set of nodes, but we can also choose a set of nodes to improve accuracy:
%
%\lgcond{
%Choice of nodes gives additional $n$ parameters for a total of $2n$ degrees of freedom, permitting representation of polynomials of degree $2n-1$.
%}
%\item The \coloremph{unique} $n$-point \coloremph{Gaussian quadrature rule} is defined by the solution of the nonlinear form of the moment equations in terms of \textit{both} $\B x$ and $\B w$:
%\lgcond{
%Given any complete basis, we seek to solve the nonlinear equations,
%\[\B V(\B x, \{\phi_i\}_{i=1}^{2n+1}) \B w = \B y(\{\phi_i\}_{i=1}^{2n+1}), \quad \text{where} \quad y_i = \mathcal{I}(\phi_i)\]
%For fixed $\B x$, we have an overdetermined system of linear equations for $\B w$.
%}
%%To obtain a more stable quadrature rule, we need to ensure the integrated interpolant is well-behaved as $n$ increases:
%%
%%\lgcond {
%%Chebyshev quadrature nodes ensure that interpolant polynomial has bounded coefficients so long as $f$ is bounded, since the Vandermonde system defining its coefficients is well-conditioned.
%%
%%Formally, it can be shown that $w_i>0$ for Chebyshev-node (Clenshaw-Curtis) quadrature.
%%}
%\end{itemize}
%
%\end{frame}
%
%\begin{frame}{Using Gaussian Quadrature Rules}
%
%\begin{itemize}
%\item Gaussian quadrature rules are hard to determine and usually, but can be enumerated for a fixed interval, e.g. $a=0,b=1$, so it suffices to transform the integral to $[0,1]$
%
%\lgcond{
%We have that 
%\[\mathcal{I}(f) = \int_{a}^b f(x)dx = \int_0^1 g(t) dt \quad \text{where} \quad g(x) = f(t), t=\frac{x+b-a}{b-a}\]
%
%Choice of nodes gives additional $n$ parameters for a total of $2n$ degrees od freedom, permitting representation of polynomials of degree $2n-1$.
%}
%\item The \coloremph{unique} $n$-point \coloremph{Gaussian quadrature rule} is defined by the solution of the nonlinear form of the moment equations in terms of \textit{both} $\B x$ and $\B w$:
%\lgcond{
%Given any complete basis, we seek to solve the nonlinear equations,
%\[\B V(\B x, \{\phi_i\}_{i=1}^{2n+1}) \B w = \B y(\{\phi_i\}_{i=1}^{2n+1}), \quad \text{where} \quad y_i = \mathcal{I}(\phi_i)\]
%For fixed $\B x$, we have an overdetermined system of linear equations for $\B w$.
%}
%%To obtain a more stable quadrature rule, we need to ensure the integrated interpolant is well-behaved as $n$ increases:
%%
%%\lgcond {
%%Chebyshev quadrature nodes ensure that interpolant polynomial has bounded coefficients so long as $f$ is bounded, since the Vandermonde system defining its coefficients is well-conditioned.
%%
%%Formally, it can be shown that $w_i>0$ for Chebyshev-node (Clenshaw-Curtis) quadrature.
%%}
%\end{itemize}
%
%\end{frame}
%
%\begin{frame}{Improvements to Gaussian Quadrature}
%
%\begin{itemize}
%\item Gaussian quadrature rules are are accurate and stable but not progressive (nodes cannot be reused to obtain higher-degree approximation):
%
%\mdcond{}
%
%\item \coloremph{Kronod} quadrature rules construct $(2n+1)$-point quadrature $K_{2n+1}$ that is progressive w.r.t. Gaussian quadrature rule $G_n$
%
%\mdcond{}
%
%\item Gaussian quadrature rules are in general open, but Gauss-Radau and Gauss-Lobatto rules permit including end-points:
%
%\mdcond{
%Gauss-Radau uses one of two end-points as a node, while Gauss-Lobatto quadrature uses both.
%}
%
%\end{itemize}
%
%
%\end{frame}
%
%
%\begin{frame}{Composite and Adaptive Quadrature}
%
%\begin{itemize}
%\item Composite quadrature rules are obtained by integrating a piecewise interpolant of $f$:
%
%\lgcond{}
%
%\item Adaptive quadrature corresponds to using composite quadrature with adaptive refinement:
%
%\lgcond{
%Introduce new nodes where error estimate is large. 
%Error estimate can be obtained by e.g. comparing trapezoid and midpoint rules, but can be completely wrong if function is insufficiently smooth.
%}
%
%\end{itemize}
%
%
%\end{frame}








%\end{document}

\subsection{Gaussian Quadrature}

\begin{frame}{Gaussian Quadrature}
\urcornerlinkdemo{08-quadrature-and-differentiation}{Gaussian quadrature weight finder}

\begin{itemize}
\item So far, we have only considered quadrature rules based on a fixed set of nodes, but we may also be able to choose nodes to maximize accuracy:

\lgcond{
\begin{itemize}
\item Choice of nodes gives additional $n$ parameters for total $2n$ degrees of freedom.
\mitem Permits exact integration of degree-$(2n-1)$ polynomials and corresponding general accuracy.
\end{itemize}
}
\item The \coloremph{unique} $n$-point \coloremph{Gaussian quadrature rule} is defined by the solution of the nonlinear form of the moment equations in terms of \textit{both} $\B x$ and $\B w$:

\lgcond{
Given any complete basis, we seek to solve the nonlinear equations for $\B x,\B w$,
\[\B V(\B x, \{\phi_i\}_{i=1}^{2n+1})^T \B w = \B y, \quad \text{where} \quad y_i = \mathcal{I}(\phi_i).\]
\begin{itemize}
\item These nonlinear equations generally have a unique solution $(\B x^*, \B w^*)$.
\mitem For fixed $\B x$, we have an overdetermined system of linear equations for $\B w$.
\end{itemize}
}
%To obtain a more stable quadrature rule, we need to ensure the integrated interpolant is well-behaved as $n$ increases:
%
%\lgcond {
%Chebyshev quadrature nodes ensure that interpolant polynomial has bounded coefficients so long as $f$ is bounded, since the Vandermonde system defining its coefficients is well-conditioned.
%
%Formally, it can be shown that $w_i>0$ for Chebyshev-node (Clenshaw-Curtis) quadrature.
%}
\end{itemize}

\end{frame}

\begin{frame}{Using Gaussian Quadrature Rules}

\begin{itemize}
\item Gaussian quadrature rules are hard to compute, but can be enumerated for a fixed interval, e.g. $a=0,b=1$, so it suffices to transform the integral to $[0,1]$

\lgcond{
\begin{itemize}
\item We can transform a given integral using variable substitution $t=\frac{x-a}{b-a}$,
\[\mathcal{I}(f) = \int_{a}^b f(x)dx = (b-a)\int_0^1 g(t) dt \quad \text{where} \quad g(t) = f(t(b-a)+a)).\]
\item For quadrature rules defined on $[-1,1]$, we can transform via the substitution $t=2\frac{x-a}{b-a}-1$,
\[\mathcal{I}(f) = \int_{a}^b f(x)dx = \frac{b-a}{2}\int_{-1}^1 g(t) dt \quad \text{where} \quad g(t) = f((t+1)(b-a)/2+a).\]
\end{itemize}
}

\item Gaussian quadrature rules are accurate and stable but not progressive (nodes cannot be reused to obtain higher-degree approximation):

\mdcond{
\begin{itemize}
\item maximal degree is obtained
\sitem weights are always positive (perfect conditioning)
\end{itemize}
}


\end{itemize}

\end{frame}

\begin{frame}{Progressive Gaussian-like Quadrature Rules}

\begin{itemize}
\item \coloremph{Kronod} quadrature rules construct $(2n+1)$-point $(3n+1)$-degree quadrature $K_{2n+1}$ that is progressive with respect to Gaussian quadrature rule $G_n$:

\lgcond{
\begin{itemize}
\item Gaussian quadrature rule $G_{2n+1}$ would use same number of points and have degree $4n+1$.
\mitem Kronod rule points are optimal chosen to reuse all points of $G_n$, so $n+1$ rather than $2n+1$ new evaluations are necessary.
\end{itemize}
}
\item \coloremph{Patterson} quadrature rules use $2n+2$ more points to extend $(2n+1)$-point Kronod rule  to degree $6n+4$, while reusing all $2n+1$ points.

\item Gaussian quadrature rules are in general open, but \coloremph{Gauss-Radau} and \coloremph{Gauss-Lobatto} rules permit including end-points:

\lgcond{
Gauss-Radau uses one of two end-points as a node, while Gauss-Lobatto quadrature uses both.
}

\end{itemize}


\end{frame}


\begin{frame}{Composite and Adaptive Quadrature}

\begin{itemize}
\item \coloremph{Composite quadrature rules} are obtained by integrating a piecewise interpolant of $f$:

\lgcond{
For example, we can derive simple composite Newton-Cotes rules by partitioning the domain into sub-intervals $[x_i,x_{i+1}]$:
\begin{itemize}
\sitem composite midpoint rule 
  \[\mathcal{I}(f) = \sum_{i=1}^{n-1} \int_{x_i}^{x_{i+1}} f(x)dx \approx \sum_{i=1}^{n-1} (x_{i+1}-x_i)f((x_{i+1}+x_i)/2)\]
\item composite trapezoid rule
  \[\mathcal{I}(f) = \sum_{i=1}^{n-1} \int_{x_i}^{x_{i+1}} f(x)dx \approx \sum_{i=1}^{n-1} \frac{(x_{i+1}-x_i)}2 (f(x_{i+1})+f(x_i))\]
\end{itemize}
}

\item Composite quadrature can be done with adaptive refinement:

\lgcond{
Introduce new nodes where error estimate is large. 
Error estimate can be obtained by e.g. comparing trapezoid and midpoint rules, but can be completely wrong if function is insufficiently smooth.
}

\end{itemize}


\end{frame}

\begin{frame}{More Complicated Integration Problems}

\begin{itemize}
\item To handle improper integrals can either transform integral to get rid of infinite limit or use appropriate open quadrature rules.

\mdcond{}

\item Double integrals can simply be computed by successive 1-D integration.

\mdcond{
Composite multidimensional rules are also possible by partitioning the domain into chunks.
}

\item High-dimensional integration is often effectively done by \coloremph{Monte Carlo}:

\lgcond{
\[\int_\Omega f(\B x) d\B x = E[Y], \quad Y = \frac{|\Omega|}{N} \sum_{i=1}^N Y_i, \quad Y_i = f(\B x_i), \quad \B x_i \ \  \text{chosen randomly from $\Omega$}.\]
\begin{itemize}
\item Convergence rate is independent of function (effective polynomial degree approximation) or dimension of integration domain.
\mitem Instead, it depends on number of samples ($N$), with error scaling as $O(1/\sqrt{N})$.
\end{itemize}
}


\end{itemize}


\end{frame}


\begin{frame}{Integral Equations}

\begin{itemize}
\item Rather than evaluating an integral, in solving an \coloremph{integral equation} we seek to compute the integrand. A typical linear integral equation has the form
\[\int_a^b K(s,t)u(t) dt = f(s), \quad \text{where} \quad K \quad \text{and} \quad f \quad \text{are known}.\]
\lgcond{
\begin{itemize}
\item Useful for recovering signal $u$ given response function with kernel $K$ and measurements of $f$.
\mitem Also arise from solve equations arising from Green's function methods for PDEs.
\end{itemize}
}
\item Using a quadrature rule with weights $w_1,\ldots, w_n$ and nodes $t_1,\ldots, t_n$ obtain
\lgcond{
\[\sum_{j=1}^n w_jK(s,t_j)u(t_j) = f(s).\]
Discrete sample of $f$ on $s_1,\ldots, s_n$ yields a linear system of equations,
\[\sum_{j=1}^n w_jK(s_i,t_j)u(t_j) = f(s_i).\]
%The resulting matrix $\B A$ can be ill-conditioned for many kernels $K$.
}
\end{itemize}


\end{frame}


%\begin{frame}{Challenges in Solving Integral Equations}
%
%\begin{itemize}
%\item Integral equations based on response functions tend to be ill-conditioned, which is resolved using
%
%\begin{itemize}
%\item truncated singular value decomposition of $\B A$, where $a_{ij}=w_jK(s_i,t_j)$
%
%\mdcond{}
%
%\item replacing the linear system with a regularized linear least squares problem,
%
%\mdcond{}
%
%\item expressing the solution using a basis
%
%\mdcond{Let $u(t)\approx \sum_{j=1}^n c_j \phi_j(t)$ and derive equations for the coefficients.}
%
%\end{itemize}
%\end{itemize}
%
%
%\end{frame}


\begin{frame}{Numerical Differentiation}
\urcornerlinkdemo{08-quadrature-and-differentiation}{Taking Derivatives with Vandermonde Matrices}

\begin{itemize}
\item Automatic (symbolic) differentiation is a surprisingly viable option:
\lgcond{
\begin{itemize}
\item Any computer program is differentiable, since it is an assembly of basic arithmetic operations.
\item Existing software packages can automatically differentiate whole programs.
\end{itemize}
}
\item Numerical differentiation can be done by interpolation or finite differencing:

\lgcond{
\begin{itemize}
\item Given polynomial interpolant, its derivative is easy to obtain by differentiating the basis in which it is expressed,
\[f'(x)\approx p'(x) = \begin{bmatrix} \phi'_1(x) & \cdots &\phi'_n(x) \end{bmatrix}^T\B V(\B t, \{\phi_i\}_{i=1}^n)^{-1} \B y, \ \text{where} \ y_i=f(t_i).\]
\item Obtaining the values of the derivative at the interpolation nodes, can be done via
\[\underbrace{\B V(\B t, \{\phi_i'\}_{i=1}^n)\B V(\B t, \{\phi_i\}_{i=1}^n)^{-1}}_\text{Differentiation matrix} \B y, \ \text{where} \ y_i=f(t_i).\]
\item Finite-differencing formulas effectively use linear interpolant.
\end{itemize}
}

\end{itemize}

\end{frame}

\begin{frame}{Accuracy of Finite Differences}

\dblurcornerlinkdemo{08-quadrature-and-differentiation}{Finite Differences vs Noise}{08-quadrature-and-differentiation}{Floating point vs Finite Differences}

\begin{itemize}
\item \coloremph{Forward and backward differencing} provide first-order accuracy:

\lgcond{
These can be derived, respectively from forward and backward Taylor expansions of $f$ about $x$,
\begin{align*}
f(x+h) &= f(x) + f'(x)h + f''(x) h^2 /2 + \ldots \\
f(x-h) &= f(x) - f'(x)h + f''(x) h^2 /2 - \ldots 
\end{align*}
For forward differencing, we obtain an approximation from the first equation,
\[f'(x) = \frac{f(x+h)- f(x)}h + f''(x) h /2 + \ldots.\]
}
\item \coloremph{Centered differencing} provides second-order accuracy.
\lgcond{
Subtracting the backward Taylor expansion from the forward, we obtain centered differencing,
\[f'(x) = \frac{f(x+h)- f(x-h)}{2h} + O(h^2).\]
Second order accuracy is due to cancellation of odd terms like $f''(x)h/2$.
}

\end{itemize}

\end{frame}

\begin{frame}{Extrapolation Techniques}

\urcornerlinkdemoinclass{08-quadrature-and-differentiation}{Richardson with Finite Differences}{inclass-richardson}{Richardson Extrapolation}

\begin{itemize}
\item Given a series of approximate solutions produced by an iterative procedure, a more accurate approximation may be obtained by \coloremph{extrapolating} this series.

\lgcond{
For example, as we lower the step size $h$ in a finite-difference formula, we can try to extrapolate the series to $h=0$, if we know that 
\[F(h)=a_0 + a_1h^p + O(h^r) \ \text{as} \ h\to 0 \ \text{and seek to determine $F(0)=a_0$},\]
for example in centered differences $p=2$ and $r=4$.
}
\item In particular, given two guesses, \coloremph{Richardson extrapolation} eliminates the leading order error term.

\lgcond{
Seek to eliminate $a_1h^p$ term in  $F(h)$, $F(h/2)$ to improve approximation of $a_0$,
\begin{align*}
F(h) &= a_0 + a_1h^p + O(h^r), \\
F(h/2) &= a_0 + a_1h^p/2^p + O(h^r), \\
a_0 &= F(h) - \frac{F(h)-F(h/2)}{1-1/2^{p}} + O(h^r).
\end{align*}
}

\end{itemize}

\end{frame}

\begin{frame}{High-Order Extrapolation}

\begin{itemize}
\item Given a series of $k$ approximations, \coloremph{Romberg integration} applies $(k-1)$-levels of Richardson extrapolation.

\lgcond{
Can apply Richardson extrapolation to each of $k-1$ pairs of consecutive nodes, then proceed recursively on the $k-1$ resulting approximations.
}

\item Extrapolation can be used within an iterative procedure at each step:

\lgcond{
For example, Steffensen's method for finding roots of nonlinear equations,
\[x_{n+1} = x_n + \frac{f(x_n)}{1-f(x_n+f(x_n))/f(x_n)},\]
derived from Aitken's delta-squared extrapolation process:
\begin{itemize}
\item achieves quadratic convergence,
\sitem requires no derivative,
\sitem competes with the Secant method (quadratic versus superlinear convergence, but an extra function evaluation necessary).
\end{itemize}
}

\end{itemize}

\end{frame}













%\end{document}
