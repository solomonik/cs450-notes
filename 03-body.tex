%%\documentclass[handout]{beamer}
%\documentclass[aspectratio=169,13pt]{beamer}

%\newcommand{\dee}{\partial}

\newcommand{\bmat}[1]{\begin{bmatrix} \B{#1}_{11} & \B{#1}_{12} \\ \B{#1}_{21} & \B{#1}_{22}\end{bmatrix}}
\newcommand{\blmat}[1]{\begin{bmatrix} \B{#1}_{11} &  \\ \B{#1}_{21} & \B{#1}_{22}\end{bmatrix}}
\newcommand{\brmat}[1]{\begin{bmatrix} \B{#1}_{11} & \B{#1}_{12} \\ & \B{#1}_{22}\end{bmatrix}}
\newcommand{\bomat}[1]{\begin{bmatrix} {#1}_{11} & \B{#1}_{12} \\ \B{#1}_{21} & \B{\uppercase{#1}}_{22}\end{bmatrix}}
\newcommand{\blomat}[1]{\begin{bmatrix} {#1}_{11} & \\ \B{#1}_{21} & \B{\uppercase{#1}}_{22}\end{bmatrix}}
\newcommand{\bromat}[1]{\begin{bmatrix} {#1}_{11} & \B{#1}_{12} \\  & \B{\uppercase{#1}}_{22}\end{bmatrix}}

\DeclareMathOperator{\vcop}{vec}
\newcommand{\vc}[1]{\vcop\left(#1\right)}
\newcommand{\prm}{\mu}

\newcommand{\fl}{fl}
\newcommand{\asmcond}[1]{\smallskip{\it #1}\smallskip}
\newcommand{\amdcond}[1]{\smallskip{\it #1}\smallskip}
\newcommand{\algcond}[1]{\smallskip{\it #1}\smallskip}
\newcommand{\atsmcond}[1]{{\it #1}}
\newcommand{\atmdcond}[1]{{\it #1}}
\newcommand{\atlgcond}[1]{{\it #1}}
\newcommand{\afillnum}[1]{{\underline{ \ #1\ \ }}}

\mode<presentation>
{
  \usetheme{Malmoe}
}

\usepackage[english]{babel}
%
%\usepackage[latin1]{inputenc}
%
%\usepackage{times}
%\usepackage[T1]{fontenc}
% Or whatever. Note that the encoding and the font should match. If T1
% does not look nice, try deleting the line with the fontenc.

\usepackage{inconsolata}
\usepackage{listings}
\usepackage{bm}
\usefonttheme[onlymath]{serif}
\renewcommand\familydefault{\sfdefault}
\usepackage[sfdefault]{ClearSans} %% option 'sfdefault' activates Clear Sans as the default text font
%\usefonttheme{professionalfonts}

\usepackage{latexsym}
\usepackage{amsmath}
\usepackage{amssymb}
\usepackage{amsfonts}
\usepackage{graphics}
\usepackage{multicol}

\definecolor{darkred}{rgb}{0.8,0.2,0.2}
\newcommand{\coloremph}[1]{\textcolor{darkred}{\emph{#1}}}        % colored italics
\newcommand{\reshape}[2]{o_{#1}(#2)}        % colored italics
\newcommand{\ket}[1]{\lvert #1 \rangle}
\newcommand{\bra}[1]{\langle #1 \vert}
\newcommand{\transp}[2]{{#2}^{\langle  {#1} \rangle }}        % colored italics

% Miscellaneous special math symbols

% Big-oh notation - either Latex's calligraphic O or uppercase math italic O
\newcommand{\BIGOH}{\mathcal{O}}                              % big oh
%\newcommand{\BIGOH}{O}                                       % big oh
\newcommand{\BIGTHETA}{\Theta}                                % big theta

\newcommand{\sfrac}[2]{{#1}/{#2}}

% Various versions of the fraction one-half
% solidus 1/2 for superscripts:
\newcommand{\SHALF}{1/2}                                      % one-half power
% small 1/2 set case in displayed equations:
\newcommand{\HALF}{\mbox{\small $\frac{1}{2}$}}               % small one-half
\newcommand{\THRD}{\mbox{\scriptsize $\frac{1}{3}$}}          % small one-third
\newcommand{\TWOTH}{\mbox{\scriptsize $\frac{2}{3}$}}         % small two-thirds

% Machine epsilon - note that placement of subscript may need adjustment.
%\newcommand{\emach}{\epsilon_{\textrm{\scriptsize mach}}} % machine prec.
\newcommand{\emach}{\epsilon}

\newcommand{\lsapprox}{\cong}                             % least squares approx

% Real numbers - Prefer \mathbb{R} but it requires amsfonts.
%                Latex's calligraphic R will do if amsfonts are unavailable.
%                DO NOT use \Re, which gives old German fraktur R.
\newcommand{\Real}{\mathbb{R}}                                 % real numbers
\newcommand{\Cplx}{\mathbb{C}}                                 % complex numbers
\newcommand{\Poly}{\mathbb{P}}                                 % polynomials
\newcommand{\Float}{\mathbb{F}}                                % fl pt system
% Bold math fonts for vectors and matrices
\renewcommand{\Vec}[1]{\ensuremath{\bm{#1}}}                   % vector
\newcommand{\Mat}[1]{\ensuremath{\bm{#1}}}                     % matrix
\newcommand{\Op}[1]{#1}                              % operator
\newcommand{\loc}[1]{x_{#1}}                              % operator

% Loglike functions set in regular type
\newcommand{\diag}{\mathrm{diag}}                             % diagonal matrix
\newcommand{\cond}{\mathrm{cond}}                             % condition number
\newcommand{\sign}{\mathrm{sign}}                             % sign function
\newcommand{\Span}{\text{span}}                             % span of matrix
%\newcommand{\trace}{\mathrm{trace}}                           % trace of matrix
\DeclareMathOperator*{\trace}{trace}
\newcommand{\tentrace}[3]{\trace_{#1,#2}(#3)}
\renewcommand{\Re}{\mathrm{Re}}                               % real part
\renewcommand{\Im}{\mathrm{Im}}                               % imaginary part

% Fractions appearing in matrices - allows setting them case or solidus
%\newcommand{\mf}[2]{{#1 \over #2}}        % matrix fraction - set case
\newcommand{\mf}[2]{#1/#2}               % matrix fraction - set solidus


% Keywords for algorithm statements

\newcommand{\FOR}{\textbf{for}\ }
\newcommand{\TO}{\textbf{to}\ }
\newcommand{\IF}{\textbf{if}\ }
\newcommand{\THEN}{\textbf{then}\ }
\newcommand{\ELSE}{\textbf{else}\ }
\newcommand{\WHILE}{\textbf{while}\ }
\newcommand{\DO}{\textbf{do}\ }
\newcommand{\BEGIN}{\textbf{begin}\ }
\newcommand{\END}{\textbf{end}\ }
\newcommand{\STOP}{\textbf{stop}\ }
\newcommand{\AND}{\textbf{and}\ }
\newcommand{\OR}{\textbf{or}\ }

\newcommand{\comm}{\mathrm{comm}}
\newcommand{\comp}{\mathrm{comp}}
\newcommand{\idle}{\mathrm{idle}}
\newcommand{\msg}{\mathrm{msg}}
\newcommand{\len}{\mathrm{s}}
\newcommand{\route}{\mathrm{route}}
\newcommand{\mitem}{\medskip\item}
\newcommand{\sitem}{\smallskip\item}
\DeclareMathOperator*{\argmin}{argmin}

%Gets rid of headline

\setbeamertemplate{footline}{}

\setbeamertemplate{headline}{}
\beamertemplatenavigationsymbolsempty



\definecolor{mygreen}{rgb}{0,0.2,0}
\definecolor{mygray}{rgb}{0.5,0.5,0.5}
\definecolor{mymauve}{rgb}{0.58,0,0.82}
\definecolor{mypurple}{rgb}{0.38,0,0.32}
\definecolor{myblue}{rgb}{0.2,0,0.5}
\definecolor{darkgreen}{rgb}{0.2,0.6,0.2}
\definecolor{Brown}{rgb}{0.39, 0.09, 0.0}
\definecolor{Violet}{rgb}{0.38, 0.0, 0.63}
\definecolor{myvlgray}{rgb}{0.9,0.9,1.0}
\definecolor{mylgray}{rgb}{0.85,0.85,0.9}
\definecolor{darkgreen}{rgb}{0,0.4,0}
\definecolor{brown}{rgb}{.6,.1,.1}

\newcommand{\bemph}[1]{{\color{blue} \emph{#1}}}

\usepackage{mathtools}
\newcommand{\defeq}{\coloneqq}

\newcommand{\inti}[2]{[{#1},{#2}]}
\newcommand{\BF}{\mathbf}
\newcommand{\CF}{\mathcal}
\newcommand{\B}[1]{\bm{#1}}
\newcommand{\E}[1]{\bm{#1}}
\newcommand{\dis}{\displaystyle}
\newcommand{\lt}{\left}
\newcommand{\rt}{\right}

\newcommand{\cs}{H}

\DeclareMathOperator*{\rank}{rank}
\DeclareMathOperator*{\vecn}{vec}
\DeclareMathOperator*{\vecs}{vech}

\usepackage{eso-pic}
\newcommand{\cornertext}[1]{
  \AddToShipoutPictureFG*{
    \AtPageUpperLeft{\put(0,-10){\makebox[\paperwidth][r]{#1}}}  
   }%
}
\newcommand{\cornertexttwo}[2]{
  \AddToShipoutPictureFG*{
    \AtPageUpperLeft{\put(0,-10){\makebox[\paperwidth][r]{#1}}}  
    \AtPageUpperLeft{\put(0,-20){\makebox[\paperwidth][r]{#2}}}  
   }%
}
\newcommand{\linkdemo}[2]{{\footnotesize\href{https://relate.cs.illinois.edu/course/cs450-f18/f/demos/upload/#1/#2.html}{\color{Violet}{\it\textbf{Demo:} #2}\ }}}
\newcommand{\linkinclass}[2]{{\footnotesize\href{https://relate.cs.illinois.edu/course/cs450-f18/flow/#1/start/}{\color{darkgreen}{\it\textbf{Activity:} #2}\ }}}
\newcommand{\urcornerlinkinclass}[2]{\cornertext{\linkinclass{#1}{#2}}}
\newcommand{\dblurcornerlinkinclass}[4]{\cornertexttwo{\linkinclass{#1}{#2}}{\linkinclass{#3}{#4}}}
\newcommand{\urcornerlinkdemo}[2]{\cornertext{\linkdemo{#1}{#2}}}
\newcommand{\dblurcornerlinkdemo}[4]{\cornertexttwo{\linkdemo{#1}{#2}}{\linkdemo{#3}{#4}}}
\newcommand{\urcornerlinkdemoinclass}[4]{\cornertexttwo{\linkdemo{#1}{#2}}{\linkinclass{#3}{#4}}}



% CS 554 specific:

\newcommand{\tsync}{\alpha}
\newcommand{\tword}{\beta}
\newcommand{\tflop}{\gamma}

\newcommand{\bw}{w}

\newcommand{\tpl}[2]{{\bm{#1}}}
\newcommand{\vtpl}[3]{{\bm{#1}}_#3}
\newcommand{\perm}[2]{\lt[#1\rt]_{#2}}
\newcommand{\costyle}{\ttfamily\bfseries}
\newcommand{\cotext}[1]{{\costyle{#1}}}
\newcommand{\kwstyle}{\costyle\textcolor{mypurple}}
\newcommand{\kwtext}[1]{{\kwstyle{#1}}}
\newcommand{\emstyle}{\costyle\textcolor{myblue}}
\newcommand{\emtext}[1]{{\emstyle{#1}}}
\newcommand{\const}{}
\newcommand{\work}{Q}
\newcommand{\depth}{D}
\newcommand{\flops}{F}
\newcommand{\words}{W}
\newcommand{\syncs}{S}
\newcommand{\mem}{M}
\newcommand{\eff}{E}
\newcommand{\isofun}[1]{\tilde{\work}(p)}
\newcommand{\isomem}[1]{\tilde{\mem}(p)}
\newcommand{\pstrong}{p_s}
\newcommand{\pweak}{p_w}
\newcommand{\ALL}{\star }
\newcommand{\rmn}[2]{\mathbb{R}^{#1\times #2}}
\newcommand{\rn}[1]{\mathbb{R}^{#1}}
\newcommand{\err}{\varepsilon}
\newcommand{\mat}[1]{\begin{bmatrix} #1 \end{bmatrix}}
\newcommand{\mc}[1]{\mathcal{#1}}
\newcommand{\h}[2]{\mc{H}_{#1}(#2)}
\newcommand{\T}{T}%{\mathsf{T}}
\newcommand{\dn}[2]{\mc{M}_{#1}^{\uparrow}(#2)}
%\usepackage[colorlinks=false,urlbordercolor={1.0 1.0 1.0}]{hyperref}
%
%\lstset{ %
%  postbreak=false,
%%  backgroundcolor=\color{white},   % choose the background color; you must add \usepackage{color} or \usepackage{xcolor}
%  basicstyle=\costyle,        % the size of the fonts that are used for the code
%%  breakatwhitespace=false,         % sets if automatic breaks should only happen at whitespace
%%  breaklines=false,                 % sets automatic line breaking
%  captionpos=n,                    % sets the caption-position to bottom
%  commentstyle=\color{mygreen},    % comment style
%  deletekeywords={...},            % if you want to delete keywords from the given language
%  escapeinside={\%*}{*)},          % if you want to add LaTeX within your code
%  extendedchars=true,              % lets you use non-ASCII characters; for 8-bits encodings only, does not work with UTF-8
%  frame=none,                    % adds a frame around the code
%  keepspaces=true,                 % keeps spaces in text, useful for keeping indentation of code (possibly needs columns=flexible)
%  keywordstyle=\color{mypurple},       % keyword style
%  language=C++,                 % the language of the code
%  otherkeywords={*,...},            % if you want to add more keywords to the set
%  numbers=none,                    % where to put the line-numbers; possible values are (none, left, right)
%  numbersep=5pt,                   % how far the line-numbers are from the code
%  numberstyle=\tiny\color{mygray}, % the style that is used for the line-numbers
%  rulecolor=\color{black},         % if not set, the frame-color may be changed on line-breaks within not-black text (e.g. comments (green here))
%  showspaces=false,                % show spaces everywhere adding particular underscores; it overrides 'showstringspaces'
%  showstringspaces=false,          % underline spaces within strings only
%  showtabs=false,                  % show tabs within strings adding particular underscores
%  stepnumber=2,                    % the step between two line-numbers. If it's 1, each line will be numbered
%  stringstyle=\color{mymauve},     % string literal style
%  tabsize=2,                     % sets default tabsize to 2 spaces
%  title=\lstname,                   % show the filename of files included with \lstinputlisting; also try caption instead of title
%  keywordstyle=\emstyle
%}

\lstset{ %
  postbreak=false,
%  backgroundcolor=\color{white},   % choose the background color; you must add \usepackage{color} or \usepackage{xcolor}
  basicstyle=\footnotesize\costyle,        % the size of the fonts that are used for the code
%  breakatwhitespace=false,         % sets if automatic breaks should only happen at whitespace
%  breaklines=false,                 % sets automatic line breaking
  captionpos=n,                    % sets the caption-position to bottom
  commentstyle=\color{mygreen},    % comment style
  deletekeywords={privatei,shared},            % if you want to delete keywords from the given language
  escapeinside={\%*}{*)},          % if you want to add LaTeX within your code
  extendedchars=true,              % lets you use non-ASCII characters; for 8-bits encodings only, does not work with UTF-8
  frame=none,                    % adds a frame around the code
  keepspaces=true,                 % keeps spaces in text, useful for keeping indentation of code (possibly needs columns=flexible)
  keywordstyle=\footnotesize\color{mypurple},       % keyword style
  language=C++,                 % the language of the code
  otherkeywords={*,pragma},            % if you want to add more keywords to the set
  numbers=none,                    % where to put the line-numbers; possible values are (none, left, right)
  numbersep=5pt,                   % how far the line-numbers are from the code
  numberstyle=\footnotesize\color{mygray}, % the style that is used for the line-numbers
  rulecolor=\color{black},         % if not set, the frame-color may be changed on line-breaks within not-black text (e.g. comments (green here))
  showspaces=false,                % show spaces everywhere adding particular underscores; it overrides 'showstringspaces'
  showstringspaces=false,          % underline spaces within strings only
  showtabs=false,                  % show tabs within strings adding particular underscores
  stepnumber=2,                    % the step between two line-numbers. If it's 1, each line will be numbered
  stringstyle=\footnotesize\color{mymauve},     % string literal style
  tabsize=2,                     % sets default tabsize to 2 spaces
  title=\lstname,                   % show the filename of files included with \lstinputlisting; also try caption instead of title
  emph={MPI_Send,MPI_Recv,MPI_Comm,MPI_Comm_rank,MPI_Send,MPI_Comm_split,MPI_Init,MPI_Finalize,MPI_Status,MPI_COMM_WORLD,MPI_Datatype,MPI_Comm_size,MPI_FLOAT,mpi.h,omp,parallel,shared,private,default,reduction},
  emphstyle=\footnotesize\emstyle
}

\title{CS 450: Numerical Anlaysis\footnote{{\it These slides have been drafted by Edgar Solomonik as lecture templates and supplementary material for the book ``Scientific Computing: An Introductory Survey'' by Michael T. Heath (\href{http://heath.cs.illinois.edu/scicomp/notes/index.html}{{\color{blue} slides}}).}}}
\author{}
\institute{University of Illinois at Urbana-Champaign}




\subtitle{Linear Least Squares}
\date{}
\begin{document}

\begin{frame}
  \titlepage
\end{frame}

\section{Existence and Conditioning}

\begin{frame}[fragile]{Linear Least Squares}
\begin{itemize}
\item Find $\B{x}^\star= \argmin_{\B{x}\in\mathbb{R}^{n}}||\B A \B x - \B b||_2$ where $\B A \in\mathbb{R}^{m\times n}$:

\mdcond{
Since $m\geq n$, the minimizer generally does not attain a zero residual $\B A \B x -\B b$.

We can rewrite the optimization problem constraint via
\[\B{x}^\star= \argmin_{\B{x}\in\mathbb{R}^{n}}||\B A \B x - \B b||_2^2=\argmin_{\B{x}\in\mathbb{R}^{n}}\Big[(\B A \B x - \B b)^\T(\B A \B x - \B b)\Big]\]
}

\item Given the SVD $\B{A}=\B U \B \Sigma \B V^\T$ we have $\B{x}^\star = \B V \B \Sigma^{\dagger}\B U^\T \B b$, where $\B \Sigma^{\dagger}$ contains the reciprocal of all nonzeros in $\B \Sigma$:
%$\underbrace{\B \Sigma^{-1}}_{\tiny \begin{bmatrix} \sigma_\text{max}^{-1} & & &\\ & \ddots & &\\ & & \sigma_\text{min}^{-1} & \\ & & & \B 0 \end{bmatrix}} \B U^\T \B b$:

  \lgcond{
\begin{itemize}
\item  The minimizer satisfies $\B U \B \Sigma \B V^\T \B{x}^\star \cong \B b$ and consequently also satisfies
  \[\B \Sigma \B{y}^\star \cong \B{d} \quad \text{where $\B{y}^\star = \B{V}^\T \B{x}^\star$ and $\B{d}=\B U^\T \B b$}.\]
\item The minimizer of the reduced problem is $\B{y}^\star = \B \Sigma^{\dagger}\B{d}$, so $y_i = d_i/\sigma_i$ for $i\in \{1,\ldots, n\}$ and $y_i=0$ for $i\in\{n+1,\ldots, m\}$.
\end{itemize}
%  \begin{align*}
%  0 
%  &=(\B U \B \Sigma \B V^\T \B x^\star - \B b)^\T\B U \B U^\T(\B U \B \Sigma \B V^\T  \B x^\star - \B b )  
%  - \min_{\B{x}\in\mathbb{R}^{n}}||\B A \B x - \B b||_2 \\
%  &=(\B \Sigma \B V^\T \B x^\star - \B U^\T \B b)^\T (\B \Sigma \B V^\T  \B x^\star - \B U^\T \B b )
%  - \min_{\B{x}\in\mathbb{R}^{n}}||\B A \B x - \B b||_2 \\
%  &=(\B V^\T \B x^\star - \B \Sigma^{\dagger} \B U^\T \B b)^\T\B \Sigma^2  (\B V^\T  \B x^\star - \B \Sigma^{\dagger}\B U^\T \B b ) \\
%  &=(\B x^\star - \B V \B \Sigma^{\dagger} \B U^\T \B b)^\T\B V \B \Sigma^2 \B V^\T  ( \B x^\star - \B V\B \Sigma^{\dagger}\B U^\T \B b ) 
%  \end{align*}
%  implies $\B x^\star = \B V\B \Sigma^{\dagger}\B U^\T \B b$, where $\B \Sigma^{\dagger}$ contains the reciprocal of all nonzeros in $\B \Sigma$.
  }
\smcond{}

\end{itemize}
\end{frame}


\begin{frame}{Conditioning of Linear Least Squares}
\urcornerlinkdemo{03-least-squares}{Polynomial fitting via the normal equations}

\begin{itemize}
\item Consider fitting a line to a collection of points, then perturbing the points:

\lgcond{ 
\begin{itemize}
\item If our line closely fits all of the points, a small perturbation to the points will not change the ideal fit line (least squares solution) much.
      Note that, if a least squares solution has a very small residual, any other solution with a residual close to as small, should be close to parallel to this solution.
\item When the points are distributed erratically and do not admit a reasonable linear fit, then the least squares solution has a large residual, and totally different lines may exist with a residual nearly as small.
      For example, if the points are in a ball around the origin, any linear fit has the same residual.
      A tiny perturbation could then perturb the least squares solution to be perpendicular to the original.
\end{itemize}
}

\item LLS is ill-posed for any $\B A$, unless we consider solving for a particular $\B b$

\lgcond{ 
\begin{itemize}
\item If $\B b$ is entirely outside the span of $\B A$ then any perturbation to $\B A$ or $\B b$ can completely defines the new solution.
Similarly, if most of $\B b$ is outside the span of $\B A$, a perturbation can cause the solution to fluctuate wildly.
\item On other hand, if for a particular $\B b$ we can find a solution with (near-)zero residual, a small relative perturbation to $\B b$ or $\B A$ will have an effect similar to that of a linear system perturbation (growth bounded by $\kappa(\B A)=\sigma_\text{max}/\sigma_\text{min}$).
\end{itemize}
}

\end{itemize}
\end{frame}

%\begin{frame}{Stability of Normal Equations}
%
%\begin{itemize}
%\item Normal equations solve $\B A^\T\B A\B{x}= \B A \B b$:
%
%\lgcond{
%Consider perturbation $\B{\delta b}$ to $\B b$, in the worst case $\B A \B b$ is perturbed by $\sigma_\text{max} \B{\delta b}$.
%Further $\kappa(\B A^\T\B A)=\kappa(\B A)^2$ so the relative error can be amplified by the square of the expected condition number, given a small residual in the true solution.
%
%%so the linear system attains a worst case solution perturbation of norm $\sigma_\text{max}^3 ||\B{\delta b}||$.
%%Assuming the residual is small, we can bound the backward error on $\B b$ by $\frac{\sigma_\text{max}^3}{\sigma_\text{min}} ||\B{\delta b}||$
%}
%
%\item We can also obtain a QR factorization by means of Cholesky of $\B A^\T\B A$ (known as Cholesky-QR):
%
%\lgcond{
%\begin{itemize}
%\item The QR factorization $\B A =\B Q \B R$ so long as $\B R$ has a positive diagonal, so we have 
%\[\B A^\T\B A=\B R^T \B Q^T\B Q\B R = \B R^T \B R\]
%and thus can obtain $\B R$ by Cholesky $\B A^\T\B A$, which is symmetric positive definite.
%\item However, if condition number of $\B A^\T\B A$ is high, we can lose positive-definiteness due to roundoff error and Cholesky factorization can break down.
%\item Moreover, if we obtain $\B{\hat{Q}}$ via $\B A \B R^{-1}$, round-off error can result in a loss of orthogonality in the columns of $\B{\hat{Q}}$.
%\item So long as $\kappa(\B A)<\sqrt{1/\epsilon_\text{mach}}$, a second iteration of Cholesky-QR on the computed reduced $\B{\hat{Q}}=\B{Q}\B{\hat{R}}$ obtains an orthogonal $\B Q$ and a correction $\B{\hat{R}}$ to $\B{R}$.
%\end{itemize}
%
%}
%
%\end{itemize}
%\end{frame}


\section{Normal Equations}

\begin{frame}[fragile]{Normal Equations}
\dblurcornerlinkdemo{03-least-squares}{Normal equations vs Pseudoinverse}{03-least-squares}{Issues with the normal equations}

\begin{itemize}
\item \coloremph{Normal equations} are given by solving $\B A^\T\B A\B{x}= \B A^\T \B b$:

\lgcond{
If $\B A^\T\B A\B{x}= \B A^\T \B b$ then
\begin{align*}
 %\B A^\dagger\B A^T\B A\B{x}^\star= \B A^\dagger\B A \B b  \quad \text{note} \ (\B A^{\dagger}= \B U \B \Sigma^\dagger \B V^\T \\
 (\B U \B \Sigma \B V^\T)^\T \B U \B \Sigma \B V^\T\B{x}&= (\B U \B \Sigma \B V^\T)^\T \B b\\
\B\Sigma^\T \B \Sigma \B V^\T\B{x}&= \B \Sigma^\T \B U^\T \B b\\
\B V^\T\B{x}&= (\B\Sigma^\T \B \Sigma)^{-1}\B \Sigma^\T \B U^\T \B b=  \B \Sigma^\dagger \B U^\T \B b \\
\B{x} &=  \B V\B \Sigma^\dagger \B U^\T \B b = \B{x}^\star
\end{align*}
}
\item However, solving the normal equations is a more ill-conditioned problem then the original least squares algorithm

  \lgcond{
  Generally we have $\kappa(\B A^T \B A)= \kappa(\B A)^2$ (the singular values of $\B A^T\B A$ are the squares of those in $\B A$).
  Consequently, solving the least squares problem via the normal equations may be unstable because it involves solving a problem that has worse conditioning than the initial least squares problem.
  }


\end{itemize}
\end{frame}


\begin{frame}[fragile]{Solving the Normal Equations}

\begin{itemize}
\item If $\B A$ is full-rank, then $\B A^\T \B A$ is symmetric positive definite (SPD):

\lgcond{
\begin{itemize}
\item Symmetry is easy to check $(\B A^\T \B A)^\T=\B A^\T \B A$.
\sitem $\B A$ being full-rank implies $\sigma_\text{min} > 0$ and further if $\B A = \B U \B \Sigma \B V^T$ we have
\[\B A^\T\B A = \B V^T \B \Sigma^2 \B V\]
which implies that rows of $\B V$ are the eigenvectors of $\B A^\T \B A$ with eigenvalues $\B \Sigma^2$ since $\B A^T \B A \B V^\T = \B V^T \B \Sigma^2$.
\end{itemize}
}

\item Since $\B A^\T \B A$ is SPD we can use Cholesky factorization, to factorize it and solve linear systems:

\lgcond{
\[\B A^\T \B A = \B L \B L^\T\]
}
\end{itemize}
\end{frame}


\section{QR Decomposition}

\subsection{Existence and Applicability}

\begin{frame}{QR Factorization}


\begin{itemize}
\item If $\B A$ is full-rank there exists an orthogonal matrix $\B Q$ and a unique upper-triangular matrix $\B R$ with a positive diagonal such that \(\B A = \B Q \B R\)

\lgcond{
\begin{itemize}
\item Given $\B A^\T\B A =\B L \B L^\T$, we can take $\B R = \B L^\T$ and obtain $\B Q=\B A \B L^{-T}$, since
\(\underbrace{\B L^{-1} \B A^\T}_{\B Q^T} \underbrace{\B A \B L^{-T}}_{\B Q} = \B I\)
implies that $\B Q$ has orthonormal columns.
%\item  We have $\B A^\T \B A = \B{{R}}^T \B{{R}}$, so the solution to the normal equations (which is also the minimizer $\B{x}^\star$) satisfies $\B{{R}}^T \B{R} \B x^\star = \B{{R}}^\T\B Q^\T \B b$. 
%  Furthermore, it suffices to solve $\B{R} \B x^\star = \B Q^\T \B b$, which can be done by backward substitution after transforming $\B b$.
\end{itemize}
}

\item A reduced QR factorization (unique part of general QR) is defined so that $\B Q\in\mathbb{R}^{m\times n}$ has orthonormal columns and $\B{R}$ is square and upper-triangular

  \lgcond{
  
  A full QR factorization gives $\B Q \in\mathbb{R}^{m\times m}$ and $\B R \in \mathbb{R}^{m\times n}$, but since $\B R$ is upper triangular, the latter $m-n$ columns of $\B Q$ are only constrained so as to keep $\B Q$ orthogonal.
  The \coloremph{reduced QR} factorization is given by taking the first $n$ columns $\B Q$ and  $\B{\hat{Q}}$ the upper-triangular block of $\B R$, $\B{\hat{R}}$ giving $\B A = \B{\hat{Q}}\B{\hat{R}}$.
  }

\item We can solve the normal equations (and consequently the linear least squares problem) via reduced QR as follows
\mdcond{
\[\B A^\T \B A \B x = \B A^\T\B b 
\quad \Rightarrow\quad  \B{\hat{R}}^\T \underbrace{\B{\hat{Q}}^\T \B{\hat{Q}}}_{\B I} \B{\hat{R}} \B x = \B{\hat{R}}^\T\B{\hat{Q}}^\T\B b 
\quad \Rightarrow\quad \B{\hat{R}} \B x = \B{\hat{Q}}^\T\B b\]
}
\end{itemize}
\end{frame}

\subsection{Gram-Schmidt Algorithm}

\begin{frame}{Gram-Schmidt Orthogonalization}

\dblurcornerlinkdemo{03-least-squares}{Gram-Schmidt--The Movie}{03-least-squares}{Gram-Schmidt and Modified Gram-Schmidt}

\begin{itemize}
\item {\bf Classical Gram-Schmidt process for QR:}

\lgcond{
The Gram-Schmidt process orthogonalizes a rectangular matrix, i.e. it finds a set of orthonormal vectors
with the same span as the columns of the given matrix.
If $\B{a}_i$ is the $i$th column of the input matrix, the $i$th orthonormal vector ($i$th column of $\B Q$) is 
\[\B{q}_i = \B{b}_i/\underbrace{||\B{b}_i||_2}_{r_{ii}}, \quad \text{where} \quad \B{b}_i =\B{a}_i - \sum_{j=1}^{i-1}\underbrace{\langle\B{q}_j,\B{a}_i\rangle}_{r_{ji}}\B{q}_j.\]
}

\item {\bf Modified Gram-Schmidt process for QR:}

  \lgcond{
  Better numerical stability is achieved by orthogonalizing each vector with respect to each previous vector in sequence (modifying the vector prior to orthogonalizing to the next vector), so 
$\B{b}_i = \text{MGS}(\B{a}_i,i-1)$, where $\text{MGS}(\B{d},0)=\B{d}$ and 
\[\text{MGS}(\B d,j) = \text{MGS}(\B{d} - \langle\B{q}_j,\B{d}\rangle\B{q}_j,j-1)\]
  }


\end{itemize}
\end{frame}

\subsection{Householder Algorithm}

\begin{frame}{Householder QR Factorization}

\urcornerlinkdemo{03-least-squares}{3x3 Householder demo}

\begin{itemize}
\item {\bf A Householder transformation $\B Q=\B I -2\B u \B u^\T$ is an orthogonal matrix defined to annihilate entries of a given vector $\B z$, so $||\B z||_2\B Q  \B e_1 = \B z$:}

  \lgcond{
  \begin{itemize}
  \item Householder QR achieves unconditional stability, by applying only orthogonal transformations to reduce the matrix to upper-triangular form.
  \item Householder transformations (reflectors) are orthogonal matrices, that reduce a vector to a multiple of the first elementary vector, 
  $\alpha \B e_1 = \B Q\B z.$
  \item Because multiplying a vector by an orthogonal matrix preserves its norm, we must have that $|\alpha| = ||\B z||_2$.
  \item As we will see, this transformation can be achieved by a rank-1 perturbation of identify of the form $\B Q=\B I -2\B u \B u^\T$ where $\B u$ is a normalized vector.
  \item Householder matrices are both symmetric and orthogonal implying that $\B Q = \B Q^\T  = \B Q^{-1}$.
\end{itemize}
}
  \item Imposing this form on $\B{Q}$ leaves exactly two choices for $\B u$ given $\B z$,
  \[\B u =\frac{\B z \pm ||\B z||_2\B e_1}{||\B z \pm ||\B z||_2\B e_1||_2}\]


\end{itemize}
\end{frame}

%\begin{frame}{Computing Householder Transformations}
%
%
%\begin{itemize}
%\item To find a Householder transformation that annihilates a given vector $\B z$, compute $\B u =\frac{\B z \pm ||\B z||_2\B e_1}{||\B z \pm ||\B z||_2\B e_1||_2}$
%
%\lgcond{
%
%}
%\item Householder transformations can be \coloremph{aggregated} in the form $\B I-\B Y \B T \B Y^\T$ where $\B Y$ is lower-trapezoidal and $\B T$ is upper-triangular
%  \lgcond{
%  }
%
%\end{itemize}
%\end{frame}
%
%
\begin{frame}{Applying Householder Transformations}

\urcornerlinkinclass{inclass-householder}{Householder QR}

\begin{itemize}
\item The product $\B x = \B Q \B w$ can be computed using $O(n)$ operations if $\B Q$ is a Householder transformation

\lgcond{
\[\B x = (\B I - 2\B u \B u^\T)\B w = \B w - 2\langle\B u, \B w\rangle \B u \]
}

\item Householder transformations are also called \coloremph{reflectors} because their application reflects a vector along a hyperplane (changes sign of component of $\B w$ that is parallel to $\B u$)

  \lgcond{
\begin{itemize}
\item $\B I - \B u\B u^\T$ would be an elementary projector, since $\langle  \B u, \B w\rangle\B u$ gives component of $\B w$ pointing in the direction of $\B u$ and 
\[\B x = (\B I - \B u\B u^\T)\B w = \B w - \langle \B u, \B w\rangle\B u\]
subtracts it out.
\sitem On the other hand, Householder reflectors give
\[\B y = (\B I - 2\B u\B u^\T)\B w = \B w - 2\langle \B u, \B w\rangle\B u = \B x -\langle \B u, \B w\rangle\B u\]
which reverses the sign of that component, so that $||\B y||_2 = ||\B w||_2$.
\end{itemize}
  }



\end{itemize}
\end{frame}

\subsection{Givens Algorithm}

\begin{frame}{Givens Rotations}

\begin{itemize}
\item Householder reflectors reflect vectors, Givens rotations rotate them

\lgcond{
\begin{itemize}
\sitem Householder matrices reflect vectors across a hyperplane, by negating the sign of the vector component that is perpendicular to the hyperplane (parallel to $\B u$)
\sitem Any vector can be reflected to a multiple of an elementary vector by a single Householder rotation (in fact, there are two rotations, resulting in a different sign of the resulting vector)
\sitem Givens rotations instead rotate vectors by an axis of rotation that is perpendicular to a hyperplane spanned by two elementary vectors
\sitem Consequently, each Givens rotation can be used to zero-out (annihilate) one entry of a vector, by rotating it so that the component of the vector pointing in the direction of the axis corresponding to that entry, points into a different axis
\end{itemize}
}

\item Givens rotations are defined by orthogonal matrices of the form $\begin{bmatrix} c & s \\ -s & c\end{bmatrix}$

\lgcond{
\begin{itemize}
\sitem Given a vector $\begin{bmatrix} a \\ b \end{bmatrix}$ we define $c$ and $s$ so that
\(\begin{bmatrix} c & s \\ -s & c\end{bmatrix}\begin{bmatrix} a \\ b \end{bmatrix} = \begin{bmatrix} \sqrt{a^2+b^2} \\ 0 \end{bmatrix}\)
\sitem Solving for $c$ and $s$, we get
\(c = \frac{a}{\sqrt{a^2 + b^2}}, \quad s = \frac{b}{\sqrt{a^2+b^2}}\)
\end{itemize}
}


\end{itemize}

\end{frame}


\begin{frame}{QR via Givens Rotations}
\urcornerlinkdemo{03-least-squares}{Relative cost of matrix factorizations}

\begin{itemize}
\item We can apply a Givens rotation to a pair of matrix rows, to eliminate the first nonzero entry of the second row

\lgcond{
\[\begin{bmatrix} \B I & & & & \\ & c & & s & \\ & & \B I & & \\ & -s & & c & \\ & & & & \B I\end{bmatrix}\begin{bmatrix} \vdots \\ a \\ \vdots \\ b \\ \vdots \end{bmatrix} =\begin{bmatrix} \vdots \\ \sqrt{a^2+b^2} \\ \vdots \\ 0 \\ \vdots \end{bmatrix} \]

}

\item Thus, $n(n-1)/2$ Givens rotations are needed for QR of a square matrix

\lgcond{
\begin{itemize}
\sitem Each rotation modifies two rows, which has cost $O(n)$
\sitem Overall, Givens rotations cost $2n^3$, while Householder QR has cost $(4/3)n^3$
\sitem Givens rotations provide a convenient way of thinking about QR for sparse matrices, since nonzeros can be successively annihilated, although they introduce the same amount of fill (new nonzeros) as Householder reflectors
\end{itemize}
}

\end{itemize}

\end{frame}

\section{Rank-Deficient Problems}

\begin{frame}{Rank-Deficient Least Squares}
\urcornerlinkinclass{inclass-pinv}{Rank Deficient Least Squares Problems}

\begin{itemize}
\item Suppose we want to solve a linear system or least squares problem with a (nearly) rank deficient matrix $\B A$

\lgcond{
\begin{itemize}
\item A rank-deficient (singular) matrix satisfies $\B A \B x = \B 0$ for some $\B x \neq \B 0$
\sitem Rank-deficient matrices must have at least one zero singular value
\sitem Matrices are said to be deficient in \coloremph{numerical rank} if they have extremely small singular values
\sitem The solution to both linear systems (if it exists) and least squares is not unique, since we can add to it any multiple of $\B x$
\end{itemize}
}

\item Rank-deficient least squares problems seek a minimizer $\B x$ of $||\B A \B x -\B b||_2$ of minimal norm $||\B x||_2$

\lgcond{
\begin{itemize}
\item If $\B A$ is a diagonal matrix (with some zero diagonal entries), the best we can do is $x_i=b_i/a_{ii}$ for all $i$ such that $a_{ii}\neq 0$ and $x_i=0$ otherwise
\item We can solve general rank-deficient systems and least squares problems via $\B x = \B A^{\dagger}\B b$ where the pseudoinverse is
\[\B A^{\dagger} = \B V \B \Sigma^{\dagger} \B U^T \quad \sigma^\dagger_{i} = \begin{cases} 1/ \sigma_i &: \sigma_i > 0 \\ 0 &: \sigma_i = 0 \end{cases}\]
Each rotation modifies two rows, which has cost $O(n)$
\end{itemize}
}

\end{itemize}

\end{frame}


\begin{frame}[fragile]{Truncated SVD}
\urcornerlinkdemo{03-least-squares}{Image compression}

\begin{itemize}
\item After floating-point rounding, rank-deficient matrices typically regain full-rank but have nonzero singular values on the order of $\epsilon_\text{mach}\sigma_\text{max}$

\lgcond{
\begin{itemize}
\item Very small singular values can cause large fluctuations in the solution
\sitem To ignore them, we can use a pseudoinverse based on the \coloremph{truncated SVD} which retains singular values above an appropriate threshold
\sitem Alternatively, we can use Tykhonov regularization, solving least squares problems of the form $\min_{\B x} ||\B A\B x - \B b||_2^2 + \alpha ||\B x||^2$, which are equivalent to the augmented least squares problem
\[\begin{bmatrix} \B A \\ \sqrt{\alpha}\B I \end{bmatrix} \B x \cong \begin{bmatrix} \B b \\ \B 0 \end{bmatrix}\]
\end{itemize}
}
\item By the \coloremph{Eckart-Young-Mirsky theorem}, truncated SVD also provides the best low-rank approximation of a matrix (in 2-norm and Frobenius norm)

\lgcond{
\begin{itemize}
\sitem The SVD provides a way to think of a matrix as a sum of outer-products $\sigma_i \B u_i\B v_i^\T$ that are disjoint by orthogonality and the norm of which is $\sigma_i$
\sitem Keeping the $r$ outer products with largest norm provides the best rank-$r$ approximation
\end{itemize}
}
\end{itemize}

\end{frame}




\begin{frame}[fragile]{QR with Column Pivoting}

\begin{itemize}
\item QR with column pivoting provides a way to approximately solve rank-deficient least squares problems and compute the truncated SVD

\lgcond{
\begin{itemize}
\item We seek a factorization of the form $\B Q \B R = \B A \B P$ where $\B P$ is a permutation matrix that permutes the columns of $\B A$
\sitem For $n \times n$ matrix $\B A$ of rank $r$, the bottom $r\times r$ block of $\B R$ will be $\B 0$
\sitem To solve least squares, we can solve the rank-deficient triangular system $\B R \B y = \B Q^T \B b$ then compute $\B x = \B P \B y$
\end{itemize}
}

\item A pivoted QR factorization can be used to compute a rank-$r$ approximation

\lgcond{
\begin{itemize}
\item To compute QR with column pivoting,
  \begin{enumerate}
  \sitem pivot the column of largest norm to be the leading column,
  \item form and apply a Householder reflector $\B H$ so that $\B H \B A = \begin{bmatrix} \alpha & \B b \\ \B 0 & \B B \end{bmatrix}$,
  \item proceed recursively (go back to step 1) to pivot the next column and factorize $\B B$
  \end{enumerate}
\sitem Computing the SVD of the first $r$ columns of $\B A \B P^T$ generally (but not always) gives the truncated SVD
\sitem Halting after $r$ steps leads to a cost of $O(n^2r)$
\end{itemize}
}

\end{itemize}

\end{frame}



%\end{document}
