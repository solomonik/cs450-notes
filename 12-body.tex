%%\documentclass[handout]{beamer}
%\documentclass[aspectratio=169,13pt]{beamer}

%\newcommand{\dee}{\partial}

\newcommand{\bmat}[1]{\begin{bmatrix} \B{#1}_{11} & \B{#1}_{12} \\ \B{#1}_{21} & \B{#1}_{22}\end{bmatrix}}
\newcommand{\blmat}[1]{\begin{bmatrix} \B{#1}_{11} &  \\ \B{#1}_{21} & \B{#1}_{22}\end{bmatrix}}
\newcommand{\brmat}[1]{\begin{bmatrix} \B{#1}_{11} & \B{#1}_{12} \\ & \B{#1}_{22}\end{bmatrix}}
\newcommand{\bomat}[1]{\begin{bmatrix} {#1}_{11} & \B{#1}_{12} \\ \B{#1}_{21} & \B{\uppercase{#1}}_{22}\end{bmatrix}}
\newcommand{\blomat}[1]{\begin{bmatrix} {#1}_{11} & \\ \B{#1}_{21} & \B{\uppercase{#1}}_{22}\end{bmatrix}}
\newcommand{\bromat}[1]{\begin{bmatrix} {#1}_{11} & \B{#1}_{12} \\  & \B{\uppercase{#1}}_{22}\end{bmatrix}}

\DeclareMathOperator{\vcop}{vec}
\newcommand{\vc}[1]{\vcop\left(#1\right)}
\newcommand{\prm}{\mu}

\newcommand{\fl}{fl}
\newcommand{\asmcond}[1]{\smallskip{\it #1}\smallskip}
\newcommand{\amdcond}[1]{\smallskip{\it #1}\smallskip}
\newcommand{\algcond}[1]{\smallskip{\it #1}\smallskip}
\newcommand{\atsmcond}[1]{{\it #1}}
\newcommand{\atmdcond}[1]{{\it #1}}
\newcommand{\atlgcond}[1]{{\it #1}}
\newcommand{\afillnum}[1]{{\underline{ \ #1\ \ }}}

\mode<presentation>
{
  \usetheme{Malmoe}
}

\usepackage[english]{babel}
%
%\usepackage[latin1]{inputenc}
%
%\usepackage{times}
%\usepackage[T1]{fontenc}
% Or whatever. Note that the encoding and the font should match. If T1
% does not look nice, try deleting the line with the fontenc.

\usepackage{inconsolata}
\usepackage{listings}
\usepackage{bm}
\usefonttheme[onlymath]{serif}
\renewcommand\familydefault{\sfdefault}
\usepackage[sfdefault]{ClearSans} %% option 'sfdefault' activates Clear Sans as the default text font
%\usefonttheme{professionalfonts}

\usepackage{latexsym}
\usepackage{amsmath}
\usepackage{amssymb}
\usepackage{amsfonts}
\usepackage{graphics}
\usepackage{multicol}

\definecolor{darkred}{rgb}{0.8,0.2,0.2}
\newcommand{\coloremph}[1]{\textcolor{darkred}{\emph{#1}}}        % colored italics
\newcommand{\reshape}[2]{o_{#1}(#2)}        % colored italics
\newcommand{\ket}[1]{\lvert #1 \rangle}
\newcommand{\bra}[1]{\langle #1 \vert}
\newcommand{\transp}[2]{{#2}^{\langle  {#1} \rangle }}        % colored italics

% Miscellaneous special math symbols

% Big-oh notation - either Latex's calligraphic O or uppercase math italic O
\newcommand{\BIGOH}{\mathcal{O}}                              % big oh
%\newcommand{\BIGOH}{O}                                       % big oh
\newcommand{\BIGTHETA}{\Theta}                                % big theta

\newcommand{\sfrac}[2]{{#1}/{#2}}

% Various versions of the fraction one-half
% solidus 1/2 for superscripts:
\newcommand{\SHALF}{1/2}                                      % one-half power
% small 1/2 set case in displayed equations:
\newcommand{\HALF}{\mbox{\small $\frac{1}{2}$}}               % small one-half
\newcommand{\THRD}{\mbox{\scriptsize $\frac{1}{3}$}}          % small one-third
\newcommand{\TWOTH}{\mbox{\scriptsize $\frac{2}{3}$}}         % small two-thirds

% Machine epsilon - note that placement of subscript may need adjustment.
%\newcommand{\emach}{\epsilon_{\textrm{\scriptsize mach}}} % machine prec.
\newcommand{\emach}{\epsilon}

\newcommand{\lsapprox}{\cong}                             % least squares approx

% Real numbers - Prefer \mathbb{R} but it requires amsfonts.
%                Latex's calligraphic R will do if amsfonts are unavailable.
%                DO NOT use \Re, which gives old German fraktur R.
\newcommand{\Real}{\mathbb{R}}                                 % real numbers
\newcommand{\Cplx}{\mathbb{C}}                                 % complex numbers
\newcommand{\Poly}{\mathbb{P}}                                 % polynomials
\newcommand{\Float}{\mathbb{F}}                                % fl pt system
% Bold math fonts for vectors and matrices
\renewcommand{\Vec}[1]{\ensuremath{\bm{#1}}}                   % vector
\newcommand{\Mat}[1]{\ensuremath{\bm{#1}}}                     % matrix
\newcommand{\Op}[1]{#1}                              % operator
\newcommand{\loc}[1]{x_{#1}}                              % operator

% Loglike functions set in regular type
\newcommand{\diag}{\mathrm{diag}}                             % diagonal matrix
\newcommand{\cond}{\mathrm{cond}}                             % condition number
\newcommand{\sign}{\mathrm{sign}}                             % sign function
\newcommand{\Span}{\text{span}}                             % span of matrix
%\newcommand{\trace}{\mathrm{trace}}                           % trace of matrix
\DeclareMathOperator*{\trace}{trace}
\newcommand{\tentrace}[3]{\trace_{#1,#2}(#3)}
\renewcommand{\Re}{\mathrm{Re}}                               % real part
\renewcommand{\Im}{\mathrm{Im}}                               % imaginary part

% Fractions appearing in matrices - allows setting them case or solidus
%\newcommand{\mf}[2]{{#1 \over #2}}        % matrix fraction - set case
\newcommand{\mf}[2]{#1/#2}               % matrix fraction - set solidus


% Keywords for algorithm statements

\newcommand{\FOR}{\textbf{for}\ }
\newcommand{\TO}{\textbf{to}\ }
\newcommand{\IF}{\textbf{if}\ }
\newcommand{\THEN}{\textbf{then}\ }
\newcommand{\ELSE}{\textbf{else}\ }
\newcommand{\WHILE}{\textbf{while}\ }
\newcommand{\DO}{\textbf{do}\ }
\newcommand{\BEGIN}{\textbf{begin}\ }
\newcommand{\END}{\textbf{end}\ }
\newcommand{\STOP}{\textbf{stop}\ }
\newcommand{\AND}{\textbf{and}\ }
\newcommand{\OR}{\textbf{or}\ }

\newcommand{\comm}{\mathrm{comm}}
\newcommand{\comp}{\mathrm{comp}}
\newcommand{\idle}{\mathrm{idle}}
\newcommand{\msg}{\mathrm{msg}}
\newcommand{\len}{\mathrm{s}}
\newcommand{\route}{\mathrm{route}}
\newcommand{\mitem}{\medskip\item}
\newcommand{\sitem}{\smallskip\item}
\DeclareMathOperator*{\argmin}{argmin}

%Gets rid of headline

\setbeamertemplate{footline}{}

\setbeamertemplate{headline}{}
\beamertemplatenavigationsymbolsempty



\definecolor{mygreen}{rgb}{0,0.2,0}
\definecolor{mygray}{rgb}{0.5,0.5,0.5}
\definecolor{mymauve}{rgb}{0.58,0,0.82}
\definecolor{mypurple}{rgb}{0.38,0,0.32}
\definecolor{myblue}{rgb}{0.2,0,0.5}
\definecolor{darkgreen}{rgb}{0.2,0.6,0.2}
\definecolor{Brown}{rgb}{0.39, 0.09, 0.0}
\definecolor{Violet}{rgb}{0.38, 0.0, 0.63}
\definecolor{myvlgray}{rgb}{0.9,0.9,1.0}
\definecolor{mylgray}{rgb}{0.85,0.85,0.9}
\definecolor{darkgreen}{rgb}{0,0.4,0}
\definecolor{brown}{rgb}{.6,.1,.1}

\newcommand{\bemph}[1]{{\color{blue} \emph{#1}}}

\usepackage{mathtools}
\newcommand{\defeq}{\coloneqq}

\newcommand{\inti}[2]{[{#1},{#2}]}
\newcommand{\BF}{\mathbf}
\newcommand{\CF}{\mathcal}
\newcommand{\B}[1]{\bm{#1}}
\newcommand{\E}[1]{\bm{#1}}
\newcommand{\dis}{\displaystyle}
\newcommand{\lt}{\left}
\newcommand{\rt}{\right}

\newcommand{\cs}{H}

\DeclareMathOperator*{\rank}{rank}
\DeclareMathOperator*{\vecn}{vec}
\DeclareMathOperator*{\vecs}{vech}

\usepackage{eso-pic}
\newcommand{\cornertext}[1]{
  \AddToShipoutPictureFG*{
    \AtPageUpperLeft{\put(0,-10){\makebox[\paperwidth][r]{#1}}}  
   }%
}
\newcommand{\cornertexttwo}[2]{
  \AddToShipoutPictureFG*{
    \AtPageUpperLeft{\put(0,-10){\makebox[\paperwidth][r]{#1}}}  
    \AtPageUpperLeft{\put(0,-20){\makebox[\paperwidth][r]{#2}}}  
   }%
}
\newcommand{\linkdemo}[2]{{\footnotesize\href{https://relate.cs.illinois.edu/course/cs450-f18/f/demos/upload/#1/#2.html}{\color{Violet}{\it\textbf{Demo:} #2}\ }}}
\newcommand{\linkinclass}[2]{{\footnotesize\href{https://relate.cs.illinois.edu/course/cs450-f18/flow/#1/start/}{\color{darkgreen}{\it\textbf{Activity:} #2}\ }}}
\newcommand{\urcornerlinkinclass}[2]{\cornertext{\linkinclass{#1}{#2}}}
\newcommand{\dblurcornerlinkinclass}[4]{\cornertexttwo{\linkinclass{#1}{#2}}{\linkinclass{#3}{#4}}}
\newcommand{\urcornerlinkdemo}[2]{\cornertext{\linkdemo{#1}{#2}}}
\newcommand{\dblurcornerlinkdemo}[4]{\cornertexttwo{\linkdemo{#1}{#2}}{\linkdemo{#3}{#4}}}
\newcommand{\urcornerlinkdemoinclass}[4]{\cornertexttwo{\linkdemo{#1}{#2}}{\linkinclass{#3}{#4}}}



% CS 554 specific:

\newcommand{\tsync}{\alpha}
\newcommand{\tword}{\beta}
\newcommand{\tflop}{\gamma}

\newcommand{\bw}{w}

\newcommand{\tpl}[2]{{\bm{#1}}}
\newcommand{\vtpl}[3]{{\bm{#1}}_#3}
\newcommand{\perm}[2]{\lt[#1\rt]_{#2}}
\newcommand{\costyle}{\ttfamily\bfseries}
\newcommand{\cotext}[1]{{\costyle{#1}}}
\newcommand{\kwstyle}{\costyle\textcolor{mypurple}}
\newcommand{\kwtext}[1]{{\kwstyle{#1}}}
\newcommand{\emstyle}{\costyle\textcolor{myblue}}
\newcommand{\emtext}[1]{{\emstyle{#1}}}
\newcommand{\const}{}
\newcommand{\work}{Q}
\newcommand{\depth}{D}
\newcommand{\flops}{F}
\newcommand{\words}{W}
\newcommand{\syncs}{S}
\newcommand{\mem}{M}
\newcommand{\eff}{E}
\newcommand{\isofun}[1]{\tilde{\work}(p)}
\newcommand{\isomem}[1]{\tilde{\mem}(p)}
\newcommand{\pstrong}{p_s}
\newcommand{\pweak}{p_w}
\newcommand{\ALL}{\star }
\newcommand{\rmn}[2]{\mathbb{R}^{#1\times #2}}
\newcommand{\rn}[1]{\mathbb{R}^{#1}}
\newcommand{\err}{\varepsilon}
\newcommand{\mat}[1]{\begin{bmatrix} #1 \end{bmatrix}}
\newcommand{\mc}[1]{\mathcal{#1}}
\newcommand{\h}[2]{\mc{H}_{#1}(#2)}
\newcommand{\T}{T}%{\mathsf{T}}
\newcommand{\dn}[2]{\mc{M}_{#1}^{\uparrow}(#2)}
%\usepackage[colorlinks=false,urlbordercolor={1.0 1.0 1.0}]{hyperref}
%
%\lstset{ %
%  postbreak=false,
%%  backgroundcolor=\color{white},   % choose the background color; you must add \usepackage{color} or \usepackage{xcolor}
%  basicstyle=\costyle,        % the size of the fonts that are used for the code
%%  breakatwhitespace=false,         % sets if automatic breaks should only happen at whitespace
%%  breaklines=false,                 % sets automatic line breaking
%  captionpos=n,                    % sets the caption-position to bottom
%  commentstyle=\color{mygreen},    % comment style
%  deletekeywords={...},            % if you want to delete keywords from the given language
%  escapeinside={\%*}{*)},          % if you want to add LaTeX within your code
%  extendedchars=true,              % lets you use non-ASCII characters; for 8-bits encodings only, does not work with UTF-8
%  frame=none,                    % adds a frame around the code
%  keepspaces=true,                 % keeps spaces in text, useful for keeping indentation of code (possibly needs columns=flexible)
%  keywordstyle=\color{mypurple},       % keyword style
%  language=C++,                 % the language of the code
%  otherkeywords={*,...},            % if you want to add more keywords to the set
%  numbers=none,                    % where to put the line-numbers; possible values are (none, left, right)
%  numbersep=5pt,                   % how far the line-numbers are from the code
%  numberstyle=\tiny\color{mygray}, % the style that is used for the line-numbers
%  rulecolor=\color{black},         % if not set, the frame-color may be changed on line-breaks within not-black text (e.g. comments (green here))
%  showspaces=false,                % show spaces everywhere adding particular underscores; it overrides 'showstringspaces'
%  showstringspaces=false,          % underline spaces within strings only
%  showtabs=false,                  % show tabs within strings adding particular underscores
%  stepnumber=2,                    % the step between two line-numbers. If it's 1, each line will be numbered
%  stringstyle=\color{mymauve},     % string literal style
%  tabsize=2,                     % sets default tabsize to 2 spaces
%  title=\lstname,                   % show the filename of files included with \lstinputlisting; also try caption instead of title
%  keywordstyle=\emstyle
%}

\lstset{ %
  postbreak=false,
%  backgroundcolor=\color{white},   % choose the background color; you must add \usepackage{color} or \usepackage{xcolor}
  basicstyle=\footnotesize\costyle,        % the size of the fonts that are used for the code
%  breakatwhitespace=false,         % sets if automatic breaks should only happen at whitespace
%  breaklines=false,                 % sets automatic line breaking
  captionpos=n,                    % sets the caption-position to bottom
  commentstyle=\color{mygreen},    % comment style
  deletekeywords={privatei,shared},            % if you want to delete keywords from the given language
  escapeinside={\%*}{*)},          % if you want to add LaTeX within your code
  extendedchars=true,              % lets you use non-ASCII characters; for 8-bits encodings only, does not work with UTF-8
  frame=none,                    % adds a frame around the code
  keepspaces=true,                 % keeps spaces in text, useful for keeping indentation of code (possibly needs columns=flexible)
  keywordstyle=\footnotesize\color{mypurple},       % keyword style
  language=C++,                 % the language of the code
  otherkeywords={*,pragma},            % if you want to add more keywords to the set
  numbers=none,                    % where to put the line-numbers; possible values are (none, left, right)
  numbersep=5pt,                   % how far the line-numbers are from the code
  numberstyle=\footnotesize\color{mygray}, % the style that is used for the line-numbers
  rulecolor=\color{black},         % if not set, the frame-color may be changed on line-breaks within not-black text (e.g. comments (green here))
  showspaces=false,                % show spaces everywhere adding particular underscores; it overrides 'showstringspaces'
  showstringspaces=false,          % underline spaces within strings only
  showtabs=false,                  % show tabs within strings adding particular underscores
  stepnumber=2,                    % the step between two line-numbers. If it's 1, each line will be numbered
  stringstyle=\footnotesize\color{mymauve},     % string literal style
  tabsize=2,                     % sets default tabsize to 2 spaces
  title=\lstname,                   % show the filename of files included with \lstinputlisting; also try caption instead of title
  emph={MPI_Send,MPI_Recv,MPI_Comm,MPI_Comm_rank,MPI_Send,MPI_Comm_split,MPI_Init,MPI_Finalize,MPI_Status,MPI_COMM_WORLD,MPI_Datatype,MPI_Comm_size,MPI_FLOAT,mpi.h,omp,parallel,shared,private,default,reduction},
  emphstyle=\footnotesize\emstyle
}

\title{CS 450: Numerical Anlaysis\footnote{{\it These slides have been drafted by Edgar Solomonik as lecture templates and supplementary material for the book ``Scientific Computing: An Introductory Survey'' by Michael T. Heath (\href{http://heath.cs.illinois.edu/scicomp/notes/index.html}{{\color{blue} slides}}).}}}
\author{}
\institute{University of Illinois at Urbana-Champaign}




\subtitle{Fast Fourier Transform}

\date{}
\begin{document}

\begin{frame}
  \titlepage
\end{frame}

\section{Model Problem}

\begin{frame}{Sparse Linear Systems and Time-independent PDEs}

\begin{itemize}
\item The Poisson equation serves as a model problem for numerical methods:

\lgcond{
\begin{itemize}
\sitem the 2D Poisson problem and resulting Kronecker product linear system are a common benchmark,
\sitem this system has the form $\B T \otimes \B I + \B I \otimes \B T$ where $\B T$ is tridiagonal.
\end{itemize}
}

\item Dense, sparse direct, iterative, FFT, and Multigrid methods provide increasingly good complexity for the problem: 

\lgcond{
\begin{itemize}
\sitem dense linear system solve costs $O(n^3)$ naively,
\sitem nested dissection with Cholesky has $O(n^{3/2})$ complexity and $O(n\log n)$ memory
\sitem Conjugate-Gradient gives $O(n^{3/2})$ complexity with $O(n)$ memory
\sitem FFT achieves $O(n \log n)$ cost and multigrid achieves $O(n)$.
\end{itemize}
}

\end{itemize}

\end{frame}

\section{Multigrid}

\begin{frame}{Multigrid}

\begin{itemize}
\item Multigrid employs a hierarchy of grids to accelerate iterative methods:

\lgcond{
\begin{itemize}
\item the residual equation $\B A \B{\hat{x}} = \B r$ on each fine grid, is approximately solved on the next coarser grid,
\mitem the equation is \coloremph{restricted} by projection matrix $\B P$, so that $\B P \B A \B P^T \B P \B{\hat{x}} = \B P \B r$
\mitem the interpolation operator (often given by $\B P^T$) is used to obtain an approximate $\B{\hat{x}}$ based on the coarse grid approximate solution,
\mitem at each level we perform some smoothing operations (e.g. Jacobi or Conjugate Gradient) before restriction and after interpolation, 
\mitem at the coarsest level we typically solve directly.
\end{itemize}
}
\mdcond{}

\item The multigrid method works by resolving high-frequency error components on finer-grids and low-frequency error components on coarser grids:

\mdcond{
\begin{itemize}
\item smoothers are usually effective at reducing local error, but slow at resolving global (low-frequency) components of the error,
\mitem on coarser grids, the low frequency error may be resolved more quickly.
\end{itemize}
}

\end{itemize}

\end{frame}


\begin{frame}{Multigrid}

\begin{itemize}
\item Consider the Galerkin approximation with linear finite elements to the Poisson equation $u''=f(t)$ with boundary conditions $u(a)=u(b)=0$:
\[%\forall i \in \{1,n\} , \quad 
\phi_i^{(h)}(t) = \begin{cases} (t-t_{i-1})/h &: t \in [t_{i-1},t_i] \\ (t_{i+1}-t)/h &: t\in [t_{i},t_{i+1}] \\ 0 &: \text{otherwise} \end{cases}\]
where $t_0=t_1=a$ and $t_{n+1}=t_n=b$.
\tlgcond{
The weak form with grid spacing of $h$ is
\begin{align*}
\int_a^b f(t) \phi^{(h)}_i(t) dt &=  - \sum_{j=1}^n x_j\int_a^b {\phi_j^{(h)}}'(t){\phi_i^{(h)}}'(t) dt.
\end{align*}
in multigrid, we define a coarse grid basis of $(n-1)/2$ functions, which are hat functions of twice the width,
\begin{align*}
\phi_i^{(2h)}(t) = \frac 12\phi_{2i-2}^{(h)}(t) + \phi_{2i-1}^{(h)}(t) + \frac 12\phi_{2i}^{(h)}(t) = \begin{cases} (t-t_{i-2})/2h &: t \in [t_{i-2},t_i] \\ (t_{i+2}-t)/2h &: t\in [t_{i},t_{i+2}] \\ 0 &: \text{otherwise} \end{cases}
\end{align*}
}
\lgcond{}
\end{itemize}
\end{frame}

\begin{frame}{Coarse Grid Matrix}

\begin{itemize}
\item Multigrid restricts the residual equation on the fine grid $\B A^{(h)}\B x = \B r^{(h)}$ to the coarse grid:
\tlgcond{
Let $\B{\phi}^{(2h)}=\begin{bmatrix} \phi_1^{(2h)} & \cdots &  \phi_{(n-1)/2}^{(2h)} \end{bmatrix}$ and
$\B{\phi}^{(h)}=\begin{bmatrix} \phi_1^{(h)} & \cdots &  \phi_{n}^{(h)} \end{bmatrix}$ and define \coloremph{restriction matrix} $\B P$ so that $\B{\phi}^{(2h)}=\B P \B{\phi}^{(h)}$, i.e., 
\[\B P = \frac 12\begin{bmatrix} 1 & 2 & 1 & & \\ & 1 & 2 & 1 & \\ & & \ddots & \ddots & \ddots \end{bmatrix} = \begin{bmatrix} {\B{p}^{(1)}} \\ {\B{p}^{(2)}} \\ \vdots \end{bmatrix}.\]
The coarse grid stiffness matrix is given by
\begin{align*}
a^{(2h)}_{ij} &= - \int_a^b {\phi_j^{(2h)}}'(t){\phi_i^{(2h)}}'(t) dt \\
&= - {\B{p}^{(i)}}\underbrace{\bigg(\int_a^b  {\B \phi^{(h)}}'(t) {{\B{\phi}^{(h)}}'}{}^T(t) dt\bigg)}_{-\B A^{(h)}}{\B p^{(j)}}^T, \\
\B A^{(2h)} &= \B P \B A^{(h)} \B P^T.
% &=  - \int_a^b ({\phi_{2i-2}^{(h)}}'(t) + 2{\phi_{2i-1}^{(h)}}'(t) + {\phi_{2i}^{(h)}(t)}'(t))
% ({\phi_{2j-2}^{(h)}}'(t) + 2{\phi_{2j-1}^{(h)}}'(t) + {\phi_{2j}^{(h)}(t)}'(t))dt
\end{align*}

%\begin{align*}
%\int_a^b f(t) \phi^{(h)}_i(t) dt &=  - \sum_i x_i\int_a^b {\phi_j^{(h)}}'(t){\phi_i^{(h)}}'(t) dt.
%\end{align*}
%
%To do so, we need to obtain a suitable basis transformation from $\{\phi^{(h)}_i\}$ to $\{\phi^{(2h)}_i\}$.
%We would like a correct answer at each $t$ and can obtain a system of equations by integrating w.r.t. test functions $\phi^{(h)}_j(h)$,
%\begin{align*}
%\sum_{i=1}^{n/2} x^{(2h)}_i \int_a^b\phi_i^{(2h)}(t)  \phi_j^{(h)}(t)dt = \sum_{i=1}^n x^{(h)}_i\int_a^b \phi_i^{(h)}(t) \phi_j^{(h)}(t)dt
%\end{align*}
}
\lgcond{}
\end{itemize}
\end{frame}

\begin{frame}{Restricting the Residual Equation}

\begin{itemize}
\item Given the fine-grid residual $\B r^{(h)}$, we seek to use the coarse grid to approximate $\B x^{(h)}$ so that $\B A \B x^{(h)} \approx \B r^{(h)}$

\lgcond{
\begin{itemize}
\item
%Let $\B t = \begin{bmatrix} a+h & a+2h & \cdots & b-h \end{bmatrix}^T$.
Given a function in the coarse grid basis, $u^{(2h)} = {\B x^{(2h)}}^T\B{\phi}^{(2h)}$, we can express it in the fine-grid basis via
\begin{align*}
u^{(2h)} &= {\B x^{(2h)}}^T \underbrace{\B P \B{\phi}^{(h)}}_{\B{\phi}^{(2h)}} = \underbrace{{\B x^{(2h)}}^T \B P}_{{\B x^{(h)}}^T} \B{\phi}^{(h)}.
\end{align*}
\item
Consequently, the solution to the restricted residual equation $\B A^{(2h)}\B x^{(2h)} = \B r^{(2h)}$ will lead to an approximate residual equation solution on the fine grid 
with $\B x^{(h)} =\B P^T\B x^{(2h)}$.
\mitem
Noting this, we derive the form of the coarse grid residual, %form the coarse grid problem via
\begin{align*}
\B r^{(2h)} &= \B A^{(2h)}\B x^{(2h)} \\
&= \B P\B A^{(h)}\B P^T \B x^{(2h)}  = \B P\B A^{(h)}\B x^{(h)}  \\
&= \B P\B r^{(h)}. 
\end{align*}
\end{itemize}
%\[w_i=u^{(2h)}(t_i)={\B x^{(2h)}}^T\B{\phi}^{(2h)}(t_i)={\B x^{(2h)}}^T\B P \underbrace{\B{\phi}^{(h)}(t_i)}_{\B e_i}\]
% so $\B w = \B P^T{\B x^{(2h)}}$.
%Note that for a function in the fine grid basis, $u^{(h)} = {\B x^{(h)}}^T\B{\phi}^{(h)}$, we have $u^{(2h)}(t_i) = x^{(h)}_i$.
%Consequently, we can express the coarse grid function in the fine grid basis via
%\begin{align*}
%u^{(h)}(t) &= \underbrace{{\B x^{(2h)}}^T \B P}_{\B x^{(h)}^T} \B{\phi}^{(h)}
%\end{align*}
}
\lgcond{}
\end{itemize}

\end{frame}

%\begin{frame}{Multigrid Restriction}
%
%\begin{itemize}
%\item Multigrid restricts the residual equation on the fine grid to the coarse grid:
%\tlgcond{
%To do so, we need to obtain a suitable basis transformation from $\{\phi^{(h)}_i\}$ to $\{\phi^{(2h)}_i\}$,
%\begin{align*}
%\B V(\{\phi_{2i-1}^{(2h)}\}_{i=1}^{(n-1)/2},\{t_i\}_{i=1}^n) \B x^{(2h)} \cong  \B V(\{\phi_i\}_{i=1}^n,\{t_i\}_{i=1}^n)  \B x^{(h)}
%\end{align*}
%For linear finite elements $\B V(\{\phi_i^{(h)}\}_{i=1}^n,\{t_i\}_{i=1}^n)= \B I$ so we obtain,
%\begin{align*}
% \B V(\{\phi_{2i-1}^{(2h)}\}_{i=1}^{(n-1)/2},\{t_i\}_{i=1}^n)\B x^{(2h)} =  \B x^{(h)}.
%\end{align*}
%Multigrid requires values of the residual $\{r(t_{2i-1}\}_{i=1}^{(n-1)/2}$ on the coarse grid, whose coefficients $\B x^{(2h)}$ are then given by
%\begin{align*}
%\B r^{(2h)} = \underbrace{\B V(\{\phi_{i}^{(h)}\}_{i=1}^{n},\{t_{2i-1}\}_{i=1}^{(n-1)/2})}_{\B I}\B x^{(h)} &=  \B V(\{\phi_{2i-1}^{(2h)}\}_{i=1}^{(n-1)/2},\{t_i\}_{i=1}^n)\B x^{(2h)}.
%\end{align*}
%}
%\end{itemize}
%\end{frame}

\section{Fast Fourier Transform}

\begin{frame}{Discrete Fourier Transform}

\begin{itemize}
\item The solutions to hyperbolic PDEs like Poisson are wave-like and take on simple representations in the frequency basis, both for continuous and discretized equations. We define the \coloremph{discrete Fourier transform} using
$$
\omega_{(n)} = \cos(2\pi/n) - i\sin(2\pi/n) = e^{-2\pi i/n}.
$$
\lgcond{
The DFT matrix $\B F\in\mathbb{R}^{n\times n}$ is given by
$f_{ij}=\omega_{(n)}^{ij}$,
$$
\B{F} =
\left[ \begin{matrix}
1 & 1 & 1 & 1 \cr
1 & \omega_{(4)}^{1} & \omega_{(4)}^{2} & \omega_{(4)}^{3} \cr
1 & \omega_{(4)}^{2} & \omega_{(4)}^{4} & \omega_{(4)}^{6} \cr
1 & \omega_{(4)}^{3} & \omega_{(4)}^{6} & \omega_{(4)}^{9} \end{matrix}
\right]$$

\begin{itemize}
\item it is complex and symmetric (not Hermitian),
\item it is unitary modulo scaling $\B F^{*} = n\B F^{-1}$.
\end{itemize}
The discrete Fourier transform of vector $\B v$ is $\B F \B v$.
}
\lgcond{}

\end{itemize}

\end{frame}


\begin{frame}{Fast Fourier Transform (FFT)}

\begin{itemize}
\item Consider $\B b=\B{F}\B a$, we have
\[\forall j\in[0,n-1] \quad b_j=\sum_{k=0}^{n-1}\omega_{(n)}^{jk}a_k,\]
the FFT computes this recursively via 2 FFTs of dimension $n/2$, using $\omega_{(n/2)}=\omega_{(n)}^2$,
\lgcond{
\begin{align*}
%\forall j\in[0,n-1] \quad 
b_j&=\sum_{k=0}^{n/2-1}\omega_{(n)}^{j(2k)}a_{2k}+\sum_{k=0}^{n/2-1}\omega_{(n)}^{j(2k+1)}a_{2k+1} \\
&=\sum_{k=0}^{n/2-1}\omega_{(n/2)}^{jk}a_{2k}+\omega_{(n)}^j\sum_{k=0}^{n/2-1}\omega_{(n/2)}^{jk}a_{2k+1}
\end{align*}
}
\lgcond{}
%, so
%\[\forall j\in[1,n/2] \quad b(2j)=\sum_{k=0}^{n/2-1}\omega_{(n/2)}^{jk}a_k\]
\end{itemize}
\end{frame}


\begin{frame}{Fast Fourier Transform Derivation}

\begin{itemize}
\item The FFT leverages similarity between the first and second half of the output,
\begin{align*}
b_j&=\underbrace{\sum_{k=0}^{n/2-1}\omega_{(n/2)}^{jk}a_{2k}}_{u_j}+\omega_{(n)}^j\underbrace{\sum_{k=0}^{n/2-1}\omega_{(n/2)}^{jk}a_{2k+1}}_{v_j}
\end{align*}
corresponds closely to the entry shifted by $n/2$,
\begin{align*}
 b_{j+n/2}&=\sum_{k=0}^{n/2-1}\omega_{(n/2)}^{(j+n/2)k}a_{2k}+\omega_{(n)}^{j+n/2}\sum_{k=0}^{n/2-1}\omega_{(n/2)}^{(j+n/2)k}a_{2k+1} 
\end{align*}
\lgcond{
Now $\omega_{(n/2)}^{(j+n/2)k}=\omega_{(n/2)}^{jk}$ since $(\omega_{(n/2)}^{n/2})^k=1^k=1$ and using $\omega_{(n)}^{n/2}=-1$,
%$\forall j\in[0,n/2-1]$
\begin{align*}
b_{j+n/2}
&=\underbrace{\sum_{k=0}^{n/2-1}\omega_{(n/2)}^{jk}a_{2k}}_{u_j}-\omega_{(n)}^{j}\underbrace{\sum_{k=0}^{n/2-1}\omega_{(n/2)}^{jk}a_{2k+1}}_{v_j} 
\end{align*}
}
\end{itemize}
\end{frame}

\begin{frame}{FFT Algorithm Summary}

%Each of these two summation can be done recursively with an FFT %of dimension $n/2$
% which can then be combined to compute the upper and lower halves of $b$
\begin{itemize}
\item Let vectors $\B u$ and $\B v$ be two recursive FFTs, $\forall j\in[0,n/2-1]$
\begin{align*}
u_j&=\sum_{k=0}^{n/2-1}\omega_{(n/2)}^{jk}a_{2k}, \quad
v_j=\sum_{k=0}^{n/2-1}\omega_{(n/2)}^{jk}a_{2k+1}
\end{align*}
\lgcond{
\begin{itemize}
%\item We can make these two recursive calls simultaneously and without any work
\item Given $\B u$ and $\B v$ scale using "twiddle factors" $z_j=\omega_{(n)}^j\cdot v_j$
\item Then it suffices to combine the vectors as follows
\(\B b=\begin{bmatrix} \B u+\B z \\ \B u-\B z\end{bmatrix}\)
%\item This recombination is an FFT of dimension 2
%\[\B b=\begin{bmatrix} \B{b}_1 \\ \B{b}_2 \end{bmatrix} = \vc{\begin{bmatrix} \B b_1 & \B b_2\end{bmatrix}} = 
%\vcop\bigg(\begin{bmatrix} \B u & \B z \end{bmatrix} \underbrace{\begin{bmatrix} 1 & 1 \\ 1 & -1 \end{bmatrix}}_{\B{F_4}[0:2,0:2]}\bigg) %=
%%\vc{\begin{bmatrix} \B u & \B z \end{bmatrix} \B{F_2}}
%\]
%\vspace{-.2in}
%\item Radix-$r$ algorithm for any $\B{A}\in\mathbb{R}^{n/r\times r}$ 
%\[
%%\vc{\B{B}}=
%\B{F}_{n}\vc{\B{A}} = \vc{\big([\B{F}_{n}[0:r,0:r] \odot (\B{F}_r \B{A})]\B{F}_{n/r})^T}\]
%%\ \text{ if and only if } \ \B{B} = \B{F}_{n/r} \B{A} \B{F}_r\]
\end{itemize}
}
\item The FFT has $O(n \log n)$ cost complexity:

\lgcond{
There are two recursive calls of dimension $n/2$ and $O(n)$ work for application to twiddle factors and final summation, thus
$$T(n)=2T(n)+O(n)=O(n\log n).$$
} 
\end{itemize}
\end{frame}


\begin{frame}{Applications of the FFT}
\begin{itemize}
\item We can rapidly multiply degree $n$ polynomials by considering their values $\omega_{(2n-1)}^i$ for $i\in\{0,\ldots,2n-1\}$
\lgcond{
\[p_c(\omega_{(2n-1)}^i)=p_a(\omega_{(2n-1)}^i)p_b(\omega_{(2n-1)}^i)\]
\begin{itemize}
\item The product of coefficients of $p_a,p_b$ with Vandermonde matrix $v_{ij}=(\omega_{(2n-1)}^{i})^{j}$, which is the DFT matrix, gives values of polynomials at $2n-1$ nodes.
\mitem Interpolation to compute coefficients of $p_c$ from the products of values of $p_a$ and $p_b$ at those nodes is multiplication by the inverted DFT matrix and is exact since $p_c$ is degree $2n-2$.
\end{itemize}
}

\item More generally the DFT can be used to solve any Toeplitz linear system (convolution):
\lgcond{
\begin{itemize}
\item
A standard convolution has the form,
\(\forall k\in [0,n-1] \quad c_k = \sum_{j=0}^k a_jb_{k-j}.\)
\mitem Convolution is equivalent to multiplications of polynomials with degree $n/2-1$ and coefficients  $\Vec a$ and $\Vec b$, where
%\[p_a(x) = \sum_{k=0}^{n/2-1} a_kx^k,\quad
%p_b(x) = \sum_{k=0}^{n/2-1} a_kx^k\]
the convolution computes the coefficients $\Vec c$ of the product of the two polynomials.
%\[p_c(x) = p_a(x)p_b(x) = \sum_{k=0}^{n-1} c_kx^k\]
\end{itemize}
}
\end{itemize}
\end{frame}




\begin{frame}{Convolution via DFT}

\begin{itemize}
\item The Fourier transform method for computing a convolution is given by
\begin{align*}
c_k &= %\sum_s D^{-1}_{ks} \Big(\sum_j D_{sj} a_j\Big)\Big(\sum_t D_{st} b_t\Big) 
  \frac 1n \sum_s \omega^{-ks}_{(n)} \Big(\sum_j \omega^{sj}_{(n)} a_j\Big)\Big(\sum_t \omega_{(n)}^{st} b_t\Big)
\end{align*}
\lgcond{
\begin{itemize}
\item Rearrange the order of the summations to see what happens to every product of $a$ and $b$
\begin{align*}
c_k &%= \frac 1n \sum_s\sum_j\sum_t \omega^{-ks}_{(n)} \omega_{(n)}^{sj}\omega^{st}_{(n)} a_j b_t
 = \frac 1n \sum_s\sum_j\sum_t \omega^{(j+t-k)s}_{(n)} a_j b_t 
\end{align*}
\item For any $u=j+t-k\neq 0$, we observe $\sum_s(\omega_{(n)}^u)^s=0$ %, as for $\Mat F\Mat F^{-1}$
\mitem When $j+t-k=0$ the products $\omega^{(s+t-j)k}_{(n)}=1$, so there are $n$ nonzero terms $a_jb_{k-j}$ in the summation
\end{itemize}
}
\lgcond{}
\end{itemize}
\end{frame}

\begin{frame}{Solving Numerical PDEs with the FFT}


\urcornerlinkdemo{12-fft}{Fast Fourier Transform}

\begin{itemize}
\item 1D finite-difference schemes on a regular grid correspond to convolutions:

\mdcond{
1D model problem is simply convolution with vector $[1, -2, 1]$.
}

\item For the 1D Poisson model problem, the eigenvectors of $\B T$ corresponds to the imaginary part of a minor of a $2(n+1)$-dimensional DFT matrix:

\mdcond{
\begin{itemize}
\item
In particular, $\B T =\B X \B D \B X^{-1}$ where $x_{ij}$ is the imaginary part of $f_{i+1,j+1}$ with $\B X\in \mathbb{R}^{n\times n}$ and $\B F \in \mathbb{R}^{2(n+1)\times 2(n+1)}$.
\mitem
Consequently, $\B T$ can be diagonalized and the overall system solved by FFT with $O(n\log n)$ cost.
\end{itemize}
}

\item Multidimensional Poisson can be handled with multidimensional FFT:

\mdcond{
For example 2D FFT (1D FFT of each row then 1D FFT of each column) suffices to solve the 2D Poisson problem.
}

\end{itemize}

\end{frame}

%\end{document}

