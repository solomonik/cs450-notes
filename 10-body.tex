%%\documentclass[handout]{beamer}
%\documentclass[aspectratio=169,13pt]{beamer}

%\newcommand{\dee}{\partial}

\newcommand{\bmat}[1]{\begin{bmatrix} \B{#1}_{11} & \B{#1}_{12} \\ \B{#1}_{21} & \B{#1}_{22}\end{bmatrix}}
\newcommand{\blmat}[1]{\begin{bmatrix} \B{#1}_{11} &  \\ \B{#1}_{21} & \B{#1}_{22}\end{bmatrix}}
\newcommand{\brmat}[1]{\begin{bmatrix} \B{#1}_{11} & \B{#1}_{12} \\ & \B{#1}_{22}\end{bmatrix}}
\newcommand{\bomat}[1]{\begin{bmatrix} {#1}_{11} & \B{#1}_{12} \\ \B{#1}_{21} & \B{\uppercase{#1}}_{22}\end{bmatrix}}
\newcommand{\blomat}[1]{\begin{bmatrix} {#1}_{11} & \\ \B{#1}_{21} & \B{\uppercase{#1}}_{22}\end{bmatrix}}
\newcommand{\bromat}[1]{\begin{bmatrix} {#1}_{11} & \B{#1}_{12} \\  & \B{\uppercase{#1}}_{22}\end{bmatrix}}

\DeclareMathOperator{\vcop}{vec}
\newcommand{\vc}[1]{\vcop\left(#1\right)}
\newcommand{\prm}{\mu}

\newcommand{\fl}{fl}
\newcommand{\asmcond}[1]{\smallskip{\it #1}\smallskip}
\newcommand{\amdcond}[1]{\smallskip{\it #1}\smallskip}
\newcommand{\algcond}[1]{\smallskip{\it #1}\smallskip}
\newcommand{\atsmcond}[1]{{\it #1}}
\newcommand{\atmdcond}[1]{{\it #1}}
\newcommand{\atlgcond}[1]{{\it #1}}
\newcommand{\afillnum}[1]{{\underline{ \ #1\ \ }}}

\mode<presentation>
{
  \usetheme{Malmoe}
}

\usepackage[english]{babel}
%
%\usepackage[latin1]{inputenc}
%
%\usepackage{times}
%\usepackage[T1]{fontenc}
% Or whatever. Note that the encoding and the font should match. If T1
% does not look nice, try deleting the line with the fontenc.

\usepackage{inconsolata}
\usepackage{listings}
\usepackage{bm}
\usefonttheme[onlymath]{serif}
\renewcommand\familydefault{\sfdefault}
\usepackage[sfdefault]{ClearSans} %% option 'sfdefault' activates Clear Sans as the default text font
%\usefonttheme{professionalfonts}

\usepackage{latexsym}
\usepackage{amsmath}
\usepackage{amssymb}
\usepackage{amsfonts}
\usepackage{graphics}
\usepackage{multicol}

\definecolor{darkred}{rgb}{0.8,0.2,0.2}
\newcommand{\coloremph}[1]{\textcolor{darkred}{\emph{#1}}}        % colored italics
\newcommand{\reshape}[2]{o_{#1}(#2)}        % colored italics
\newcommand{\ket}[1]{\lvert #1 \rangle}
\newcommand{\bra}[1]{\langle #1 \vert}
\newcommand{\transp}[2]{{#2}^{\langle  {#1} \rangle }}        % colored italics

% Miscellaneous special math symbols

% Big-oh notation - either Latex's calligraphic O or uppercase math italic O
\newcommand{\BIGOH}{\mathcal{O}}                              % big oh
%\newcommand{\BIGOH}{O}                                       % big oh
\newcommand{\BIGTHETA}{\Theta}                                % big theta

\newcommand{\sfrac}[2]{{#1}/{#2}}

% Various versions of the fraction one-half
% solidus 1/2 for superscripts:
\newcommand{\SHALF}{1/2}                                      % one-half power
% small 1/2 set case in displayed equations:
\newcommand{\HALF}{\mbox{\small $\frac{1}{2}$}}               % small one-half
\newcommand{\THRD}{\mbox{\scriptsize $\frac{1}{3}$}}          % small one-third
\newcommand{\TWOTH}{\mbox{\scriptsize $\frac{2}{3}$}}         % small two-thirds

% Machine epsilon - note that placement of subscript may need adjustment.
%\newcommand{\emach}{\epsilon_{\textrm{\scriptsize mach}}} % machine prec.
\newcommand{\emach}{\epsilon}

\newcommand{\lsapprox}{\cong}                             % least squares approx

% Real numbers - Prefer \mathbb{R} but it requires amsfonts.
%                Latex's calligraphic R will do if amsfonts are unavailable.
%                DO NOT use \Re, which gives old German fraktur R.
\newcommand{\Real}{\mathbb{R}}                                 % real numbers
\newcommand{\Cplx}{\mathbb{C}}                                 % complex numbers
\newcommand{\Poly}{\mathbb{P}}                                 % polynomials
\newcommand{\Float}{\mathbb{F}}                                % fl pt system
% Bold math fonts for vectors and matrices
\renewcommand{\Vec}[1]{\ensuremath{\bm{#1}}}                   % vector
\newcommand{\Mat}[1]{\ensuremath{\bm{#1}}}                     % matrix
\newcommand{\Op}[1]{#1}                              % operator
\newcommand{\loc}[1]{x_{#1}}                              % operator

% Loglike functions set in regular type
\newcommand{\diag}{\mathrm{diag}}                             % diagonal matrix
\newcommand{\cond}{\mathrm{cond}}                             % condition number
\newcommand{\sign}{\mathrm{sign}}                             % sign function
\newcommand{\Span}{\text{span}}                             % span of matrix
%\newcommand{\trace}{\mathrm{trace}}                           % trace of matrix
\DeclareMathOperator*{\trace}{trace}
\newcommand{\tentrace}[3]{\trace_{#1,#2}(#3)}
\renewcommand{\Re}{\mathrm{Re}}                               % real part
\renewcommand{\Im}{\mathrm{Im}}                               % imaginary part

% Fractions appearing in matrices - allows setting them case or solidus
%\newcommand{\mf}[2]{{#1 \over #2}}        % matrix fraction - set case
\newcommand{\mf}[2]{#1/#2}               % matrix fraction - set solidus


% Keywords for algorithm statements

\newcommand{\FOR}{\textbf{for}\ }
\newcommand{\TO}{\textbf{to}\ }
\newcommand{\IF}{\textbf{if}\ }
\newcommand{\THEN}{\textbf{then}\ }
\newcommand{\ELSE}{\textbf{else}\ }
\newcommand{\WHILE}{\textbf{while}\ }
\newcommand{\DO}{\textbf{do}\ }
\newcommand{\BEGIN}{\textbf{begin}\ }
\newcommand{\END}{\textbf{end}\ }
\newcommand{\STOP}{\textbf{stop}\ }
\newcommand{\AND}{\textbf{and}\ }
\newcommand{\OR}{\textbf{or}\ }

\newcommand{\comm}{\mathrm{comm}}
\newcommand{\comp}{\mathrm{comp}}
\newcommand{\idle}{\mathrm{idle}}
\newcommand{\msg}{\mathrm{msg}}
\newcommand{\len}{\mathrm{s}}
\newcommand{\route}{\mathrm{route}}
\newcommand{\mitem}{\medskip\item}
\newcommand{\sitem}{\smallskip\item}
\DeclareMathOperator*{\argmin}{argmin}

%Gets rid of headline

\setbeamertemplate{footline}{}

\setbeamertemplate{headline}{}
\beamertemplatenavigationsymbolsempty



\definecolor{mygreen}{rgb}{0,0.2,0}
\definecolor{mygray}{rgb}{0.5,0.5,0.5}
\definecolor{mymauve}{rgb}{0.58,0,0.82}
\definecolor{mypurple}{rgb}{0.38,0,0.32}
\definecolor{myblue}{rgb}{0.2,0,0.5}
\definecolor{darkgreen}{rgb}{0.2,0.6,0.2}
\definecolor{Brown}{rgb}{0.39, 0.09, 0.0}
\definecolor{Violet}{rgb}{0.38, 0.0, 0.63}
\definecolor{myvlgray}{rgb}{0.9,0.9,1.0}
\definecolor{mylgray}{rgb}{0.85,0.85,0.9}
\definecolor{darkgreen}{rgb}{0,0.4,0}
\definecolor{brown}{rgb}{.6,.1,.1}

\newcommand{\bemph}[1]{{\color{blue} \emph{#1}}}

\usepackage{mathtools}
\newcommand{\defeq}{\coloneqq}

\newcommand{\inti}[2]{[{#1},{#2}]}
\newcommand{\BF}{\mathbf}
\newcommand{\CF}{\mathcal}
\newcommand{\B}[1]{\bm{#1}}
\newcommand{\E}[1]{\bm{#1}}
\newcommand{\dis}{\displaystyle}
\newcommand{\lt}{\left}
\newcommand{\rt}{\right}

\newcommand{\cs}{H}

\DeclareMathOperator*{\rank}{rank}
\DeclareMathOperator*{\vecn}{vec}
\DeclareMathOperator*{\vecs}{vech}

\usepackage{eso-pic}
\newcommand{\cornertext}[1]{
  \AddToShipoutPictureFG*{
    \AtPageUpperLeft{\put(0,-10){\makebox[\paperwidth][r]{#1}}}  
   }%
}
\newcommand{\cornertexttwo}[2]{
  \AddToShipoutPictureFG*{
    \AtPageUpperLeft{\put(0,-10){\makebox[\paperwidth][r]{#1}}}  
    \AtPageUpperLeft{\put(0,-20){\makebox[\paperwidth][r]{#2}}}  
   }%
}
\newcommand{\linkdemo}[2]{{\footnotesize\href{https://relate.cs.illinois.edu/course/cs450-f18/f/demos/upload/#1/#2.html}{\color{Violet}{\it\textbf{Demo:} #2}\ }}}
\newcommand{\linkinclass}[2]{{\footnotesize\href{https://relate.cs.illinois.edu/course/cs450-f18/flow/#1/start/}{\color{darkgreen}{\it\textbf{Activity:} #2}\ }}}
\newcommand{\urcornerlinkinclass}[2]{\cornertext{\linkinclass{#1}{#2}}}
\newcommand{\dblurcornerlinkinclass}[4]{\cornertexttwo{\linkinclass{#1}{#2}}{\linkinclass{#3}{#4}}}
\newcommand{\urcornerlinkdemo}[2]{\cornertext{\linkdemo{#1}{#2}}}
\newcommand{\dblurcornerlinkdemo}[4]{\cornertexttwo{\linkdemo{#1}{#2}}{\linkdemo{#3}{#4}}}
\newcommand{\urcornerlinkdemoinclass}[4]{\cornertexttwo{\linkdemo{#1}{#2}}{\linkinclass{#3}{#4}}}



% CS 554 specific:

\newcommand{\tsync}{\alpha}
\newcommand{\tword}{\beta}
\newcommand{\tflop}{\gamma}

\newcommand{\bw}{w}

\newcommand{\tpl}[2]{{\bm{#1}}}
\newcommand{\vtpl}[3]{{\bm{#1}}_#3}
\newcommand{\perm}[2]{\lt[#1\rt]_{#2}}
\newcommand{\costyle}{\ttfamily\bfseries}
\newcommand{\cotext}[1]{{\costyle{#1}}}
\newcommand{\kwstyle}{\costyle\textcolor{mypurple}}
\newcommand{\kwtext}[1]{{\kwstyle{#1}}}
\newcommand{\emstyle}{\costyle\textcolor{myblue}}
\newcommand{\emtext}[1]{{\emstyle{#1}}}
\newcommand{\const}{}
\newcommand{\work}{Q}
\newcommand{\depth}{D}
\newcommand{\flops}{F}
\newcommand{\words}{W}
\newcommand{\syncs}{S}
\newcommand{\mem}{M}
\newcommand{\eff}{E}
\newcommand{\isofun}[1]{\tilde{\work}(p)}
\newcommand{\isomem}[1]{\tilde{\mem}(p)}
\newcommand{\pstrong}{p_s}
\newcommand{\pweak}{p_w}
\newcommand{\ALL}{\star }
\newcommand{\rmn}[2]{\mathbb{R}^{#1\times #2}}
\newcommand{\rn}[1]{\mathbb{R}^{#1}}
\newcommand{\err}{\varepsilon}
\newcommand{\mat}[1]{\begin{bmatrix} #1 \end{bmatrix}}
\newcommand{\mc}[1]{\mathcal{#1}}
\newcommand{\h}[2]{\mc{H}_{#1}(#2)}
\newcommand{\T}{T}%{\mathsf{T}}
\newcommand{\dn}[2]{\mc{M}_{#1}^{\uparrow}(#2)}
%\usepackage[colorlinks=false,urlbordercolor={1.0 1.0 1.0}]{hyperref}
%
%\lstset{ %
%  postbreak=false,
%%  backgroundcolor=\color{white},   % choose the background color; you must add \usepackage{color} or \usepackage{xcolor}
%  basicstyle=\costyle,        % the size of the fonts that are used for the code
%%  breakatwhitespace=false,         % sets if automatic breaks should only happen at whitespace
%%  breaklines=false,                 % sets automatic line breaking
%  captionpos=n,                    % sets the caption-position to bottom
%  commentstyle=\color{mygreen},    % comment style
%  deletekeywords={...},            % if you want to delete keywords from the given language
%  escapeinside={\%*}{*)},          % if you want to add LaTeX within your code
%  extendedchars=true,              % lets you use non-ASCII characters; for 8-bits encodings only, does not work with UTF-8
%  frame=none,                    % adds a frame around the code
%  keepspaces=true,                 % keeps spaces in text, useful for keeping indentation of code (possibly needs columns=flexible)
%  keywordstyle=\color{mypurple},       % keyword style
%  language=C++,                 % the language of the code
%  otherkeywords={*,...},            % if you want to add more keywords to the set
%  numbers=none,                    % where to put the line-numbers; possible values are (none, left, right)
%  numbersep=5pt,                   % how far the line-numbers are from the code
%  numberstyle=\tiny\color{mygray}, % the style that is used for the line-numbers
%  rulecolor=\color{black},         % if not set, the frame-color may be changed on line-breaks within not-black text (e.g. comments (green here))
%  showspaces=false,                % show spaces everywhere adding particular underscores; it overrides 'showstringspaces'
%  showstringspaces=false,          % underline spaces within strings only
%  showtabs=false,                  % show tabs within strings adding particular underscores
%  stepnumber=2,                    % the step between two line-numbers. If it's 1, each line will be numbered
%  stringstyle=\color{mymauve},     % string literal style
%  tabsize=2,                     % sets default tabsize to 2 spaces
%  title=\lstname,                   % show the filename of files included with \lstinputlisting; also try caption instead of title
%  keywordstyle=\emstyle
%}

\lstset{ %
  postbreak=false,
%  backgroundcolor=\color{white},   % choose the background color; you must add \usepackage{color} or \usepackage{xcolor}
  basicstyle=\footnotesize\costyle,        % the size of the fonts that are used for the code
%  breakatwhitespace=false,         % sets if automatic breaks should only happen at whitespace
%  breaklines=false,                 % sets automatic line breaking
  captionpos=n,                    % sets the caption-position to bottom
  commentstyle=\color{mygreen},    % comment style
  deletekeywords={privatei,shared},            % if you want to delete keywords from the given language
  escapeinside={\%*}{*)},          % if you want to add LaTeX within your code
  extendedchars=true,              % lets you use non-ASCII characters; for 8-bits encodings only, does not work with UTF-8
  frame=none,                    % adds a frame around the code
  keepspaces=true,                 % keeps spaces in text, useful for keeping indentation of code (possibly needs columns=flexible)
  keywordstyle=\footnotesize\color{mypurple},       % keyword style
  language=C++,                 % the language of the code
  otherkeywords={*,pragma},            % if you want to add more keywords to the set
  numbers=none,                    % where to put the line-numbers; possible values are (none, left, right)
  numbersep=5pt,                   % how far the line-numbers are from the code
  numberstyle=\footnotesize\color{mygray}, % the style that is used for the line-numbers
  rulecolor=\color{black},         % if not set, the frame-color may be changed on line-breaks within not-black text (e.g. comments (green here))
  showspaces=false,                % show spaces everywhere adding particular underscores; it overrides 'showstringspaces'
  showstringspaces=false,          % underline spaces within strings only
  showtabs=false,                  % show tabs within strings adding particular underscores
  stepnumber=2,                    % the step between two line-numbers. If it's 1, each line will be numbered
  stringstyle=\footnotesize\color{mymauve},     % string literal style
  tabsize=2,                     % sets default tabsize to 2 spaces
  title=\lstname,                   % show the filename of files included with \lstinputlisting; also try caption instead of title
  emph={MPI_Send,MPI_Recv,MPI_Comm,MPI_Comm_rank,MPI_Send,MPI_Comm_split,MPI_Init,MPI_Finalize,MPI_Status,MPI_COMM_WORLD,MPI_Datatype,MPI_Comm_size,MPI_FLOAT,mpi.h,omp,parallel,shared,private,default,reduction},
  emphstyle=\footnotesize\emstyle
}

\title{CS 450: Numerical Anlaysis\footnote{{\it These slides have been drafted by Edgar Solomonik as lecture templates and supplementary material for the book ``Scientific Computing: An Introductory Survey'' by Michael T. Heath (\href{http://heath.cs.illinois.edu/scicomp/notes/index.html}{{\color{blue} slides}}).}}}
\author{}
\institute{University of Illinois at Urbana-Champaign}




\subtitle{Boundary Value Problems for Ordinary Differential Equations}

\date{}
\begin{document}

\begin{frame}
  \titlepage
\end{frame}

\section{Existence, Uniqueness, and Conditioning of ODE BVP solutions}

\subsection{Problem Specification}

\begin{frame}{Boundary Conditions}

\begin{itemize}
\item Often we seek to solve a differential equation that satisfies conditions on its values and derivatives on parts of the domain boundary.
\lgcond{
\begin{itemize}
\item \coloremph{Dirichlet boundary conditions} specify values of $\B y(t)$ at boundary.
\mitem \coloremph{Neumann boundary conditions} specify values of derivative $\B f(t, \B y)$ at boundary.
\end{itemize}
}

\item Consider a first order ODE $\B y'(t)= \B f( t, \B y)$ with \coloremph{linear boundary conditions} on domain $t\in[a,b]$:
  \[\B B_a \B y(a)+ \B B_b \B y(b) = \B c\]

\lgcond{
\begin{itemize}
\item IVPs are a special case of Dirichlet condition with $\B B_a=\B I$, $\B B_b = \B 0$.
\mitem Conditions are \coloremph{separated} if they do not couple different boundary points, i.e., for all $i$, the $i$th row of either $\B B_a$ or $\B B_b$ is zero.
\mitem Higher-order boundary conditions can be reduced to linear boundary conditions in the same way as a nonlinear ODE is reduced to a linear ODE.
\end{itemize}
}
\end{itemize}
\end{frame}

\subsection{Existence of Solutions}

\begin{frame}{Existence of Solutions for Linear ODE BVPs}

\begin{itemize}
\item The solutions of linear ODE BVP $\B y'(t) = \B A(t)\B y(t) + \B b(t)$ are linear combinations of solutions to linear homogeneous ODE IVPs $\B y'(t) = \B A(t) \B y(t)$:

\lgcond{
\begin{itemize}
\item
%Let the $i$th column of $\B Y(t)$ be 
Let the solutions $\B y_i(t)$ to the homogeneous ODE, $\B y_i'(t) = \B A(t)\B y_i(t)$, with initial conditions $\B y_i(a)=\B e_i$ be columns of
\vspace*{-.05in}
\[\B Y(t) = \begin{bmatrix} \B y_1(t) & \cdots & \B y_n(t)\end{bmatrix}  = \B I + \int_{a}^t \B A(s) \B Y(s) ds.\]
\item
The ODE BVP solutions are then given by $\B y(t) = \B Y(t)\B u(t)$ for some $\B u(t)$, with
\begin{align*}
\B y'(t) = \B A(t)\B y(t) + \B b(t) \quad \Rightarrow & \quad
\B Y'(t)\B u(t) + \B Y(t) \B u'(t) = \B A(t)\B Y(t)\B u(t) + \B b(t), \\
\B Y'(t) = \B A(t) \B Y(t) \quad \Rightarrow & \quad
%\item $\B Y(t)$ satisfies $\B Y'(t) = \B A(t)\B Y(t)$, therefore, the derivative of $\B u(t)$ is
\B u'(t) = \B Y(t)^{-1}\B b(t).
\end{align*}
\end{itemize}
}
\item
%\[\B Y_a(t) = \B I + \int_{a}^t \B A(s) \B Y_a(s) ds.\]
%The boundary condition at $a$ implies that the solutions satisfy $\B u(a)=\B Y^{-1}(a)\B y(a)$:
%The boundary conditions entail the existence of a solution $u(t)$:
Solution $\B u(t)$ (and $\B y(t)$) exists if $\B Q = \B B_a  \B Y(a)+ \B B_b \B Y(b)$ is invertible:
\lgcond{
\begin{align*}
%\B B_a \B y(a)+ \B B_b \B y(b) &= \B c \\
\B B_a  \B Y(a)\B u(a)&+ \B B_b \B Y(b)\Big(\B u(a) + \int_{a}^b \B u'(s) ds\Big)= \B c, \\
\B u(a) &= \Big(\underbrace{\B B_a  \B Y(a)+ \B B_b \B Y(b)}_{\B Q}\Big)^{-1}\Big(\B c - \B B_b  \B Y(b)\int_{a}^b \B u'(s) ds\Big).
\end{align*}
}
\end{itemize}
\end{frame}

\begin{frame}{Green's Function Form of Solution for Linear ODE BVPs}

\begin{itemize}
%\item We can ascertain how the linear combinations of homogeneous solutions change ($\B u'(t)$) from the ODE:
%% to specify $\B y(t)=\B Y(t)\B u(t)$:
%%Can derive the solutions to a linear ODE BVP $\B y'(t) = \B A(t)\B y(t) + \B b(t)$ from solutions to homogeneous linear ODE $\B y' = \B A(t) \B y(t)$ IVPs:
%
%\lgcond{
%\begin{itemize}
%\item Substituting into the diffential equation $\B y'(t) = \B A(t)\B y(t) + \B b(t)$ we obtain
%\[\B Y'(t)\B u(t) + \B Y(t) \B u'(t) = \B A(t)\B Y(t)\B u(t) + \B b(t).\]
%\item $\B Y(t)$ satisfies $\B Y'(t) = \B A(t)\B Y(t)$, therefore, we obtain the form,
%\[\B u'(t) = \B Y(t)^{-1}\B b(t).\]
%\end{itemize}
%}
\item For any given $\B b(t)$ and $\B c$, the solution to the BVP can be written in the form:
\[\B y(t) = \B \Phi(t) \B c + \int_a^b \B G(t,s)\B b(s) ds\]
$\B \Phi(t) = \B Y(t) \B Q^{-1}$ is the \coloremph{fundamental matrix} and the \coloremph{Green's function} is
\[\B G(t,s) = \B Y(t) \B Q^{-1}\B I(s) \B Y^{-1}(s), \quad \B I(s) = \begin{cases}  \B B_a\B Y(a) &: s<t \\-\B B_b\B Y(b) & : s\geq t\end{cases}\]
\lgcond{
\begin{itemize}
\item From our expression for $\B u(a)$ and the integral equation for $\B y(t)$,
\begin{align*}
\B y(t) %&= \B Y(t)\Big(\B u(a) + \int_a^t\B u'(s) ds\Big) \\
&=\B Y(t) \B Q^{-1}\bigg(\B c -\B B_b \B Y(b)\int_a^b\B u'(s)ds\bigg) + \B Y(t) \int_a^t \B u'(s) ds \\
&=\B \Phi(t) \B c + \B Y(t) \B Q^{-1}\bigg(-\B B_b \B Y(b)\int_a^b\B u'(s)ds + \B Q \int_a^t \B u'(s) ds\bigg) \\
&=\B \Phi(t) \B c + \B Y(t) \B Q^{-1}\bigg(\B B_a \B Y(a)\int_a^t\B Y^{-1}(s)\B b(s) ds - \B B_b \B Y(b) \int_t^b \B Y^{-1}(s)\B b(s) \bigg).
\end{align*}
\end{itemize}
}
\lgcond{}

\end{itemize}
\end{frame}


%\begin{frame}{Green's Function Form of Solution}
%
%\begin{itemize}
%\item For any given $\B b(t)$ and $\B c$, the solution to the BVP can be written in the form:
%\[\B y(t) = \B \Phi(t) \B c + \int_a^b \B G(t,s)\B b(s) ds\]
%$\B \Phi(t) = \B Y(t) \B Q^{-1}$ is the \coloremph{fundamental matrix} and the Green's function is
%\[\B G(t,s) = \B Y(t) \B Q^{-1}\B I(s) \B Y^{-1}(s), \quad \B I(s) = \begin{cases}  \B B_a\B Y(a)& : s<t \\\B B_b\B Y(b) & : s\geq t\end{cases}\]
%%\item We can ascertain how the linear combinations of homogeneous solutions change ($\B u'(t)$) from the ODE:
%%% to specify $\B y(t)=\B Y(t)\B u(t)$:
%%%Can derive the solutions to a linear ODE BVP $\B y'(t) = \B A(t)\B y(t) + \B b(t)$ from solutions to homogeneous linear ODE $\B y' = \B A(t) \B y(t)$ IVPs:
%%
%%\lgcond{
%%\begin{itemize}
%%\item Substituting into the diffential equation $\B y'(t) = \B A(t)\B y(t) + \B b(t)$ we obtain
%%\[\B Y'(t)\B u(t) + \B Y(t) \B u'(t) = \B A(t)\B Y(t)\B u(t) + \B b(t).\]
%%\item $\B Y(t)$ satisfies $\B Y'(t) = \B A(t)\B Y(t)$, therefore, we obtain the form,
%%\[\B u'(t) = \B Y(t)^{-1}\B b(t).\]
%%\end{itemize}
%%}
%%\item From the integral equation $\B u(t) = \B u(a) + \int_a^t\B Y^{-1}(s)\B b(s) ds$, we obtain $\B y(t)$,
%\lgcond{
%\begin{align*}
%\B y(t) &= \B Y(t)\bigg(\B u(a) + \int_a^t\B Y^{-1}(s)\B b(s) ds\bigg) \\
%&=\B Y(t) \underbrace{\B Q^{-1}\bigg(\B c -\B B_b \B Y(b)\int_a^b\underbrace{\B Y^{-1}(s)\B b(s)}_{\B u'(s)}ds\bigg)}_{\B u(a)} + \B Y(t) \int_a^t \underbrace{\B Y^{-1}(s) \B b(s)}_{\B u'(s)} ds
%\end{align*}
%}
%
%\end{itemize}
%\end{frame}

%\begin{frame}{Linear ODE BVP Green's Function}
%
%\begin{itemize}
%
%\item We now express our solution (with form $\B y(t) = \B Y(t)(\B u(a) + \int_a^t \B u'(s) ds)$) in the form $\B y(t) = \B s(t) + \int_a^b \B G(t,s) \B b(s) ds$ where $\B G$ is the \coloremph{Green's function}:
%
%\lgcond{
%Using the fact that,
%\begin{align*}
%\B Y(t) \B Q^{-1}\underbrace{(\B B_a \B Y(a)+\B B_b \B Y(b))}_{\B Q}\int_a^t\B Y^{-1}(s)\B b(s)ds =  \B Y(t)\int_a^t \B Y^{-1}(s) \B b(s) ds,
%\end{align*}
%we can express our previous solution in the form,
%\begin{align*}
%\B y(t) = \B Y(t) \B Q^{-1}\bigg(\B c + \B B_a \B Y(a)\int_a^t\B Y^{-1}(s)\B b(s)ds-\B B_b \B Y(b)\int_t^b\B Y^{-1}(s)\B b(s)ds\bigg),
%\end{align*}
%which allows us to derive the Green's function,
%\[\B G(t,s) = \B Y(t) \B Q^{-1}\B I(s) \B Y^{-1}(s), \quad \B I(s) = \begin{cases}  \B B_a\B Y(a)& : s<t \\\B B_b\B Y(b) & : s\geq t\end{cases}\]
%}
%\lgcond{}
%\end{itemize}
%\end{frame}



\subsection{Conditioning of Solutions}

\begin{frame}{Conditioning of Linear ODE BVPs}

\begin{itemize}
\item For any given $\B b(t)$ and $\B c$, the solution to the BVP can be written in the form:
\[\B y(t) = \B \Phi(t) \B c + \int_a^b \B G(t,s)\B b(s) ds\]

\lgcond{
$\B \Phi(t) = \B Y(t) \B Q^{-1}$ is the fundamental matrix, which like the Green's function is associated with the homogeneous ODE as well as its linear boundary condition matrices $\B B_a$ and $\B B_b$, but is independent $\B b(t)$ and $\B c$.
}
\item The absolute condition number of the BVP is $\kappa = \max\{||\B \Phi||_\infty,||\B G||_\infty\}$:

\lgcond{
This sensitivity measure enables us to bound the perturbation $||\B{\hat{y}}-\B{y}||_\infty$ with respect to the magnitude of a perturbation to $\B b(t)$ or $\B c$.
}

\end{itemize}
\end{frame}

\section{Algorithms for ODE BVPs}

\subsection{Shooting and Multiple Shooting Methods}

\begin{frame}{Shooting Method for ODE BVPs}

\urcornerlinkdemo{10-boundary-value-problems}{Shooting method}

\begin{itemize}
\item For linear ODEs, we construct solutions from IVP solutions in $\B Y(t)$, which suggests the \coloremph{shooting method} for solving BVPs by reduction to IVPs:

\tlgcond{
\smallskip
%The \coloremph{shooting} method iteratively 
%Guess initial condition $\B{\hat{y}}^{(1)}(a)$ and 
For $k=1,2,\ldots$ repeat until convergence:
\begin{enumerate}
\item 
construct approximate initial value guesses $\B{\hat{y}}^{(k)}(a) \approx \B y(a)$,
\item solve the resulting IVP,
\item check the quality of the solution at the new boundary,
\[||\B B_b\B{\hat{y}}^{(k)}(b) - \B B_a\B{\hat{y}}^{(k)}(a) - \B c||,\]
\item pick the initial conditions for the next shot, $\B{\hat{y}}^{(k+1)}(a)$ by treating $\B{\hat{y}}^{(l)}(a)$ for $l=1,\ldots,k$ as guesses $\B x^{(1)},\ldots, \B x^{(k)}$ to root finding procedure for
%can be constructed by a 1D root finding technique on
\[\B h(\B x) = \B B_a \B x + \B B_b \B y_{\B x}(b) - \B c, \ \text{where}\ \B y_{\B x}(b)  \ \text{is the IVP solution with $\B y_{\B x}(a)=\B x$}.\]
\end{enumerate}
}
\item \coloremph{Multiple shooting} employs the shooting method over subdomains:

\lgcond{
\begin{itemize}
\item
The shooting problems on subdomains are interdependent, as they must satisfy continuity conditions on boundaries between them, leading to a system of nonlinear equations.
\mitem
Improves on conditioning of shooting method, which can suffer from ill-conditioning of large IVPs.
\end{itemize}
}

\end{itemize}
\end{frame}

\subsection{Finite Differences}

\begin{frame}{Finite Difference Methods}

\begin{itemize}
\item Rather than solve a sequence of IVPs that satisfy the ODEs until they (approximately) satisfy boundary conditions, we can refine an approximation that satisfies the boundary conditions, until it satisfies the ODE:

\lgcond{
\begin{itemize}
\item
\coloremph{Finite difference methods} work by obtaining a solution on points $t_1,\ldots, t_n$, so that $\B{\hat{y}}_k \approx \B y(t_k)$ by finite-difference formulae, for example,
\begin{align*}
\B f(t,\B y)=\B y'(t)\approx \frac{\B y(t+h) - \B y(t-h)}{2h} \Rightarrow \B f(t_k, \B{\hat{y}}_k) = \frac{\B{\hat{y}}_{k+1} - \B{\hat{y}}_{k-1}}{t_{k+1}-t_{k-1}}.
\end{align*}
\item
The resulting system of equations can be solved by standard methods and is linear if $\B f$ is linear.
\end{itemize}
}

\item Convergence to solution is obtained with decreasing step size $h$ so long as the method is consistent and stable:

\lgcond{
\begin{itemize}
\item 
Consistency implies that the truncation error goes to zero.
\mitem 
Stability ensures input perturbations have bounded effect on solution.
\end{itemize}
}
\end{itemize}
\end{frame}

%\begin{frame}{Collocation Methods}
%
%\begin{itemize}
%\item \coloremph{Collocation methods} approximate $\B y$ by representing it in a basis
%\[\B y(t) = \B v(t,\B x) = \sum_{i=1}^n x_i\B \phi_i(t)\]
%\tlgcond{
%To construct equations, consider approximation for a set of collocation points $t_1,\ldots, t_n$ with $t_1=a$ and $t_n=b$
%\[\forall_i \B v(t_i, \B x) = \B f(t_i,\B v(t_i, \B x))\]
%}
%\item \coloremph{Spectral methods} let each $\B \phi_i$ (e.g. polynomial or trigonometric function) be nonzero over most of $[a,b]$, while \coloremph{finite element} methods leverage basis functions with local support (e.g. B-splines).
%
%\lgcond{
%Trigonometric functions and polynomials are typically eigenfunctions of differential operators, hence the name ``spectral methods''.
%}
%
%\end{itemize}
%\end{frame}
%
%
%\begin{frame}{Solving BVPs by Optimization}
%
%\begin{itemize}
%\item We reformulate the collocation approximation as an optimization problem:
%
%\tlgcond{
%Consider the simplified scenario $\B f(t,y)=\B f(t)$ with residual equation,
%\[\B r(t, \B x) = \B v'(t, \B x) - \B f(t) = \sum_{j=1}^n x_j \B \phi'_j(t) - \B f(t)\]
%and minimize it using the objective function,
%\[F(\B x) = \frac{1}{2} \int_a^b \B r(t,\B x)^2 dt = \int_a^b \B r(t,\B x)\phi_i(t).\]
%}
%\item The first-order optimality conditions of the optimization problem, yields a system of linear equations $\B A\B x = \B b$:
%\tlgcond{
%\begin{align*}
%\B 0 = \frac{d\B F}{d x_i} &= \int_a^b \B r(t, \B x) \frac{dr}{dx_i} dt = \int_a^b \B r(t, \B x) \B \phi_i'(t) dt \\
% &=\sum_{j=1}^n  x_j\underbrace{\int_a^b \B \phi'_j(t)^T\B \phi_i'(t) dt}_{a_{ij}} - \underbrace{\int_a^bf(t) \B \phi_i'(t) dt}_{b_i}
%\end{align*}
%}
%
%
%\end{itemize}
%\end{frame}
%
%\begin{frame}{Weighted Residual}
%
%\begin{itemize}
%\item \coloremph{Weighted residual methods} work by ensuring the residual is orthogonal with respect to a given set of weight functions:
%
%\tlgcond{
%Rather than setting components of the gradient to zero, we instead have,
%\[\int_a^b r(t,\B x)w_i(t) dt = 0, \forall i \in \{1,\ldots, n\},\]
%which again yields a system of equations of the form $\B A \B x = \B b$ where
%\[a_{ij} = \int_a^b \phi_j'(t) w_i(t), \quad b_i=\int_a^b f(t) w_i(t).\]
%The collocation method is a weighted residual method where $w_i(t)=\delta(t_i)$.
%}
%\item The Galerkin method is a weighted residual method where $w_i=\phi_i$.
%
%\lgcond{
%For a first-order differential equation, this is equivalent to setting $w_i = \frac{dr}{dx_i}$, so we obtain a linear system with the \coloremph{stiffness matrix} $\B A$ and \coloremph{load vector} $\B b$ defined as before,
%\( \B 0 =\sum_{j=1}^n  x_j\underbrace{\int_a^b \B \phi'_j(t)^T\B \phi_i'(t) dt}_{a_{ij}} - \underbrace{\int_a^bf(t) \B \phi_i'(t) dt}_{b_i}.\)
%}
%
%
%\end{itemize}
%\end{frame}



%\end{document}


\begin{frame}{Finite Difference Methods}

\urcornerlinkdemo{10-boundary-value-problems}{Finite differences}

\begin{itemize}
\item Lets derive the finite difference method for the ODE BVP defined by

\[u''+7(1+t^2)u=0\]
with boundary conditions $u(-1)=3$ and $u(1)=-3$, 
using a centered difference approximation for $u''$ on $t_1,\ldots, t_n$, $t_{i+1}-t_i=h$.

\lgcond{
\begin{itemize}
\item We have equations $u(-1)=u(t_1)=u_1=3$, $u(1)=u(t_n)=u_n=3$ and $n-2$ finite difference equations, one for each $i\in\{2,\ldots,n-1\}$,
\[\frac{u_{i+1} - 2u_i + u_{i-1}}{h^2} + 7(1+t_i^2)u_i = 0.\]
\item These correspond to a linear system based on matrices:
{\tiny
\[\B A = \begin{bmatrix} 1 & & & & \\ 1/h^2 & -2/h^2 & 1/h^2 & & \\ & \ddots & \ddots & \ddots & \\ & & 1/h^2 & -2/h^2 & 1/h^2 \\ & & & & 1 \end{bmatrix} \quad \text{and} \quad
\B B = \begin{bmatrix} 0 & & & & \\ 0 & 7(1+t_2^2) & & & \\ &  & \ddots & \\ & & & 7(1+t_{n-1}^2) & 0 \\ & & & & 0\end{bmatrix},\]
}
where $(\B A +\B B) \B u = \begin{bmatrix} 3 & 0 & \cdots 0 & -3\end{bmatrix}^T$.
\end{itemize}
}
\lgcond{}
\end{itemize}
\end{frame}

\subsection{Collocation}

\begin{frame}{Collocation Methods}

\urcornerlinkdemo{10-boundary-value-problems}{Sparse matrices}

\begin{itemize}
\item \coloremph{Collocation methods} approximate $\B y$ by representing it in a basis
\[\B y(t) \approx \B v(t,\B x) = \sum_{i=1}^n x_i\B \phi_i(t).\]

\tlgcond{
\begin{itemize}
\item
Seek to satisfy for collocation points $t_1,\ldots, t_n$ with $t_1=a$ and $t_n=b$,
\[ \forall_{i\in\{2,\ldots,n-1\}}\quad \B v'(t_i, \B x) = \B f(t_i,\B v(t_i, \B x)).\]
\item
Two more equations typically obtained from boundary conditions at $t_1,t_n$.
\end{itemize}
}
\item Choices of basis functions give different families of methods:

\lgcond{
\begin{itemize}
\item \coloremph{Spectral methods} use polynomials or trigonometric functions for $\B \phi_i$, which are nonzero over most of $[a,b]$, and  have the advantage of corresponding to eigenfunctions of differential operators.
\mitem \coloremph{Finite element} methods leverage basis functions with local support (e.g. B-splines) and yield sparsity in the resulting problem since many pairs of basis functions have disjoint support.
\end{itemize}
}

\end{itemize}
\end{frame}


\begin{frame}{Solving BVPs by Optimization}

\begin{itemize}
%\item We reformulate the collocation approximation as an optimization problem:
\item To improve robustness, define and minimize a residual error over the whole domain rather than at collocation points. 
\tlgcond{
\begin{itemize}
\item
For simplified scenario $\B f(t,y)=\B f(t)$,
\[\B r(t, \B x) = \B v'(t, \B x) - \B f(t) = \sum_{j=1}^n x_j \B \phi'_j(t) - \B f(t).\]
\item
In particular, we seek to minimize the objective function,
\[F(\B x) = \frac{1}{2} \int_a^b ||\B r(t,\B x)||_2^2 dt .\]%= \int_a^b \B r(t,\B x)^T\B \phi_i(t) dt
\end{itemize}
}
\item The first-order optimality conditions of the optimization problem are a system of linear equations $\B A\B x = \B b$:
\vspace*{-.05in}
\tlgcond{
\begin{align*}
\B 0 = \frac{dF}{d x_i} &= \int_a^b \B r(t, \B x)^T \frac{d\B r}{dx_i} dt = \int_a^b \B r(t, \B x)^T \B \phi_i'(t) dt \\
 &=\sum_{j=1}^n  x_j\underbrace{\int_a^b \B \phi'_j(t)^T\B \phi_i'(t) dt}_{a_{ij}} - \underbrace{\int_a^b\B f(t)^T \B \phi_i'(t) dt}_{b_i}
\end{align*}
}


\end{itemize}
\end{frame}

\subsection{Finite Element Methods}

\begin{frame}{Weighted Residual}

\begin{itemize}
\item \coloremph{Weighted residual methods} work by ensuring the residual is orthogonal with respect to a given set of weight functions:

\tlgcond{
\begin{itemize}
\item
Rather than setting components of the gradient to zero, we instead have
\[\int_a^b \B r(t,\B x)^T\B w_i(t) dt = 0, \forall i \in \{1,\ldots, n\}.\]
\item
Again, we obtain a system of equations of the form $\B A \B x = \B b$, where
\[a_{ij} = \int_a^b \B \phi_j'(t)^T \B w_i(t), \quad b_i=\int_a^b \B f(t)^T \B w_i(t).\]
\item
The collocation method is a weighted residual method where $\B w_i(t)=\B \delta(t-t_i)$.
\end{itemize}
}
\item The Galerkin method is a weighted residual method where $\B w_i=\B \phi_i$.

\lgcond{
%For a first-order differential equation $\B y'(t)=\B f(t)$, this is equivalent to setting $\B w_i = \frac{d\B r}{dx_i}$, so 
Linear system with the \coloremph{stiffness matrix} $\B A$ and \coloremph{load vector} $\B b$ is
\[ \B 0 =\sum_{j=1}^n  x_j\underbrace{\int_a^b \B \phi'_j(t)^T\B \phi_i(t) dt}_{a_{ij}} - \underbrace{\int_a^b \B f(t)^T \B \phi_i(t) dt}_{b_i}.\]
}


\end{itemize}
\end{frame}


%\begin{frame}{Linear BVPs by Optimization}
%
%\begin{itemize}
%\item Lets apply the Galerkin method to the more general linear ODE $\B f(t,y)=\B A(t) \B y(t) + \B b(t)$ with residual equation,
%
%\tlgcond{
%First, choose basis functions $\{\B \phi_i\}_{i=1}^n$ to satisfy the boundary conditions, so solution automatically satisfies them, then minimize the residual,
%\[\B r = \B v'- \B A \B v - \B b, \ \ \text{so that}\ \  \B r(t,\B x) = \sum_{j=1}^n x_j (\B \phi'_j(t) - \B A(t)\B \phi_j(t)) - \B b(t).\]
%%and minimize it using the objective function,
%%\[F(\B x) = \frac{1}{2} \int_a^b \B r(t,\B x)^2 dt = \int_a^b \B r(t,\B x)\phi_i(t) dt.\]
%%}
%%\item The first-order optimality conditions of the optimization problem, yields a system of linear equations $\B A\B x = \B b$:
%%\tlgcond{
%%\begin{align*}
%%\B 0 = \frac{d\B F}{d x_i} &= \int_a^b \B r(t, \B x)^T \frac{d\B r}{dx_i} dt = \int_a^b \B r(t, \B x)^T \B \psi_i(t) dt \\
%% &=\sum_{j=1}^n  x_j\int_a^b \B \psi_j(t)^T\B \psi_i(t) dt - \int_a^b\B b(t) \B \phi_i'(t) dt
%%\end{align*}
%The Galerkin method, minimizes the residual by orthogonality with respect to a set of test functions that is the same as the set of basis functions,
%\begin{align*}
%\B 0  &= \int_a^b \B r(t, \B x)^T \B \phi_i(t) dt   \\
% &=\sum_{j=1}^n  x_j\int_a^b (\B \phi'_j(t) - \B A(t)\B \phi_j(t))^T\B \phi_i(t) dt - \int_a^b\B b(t)^T \B \phi_i(t) dt
%\end{align*}
%}
%\lgcond{}
%
%
%\end{itemize}
%\end{frame}


\begin{frame}{Second-Order BVPs: Poisson Equation}

In practice, BVPs are at least second order and its advantageous to work in the natural set of variables.
\begin{itemize}
\item Consider the \coloremph{Poisson equation} $u''=f(t)$ with boundary conditions $u(a)=u(b)=0$ and define a localized basis of hat functions:
\lgcond{
\[%\forall i \in \{1,n\} , \quad 
\phi_i(t) = \begin{cases} (t-t_{i-1})/h &: t \in [t_{i-1},t_i] \\ (t_{i+1}-t)/h &: t\in [t_{i},t_{i+1}] \\ 0 &: \text{otherwise} \end{cases}\]
for $i\in \{1,\ldots, n\}$, handling boundaries via $t_0=t_1=a$ and $t_{n+1}=t_n=b$.
}
\item
Defining residual equation by analogy to the first order case, we obtain,
\tlgcond{
\[r =  v'' -  f, \ \ \text{so that}\ \   r(t,\B x) = \sum_{j=1}^n x_j  \phi''_j(t) -  f(t).\]
However, with our choice of basis, $\B \phi''_j(t)$ is undefined, since $\B \phi'_j(t)$ is discontinuous at $t_{j-1},t_j,t_{j+1}$.
}

\end{itemize}
\end{frame}

\subsection{Weak Form of ODEs}
\begin{frame}{Weak Form and the Finite Element Method}

\begin{itemize}

\item The finite-element method permits a lesser degree of differentiability of basis functions by casting the ODE in \coloremph{weak form}:

\tlgcond{
\begin{itemize}
\item For any solution $u$, if test function $\phi_i$ satisfies the boundary conditions, the ODE satisfies the weak form,
\begin{align*}
\int_a^b f(t) \phi_i(t) dt &=\int_a^b u''(t) \phi_i(t) dt 
=u'(b) \underbrace{\phi_i(b)}_0 - u'(a) \underbrace{\phi_i(a)}_0 - \int_a^b u'(t) \phi_i'(t) dt \\
&=  - \int_a^b u'(t) \phi_i'(t) dt.
\end{align*}
\item Note that the final equation contains no second derivatives, and subsequently we can form the linear system $\B A \B x = \B b$ with
\[a_{ij} = - \int_a^b \phi_j'(t) \phi_i'(t) dt, \quad b_i= \int_a^b f(t) \phi_i(t) dt.\] 
\item The finite element method thus searches the larger (once-differentiable) function space to find a solution $u$ that is in a (twice-differentiable) subspace.
\end{itemize}
}
\lgcond{}


\end{itemize}
\end{frame}

%\begin{frame}{Finite Element Methods in Practice}
%
%\begin{itemize}
%\item Hat functions are linear instances of \coloremph{B-splines}:
%
%\lgcond{
%B-splines of degree $k$ are $k$-times differentiable.
%For higher-order ODEs or high-order convergence with $h$, its necessary to use $k>1$.
%}
%
%\item Finite-element methods readily generalize to PDEs:
%
%\lgcond{
%In its most basic form each element corresponds to a triangle (2D) or quadrilateral (3D).
%}
%
%\end{itemize}
%
%\end{frame}

\section{Eigenvalue Problems for ODEs}

\begin{frame}{Eigenvalue Problems with ODEs}

\begin{itemize}
\item A typical second-order scalar \coloremph{ODE BVP eigenvalue problem} is
\[u'' = \lambda f(t,u,u'), \quad \text{with boundary conditions } \  u(a)=0, u(b)=0.\]
These can be solved, e.g. for $f(t,u,u')=g(t)u$ by finite differences:

\lgcond{
%Lets first consider $f(t,u,u')=g(t)u$, in which case we 
\begin{itemize}
\item Approximating the solution at a set of points $t_1,\ldots, t_n$ using finite differences,
\[\frac{y_{i+1}-2y_i+y_{i-1}}{h^2} = \lambda g_iy_i.\]
\item This yields a tridiagonal matrix eigenvalue problem $\B A \B y =\lambda \B y$ where
\[\frac{y_{i+1}-2y_i+y_{i-1}}{g_ih^2} = \lambda y_i.\]
\end{itemize}
}
\lgcond{}
\end{itemize}

\end{frame}

\begin{frame}{Using Generalized Matrix Eigenvalue Problems}

\begin{itemize}
\item Generalized matrix eigenvalue problems arise from more sophisticated ODEs,
\[u'' = \lambda (g(t)u + h(t)u'), \quad \text{with boundary conditions } \  u(a)=0, u(b)=0.\]

\lgcond{
\begin{itemize}
\item
Again approximate each of the derivatives at a set of points $t_1,\ldots, t_n$ using finite differences,
\[\frac{y_{i+1}-2y_i+y_{i-1}}{h^2} = \lambda \bigg(g_i+ \frac{y_{i+1}-y_{i-1}}{2h}\bigg)y_i.\]
\item
These corresponds to a generalized matrix eigenvalue problem 
\[\B A \B y =\lambda \B B \B y,\]
where both $\B A$ and $\B B$ are tridiagonal.
\mitem
Specialized methods exist for solving generalized matrix eigenvalue problems (also referred to as matrix pencil eigenvalue problems).
\end{itemize}
}
\lgcond{}
\end{itemize}

\end{frame}






%\end{document}
