%%\documentclass[handout]{beamer}
%\documentclass[aspectratio=169,13pt]{beamer}

%\newcommand{\dee}{\partial}

\newcommand{\bmat}[1]{\begin{bmatrix} \B{#1}_{11} & \B{#1}_{12} \\ \B{#1}_{21} & \B{#1}_{22}\end{bmatrix}}
\newcommand{\blmat}[1]{\begin{bmatrix} \B{#1}_{11} &  \\ \B{#1}_{21} & \B{#1}_{22}\end{bmatrix}}
\newcommand{\brmat}[1]{\begin{bmatrix} \B{#1}_{11} & \B{#1}_{12} \\ & \B{#1}_{22}\end{bmatrix}}
\newcommand{\bomat}[1]{\begin{bmatrix} {#1}_{11} & \B{#1}_{12} \\ \B{#1}_{21} & \B{\uppercase{#1}}_{22}\end{bmatrix}}
\newcommand{\blomat}[1]{\begin{bmatrix} {#1}_{11} & \\ \B{#1}_{21} & \B{\uppercase{#1}}_{22}\end{bmatrix}}
\newcommand{\bromat}[1]{\begin{bmatrix} {#1}_{11} & \B{#1}_{12} \\  & \B{\uppercase{#1}}_{22}\end{bmatrix}}

\DeclareMathOperator{\vcop}{vec}
\newcommand{\vc}[1]{\vcop\left(#1\right)}
\newcommand{\prm}{\mu}

\newcommand{\fl}{fl}
\newcommand{\asmcond}[1]{\smallskip{\it #1}\smallskip}
\newcommand{\amdcond}[1]{\smallskip{\it #1}\smallskip}
\newcommand{\algcond}[1]{\smallskip{\it #1}\smallskip}
\newcommand{\atsmcond}[1]{{\it #1}}
\newcommand{\atmdcond}[1]{{\it #1}}
\newcommand{\atlgcond}[1]{{\it #1}}
\newcommand{\afillnum}[1]{{\underline{ \ #1\ \ }}}

\mode<presentation>
{
  \usetheme{Malmoe}
}

\usepackage[english]{babel}
%
%\usepackage[latin1]{inputenc}
%
%\usepackage{times}
%\usepackage[T1]{fontenc}
% Or whatever. Note that the encoding and the font should match. If T1
% does not look nice, try deleting the line with the fontenc.

\usepackage{inconsolata}
\usepackage{listings}
\usepackage{bm}
\usefonttheme[onlymath]{serif}
\renewcommand\familydefault{\sfdefault}
\usepackage[sfdefault]{ClearSans} %% option 'sfdefault' activates Clear Sans as the default text font
%\usefonttheme{professionalfonts}

\usepackage{latexsym}
\usepackage{amsmath}
\usepackage{amssymb}
\usepackage{amsfonts}
\usepackage{graphics}
\usepackage{multicol}

\definecolor{darkred}{rgb}{0.8,0.2,0.2}
\newcommand{\coloremph}[1]{\textcolor{darkred}{\emph{#1}}}        % colored italics
\newcommand{\reshape}[2]{o_{#1}(#2)}        % colored italics
\newcommand{\ket}[1]{\lvert #1 \rangle}
\newcommand{\bra}[1]{\langle #1 \vert}
\newcommand{\transp}[2]{{#2}^{\langle  {#1} \rangle }}        % colored italics

% Miscellaneous special math symbols

% Big-oh notation - either Latex's calligraphic O or uppercase math italic O
\newcommand{\BIGOH}{\mathcal{O}}                              % big oh
%\newcommand{\BIGOH}{O}                                       % big oh
\newcommand{\BIGTHETA}{\Theta}                                % big theta

\newcommand{\sfrac}[2]{{#1}/{#2}}

% Various versions of the fraction one-half
% solidus 1/2 for superscripts:
\newcommand{\SHALF}{1/2}                                      % one-half power
% small 1/2 set case in displayed equations:
\newcommand{\HALF}{\mbox{\small $\frac{1}{2}$}}               % small one-half
\newcommand{\THRD}{\mbox{\scriptsize $\frac{1}{3}$}}          % small one-third
\newcommand{\TWOTH}{\mbox{\scriptsize $\frac{2}{3}$}}         % small two-thirds

% Machine epsilon - note that placement of subscript may need adjustment.
%\newcommand{\emach}{\epsilon_{\textrm{\scriptsize mach}}} % machine prec.
\newcommand{\emach}{\epsilon}

\newcommand{\lsapprox}{\cong}                             % least squares approx

% Real numbers - Prefer \mathbb{R} but it requires amsfonts.
%                Latex's calligraphic R will do if amsfonts are unavailable.
%                DO NOT use \Re, which gives old German fraktur R.
\newcommand{\Real}{\mathbb{R}}                                 % real numbers
\newcommand{\Cplx}{\mathbb{C}}                                 % complex numbers
\newcommand{\Poly}{\mathbb{P}}                                 % polynomials
\newcommand{\Float}{\mathbb{F}}                                % fl pt system
% Bold math fonts for vectors and matrices
\renewcommand{\Vec}[1]{\ensuremath{\bm{#1}}}                   % vector
\newcommand{\Mat}[1]{\ensuremath{\bm{#1}}}                     % matrix
\newcommand{\Op}[1]{#1}                              % operator
\newcommand{\loc}[1]{x_{#1}}                              % operator

% Loglike functions set in regular type
\newcommand{\diag}{\mathrm{diag}}                             % diagonal matrix
\newcommand{\cond}{\mathrm{cond}}                             % condition number
\newcommand{\sign}{\mathrm{sign}}                             % sign function
\newcommand{\Span}{\text{span}}                             % span of matrix
%\newcommand{\trace}{\mathrm{trace}}                           % trace of matrix
\DeclareMathOperator*{\trace}{trace}
\newcommand{\tentrace}[3]{\trace_{#1,#2}(#3)}
\renewcommand{\Re}{\mathrm{Re}}                               % real part
\renewcommand{\Im}{\mathrm{Im}}                               % imaginary part

% Fractions appearing in matrices - allows setting them case or solidus
%\newcommand{\mf}[2]{{#1 \over #2}}        % matrix fraction - set case
\newcommand{\mf}[2]{#1/#2}               % matrix fraction - set solidus


% Keywords for algorithm statements

\newcommand{\FOR}{\textbf{for}\ }
\newcommand{\TO}{\textbf{to}\ }
\newcommand{\IF}{\textbf{if}\ }
\newcommand{\THEN}{\textbf{then}\ }
\newcommand{\ELSE}{\textbf{else}\ }
\newcommand{\WHILE}{\textbf{while}\ }
\newcommand{\DO}{\textbf{do}\ }
\newcommand{\BEGIN}{\textbf{begin}\ }
\newcommand{\END}{\textbf{end}\ }
\newcommand{\STOP}{\textbf{stop}\ }
\newcommand{\AND}{\textbf{and}\ }
\newcommand{\OR}{\textbf{or}\ }

\newcommand{\comm}{\mathrm{comm}}
\newcommand{\comp}{\mathrm{comp}}
\newcommand{\idle}{\mathrm{idle}}
\newcommand{\msg}{\mathrm{msg}}
\newcommand{\len}{\mathrm{s}}
\newcommand{\route}{\mathrm{route}}
\newcommand{\mitem}{\medskip\item}
\newcommand{\sitem}{\smallskip\item}
\DeclareMathOperator*{\argmin}{argmin}

%Gets rid of headline

\setbeamertemplate{footline}{}

\setbeamertemplate{headline}{}
\beamertemplatenavigationsymbolsempty



\definecolor{mygreen}{rgb}{0,0.2,0}
\definecolor{mygray}{rgb}{0.5,0.5,0.5}
\definecolor{mymauve}{rgb}{0.58,0,0.82}
\definecolor{mypurple}{rgb}{0.38,0,0.32}
\definecolor{myblue}{rgb}{0.2,0,0.5}
\definecolor{darkgreen}{rgb}{0.2,0.6,0.2}
\definecolor{Brown}{rgb}{0.39, 0.09, 0.0}
\definecolor{Violet}{rgb}{0.38, 0.0, 0.63}
\definecolor{myvlgray}{rgb}{0.9,0.9,1.0}
\definecolor{mylgray}{rgb}{0.85,0.85,0.9}
\definecolor{darkgreen}{rgb}{0,0.4,0}
\definecolor{brown}{rgb}{.6,.1,.1}

\newcommand{\bemph}[1]{{\color{blue} \emph{#1}}}

\usepackage{mathtools}
\newcommand{\defeq}{\coloneqq}

\newcommand{\inti}[2]{[{#1},{#2}]}
\newcommand{\BF}{\mathbf}
\newcommand{\CF}{\mathcal}
\newcommand{\B}[1]{\bm{#1}}
\newcommand{\E}[1]{\bm{#1}}
\newcommand{\dis}{\displaystyle}
\newcommand{\lt}{\left}
\newcommand{\rt}{\right}

\newcommand{\cs}{H}

\DeclareMathOperator*{\rank}{rank}
\DeclareMathOperator*{\vecn}{vec}
\DeclareMathOperator*{\vecs}{vech}

\usepackage{eso-pic}
\newcommand{\cornertext}[1]{
  \AddToShipoutPictureFG*{
    \AtPageUpperLeft{\put(0,-10){\makebox[\paperwidth][r]{#1}}}  
   }%
}
\newcommand{\cornertexttwo}[2]{
  \AddToShipoutPictureFG*{
    \AtPageUpperLeft{\put(0,-10){\makebox[\paperwidth][r]{#1}}}  
    \AtPageUpperLeft{\put(0,-20){\makebox[\paperwidth][r]{#2}}}  
   }%
}
\newcommand{\linkdemo}[2]{{\footnotesize\href{https://relate.cs.illinois.edu/course/cs450-f18/f/demos/upload/#1/#2.html}{\color{Violet}{\it\textbf{Demo:} #2}\ }}}
\newcommand{\linkinclass}[2]{{\footnotesize\href{https://relate.cs.illinois.edu/course/cs450-f18/flow/#1/start/}{\color{darkgreen}{\it\textbf{Activity:} #2}\ }}}
\newcommand{\urcornerlinkinclass}[2]{\cornertext{\linkinclass{#1}{#2}}}
\newcommand{\dblurcornerlinkinclass}[4]{\cornertexttwo{\linkinclass{#1}{#2}}{\linkinclass{#3}{#4}}}
\newcommand{\urcornerlinkdemo}[2]{\cornertext{\linkdemo{#1}{#2}}}
\newcommand{\dblurcornerlinkdemo}[4]{\cornertexttwo{\linkdemo{#1}{#2}}{\linkdemo{#3}{#4}}}
\newcommand{\urcornerlinkdemoinclass}[4]{\cornertexttwo{\linkdemo{#1}{#2}}{\linkinclass{#3}{#4}}}



% CS 554 specific:

\newcommand{\tsync}{\alpha}
\newcommand{\tword}{\beta}
\newcommand{\tflop}{\gamma}

\newcommand{\bw}{w}

\newcommand{\tpl}[2]{{\bm{#1}}}
\newcommand{\vtpl}[3]{{\bm{#1}}_#3}
\newcommand{\perm}[2]{\lt[#1\rt]_{#2}}
\newcommand{\costyle}{\ttfamily\bfseries}
\newcommand{\cotext}[1]{{\costyle{#1}}}
\newcommand{\kwstyle}{\costyle\textcolor{mypurple}}
\newcommand{\kwtext}[1]{{\kwstyle{#1}}}
\newcommand{\emstyle}{\costyle\textcolor{myblue}}
\newcommand{\emtext}[1]{{\emstyle{#1}}}
\newcommand{\const}{}
\newcommand{\work}{Q}
\newcommand{\depth}{D}
\newcommand{\flops}{F}
\newcommand{\words}{W}
\newcommand{\syncs}{S}
\newcommand{\mem}{M}
\newcommand{\eff}{E}
\newcommand{\isofun}[1]{\tilde{\work}(p)}
\newcommand{\isomem}[1]{\tilde{\mem}(p)}
\newcommand{\pstrong}{p_s}
\newcommand{\pweak}{p_w}
\newcommand{\ALL}{\star }
\newcommand{\rmn}[2]{\mathbb{R}^{#1\times #2}}
\newcommand{\rn}[1]{\mathbb{R}^{#1}}
\newcommand{\err}{\varepsilon}
\newcommand{\mat}[1]{\begin{bmatrix} #1 \end{bmatrix}}
\newcommand{\mc}[1]{\mathcal{#1}}
\newcommand{\h}[2]{\mc{H}_{#1}(#2)}
\newcommand{\T}{T}%{\mathsf{T}}
\newcommand{\dn}[2]{\mc{M}_{#1}^{\uparrow}(#2)}
%\usepackage[colorlinks=false,urlbordercolor={1.0 1.0 1.0}]{hyperref}
%
%\lstset{ %
%  postbreak=false,
%%  backgroundcolor=\color{white},   % choose the background color; you must add \usepackage{color} or \usepackage{xcolor}
%  basicstyle=\costyle,        % the size of the fonts that are used for the code
%%  breakatwhitespace=false,         % sets if automatic breaks should only happen at whitespace
%%  breaklines=false,                 % sets automatic line breaking
%  captionpos=n,                    % sets the caption-position to bottom
%  commentstyle=\color{mygreen},    % comment style
%  deletekeywords={...},            % if you want to delete keywords from the given language
%  escapeinside={\%*}{*)},          % if you want to add LaTeX within your code
%  extendedchars=true,              % lets you use non-ASCII characters; for 8-bits encodings only, does not work with UTF-8
%  frame=none,                    % adds a frame around the code
%  keepspaces=true,                 % keeps spaces in text, useful for keeping indentation of code (possibly needs columns=flexible)
%  keywordstyle=\color{mypurple},       % keyword style
%  language=C++,                 % the language of the code
%  otherkeywords={*,...},            % if you want to add more keywords to the set
%  numbers=none,                    % where to put the line-numbers; possible values are (none, left, right)
%  numbersep=5pt,                   % how far the line-numbers are from the code
%  numberstyle=\tiny\color{mygray}, % the style that is used for the line-numbers
%  rulecolor=\color{black},         % if not set, the frame-color may be changed on line-breaks within not-black text (e.g. comments (green here))
%  showspaces=false,                % show spaces everywhere adding particular underscores; it overrides 'showstringspaces'
%  showstringspaces=false,          % underline spaces within strings only
%  showtabs=false,                  % show tabs within strings adding particular underscores
%  stepnumber=2,                    % the step between two line-numbers. If it's 1, each line will be numbered
%  stringstyle=\color{mymauve},     % string literal style
%  tabsize=2,                     % sets default tabsize to 2 spaces
%  title=\lstname,                   % show the filename of files included with \lstinputlisting; also try caption instead of title
%  keywordstyle=\emstyle
%}

\lstset{ %
  postbreak=false,
%  backgroundcolor=\color{white},   % choose the background color; you must add \usepackage{color} or \usepackage{xcolor}
  basicstyle=\footnotesize\costyle,        % the size of the fonts that are used for the code
%  breakatwhitespace=false,         % sets if automatic breaks should only happen at whitespace
%  breaklines=false,                 % sets automatic line breaking
  captionpos=n,                    % sets the caption-position to bottom
  commentstyle=\color{mygreen},    % comment style
  deletekeywords={privatei,shared},            % if you want to delete keywords from the given language
  escapeinside={\%*}{*)},          % if you want to add LaTeX within your code
  extendedchars=true,              % lets you use non-ASCII characters; for 8-bits encodings only, does not work with UTF-8
  frame=none,                    % adds a frame around the code
  keepspaces=true,                 % keeps spaces in text, useful for keeping indentation of code (possibly needs columns=flexible)
  keywordstyle=\footnotesize\color{mypurple},       % keyword style
  language=C++,                 % the language of the code
  otherkeywords={*,pragma},            % if you want to add more keywords to the set
  numbers=none,                    % where to put the line-numbers; possible values are (none, left, right)
  numbersep=5pt,                   % how far the line-numbers are from the code
  numberstyle=\footnotesize\color{mygray}, % the style that is used for the line-numbers
  rulecolor=\color{black},         % if not set, the frame-color may be changed on line-breaks within not-black text (e.g. comments (green here))
  showspaces=false,                % show spaces everywhere adding particular underscores; it overrides 'showstringspaces'
  showstringspaces=false,          % underline spaces within strings only
  showtabs=false,                  % show tabs within strings adding particular underscores
  stepnumber=2,                    % the step between two line-numbers. If it's 1, each line will be numbered
  stringstyle=\footnotesize\color{mymauve},     % string literal style
  tabsize=2,                     % sets default tabsize to 2 spaces
  title=\lstname,                   % show the filename of files included with \lstinputlisting; also try caption instead of title
  emph={MPI_Send,MPI_Recv,MPI_Comm,MPI_Comm_rank,MPI_Send,MPI_Comm_split,MPI_Init,MPI_Finalize,MPI_Status,MPI_COMM_WORLD,MPI_Datatype,MPI_Comm_size,MPI_FLOAT,mpi.h,omp,parallel,shared,private,default,reduction},
  emphstyle=\footnotesize\emstyle
}

\title{CS 450: Numerical Anlaysis\footnote{{\it These slides have been drafted by Edgar Solomonik as lecture templates and supplementary material for the book ``Scientific Computing: An Introductory Survey'' by Michael T. Heath (\href{http://heath.cs.illinois.edu/scicomp/notes/index.html}{{\color{blue} slides}}).}}}
\author{}
\institute{University of Illinois at Urbana-Champaign}




\subtitle{Nonlinear Equations}

\date{}
\begin{document}

\begin{frame}
  \titlepage
\end{frame}

\section{Problem and Approach}

\begin{frame}{Solving Nonlinear Equations}

\urcornerlinkinclass{inclass-newton-2-by-2}{Newton's Method for 2-by-2 System of Equations}

\begin{itemize}
\item
Solving (systems of) nonlinear equations corresponds to root finding:
\begin{itemize}
\sitem $f(x^*)=0$ \smcond{univariate nonlinear function}
\sitem $f(\B{x}^*)=0$ \smcond{multivariate, scalar-valued nonlinear function}
\sitem $\B f(\B x^*) = \B 0$ \smcond{multivariate, vector-valued nonlinear function}
\sitem Algorithms for root-finding make it possible to solve systems of nonlinear equations and employ a similar methodology to finding minima in optimization.
\end{itemize}

%\mdcond{}

\sitem
Main algorithmic approach: find successive roots of local linear approximations of $\B f$:

\lgcond{
\begin{itemize}
%\item various alternatives exist, including simple $\B g(\B x) = \B x - \B f(\B x)$
\item Newton's method for univariate functions starting at point $x_k$ finds root of $ h(\delta x) = f(x_k) + f'(x_k)\delta x\approx f(x_k+\delta x)$, so
\[x_{k+1} = x_k + \delta x =  x_k - f(x_k)/f'(x_k)\]
\item Newton's method for multivariate functions starting at point $\B x_k$ finds root of $\B h(\B{\delta x}) = \B f(\B x_k) + \B J_{\B f}(\B x_k)\B{\delta x}\approx \B f(\B x_k+\B{\delta x})$, with
Jacobian $(\B J_{\B f}(\B x))_{ij} = \frac{\delta f_i}{\delta x_j}(\B x)$, so
  \[\B x_{k+1} = \B x_{k} + \B{\delta x} = \B x_k - \B J^{-1}_{\B f}(\B x_k) \B f(\B x_k)\]
%\sitem Newton's method achieves very rapid convergence due to the derivative/Jacobian of the fixed-point function going to zero at the root, i.e. 
%\[g'(x^*)=0 \quad \text{and more generally} \quad \B J_{\B g}(\B x^*)=\B O\]
\end{itemize}
}

\end{itemize}

\end{frame}

\section{Existence, Uniqueness, and Conditioning}

\begin{frame}{Nonexistence and Nonuniqueness of Solutions}

\urcornerlinkdemo{05-nonlinear-equations}{Three quadratic functions}

\begin{itemize}
\item Solutions do not generally exist and are not generally unique, even in the univariate case:

\lgcond{
Consider functions that are strictly greater than zero or have many zeros.
}

\item Solutions in the multivariate case correspond to intersections of hypersurfaces:

\lgcond{
The zeros of each equation define a \coloremph{hypersurface} in $\mathbb{R}^{n}$, in the linear case, these are \coloremph{hyperplanes}.
Intersections of hypersurfaces for many equations, define the solutions, which are roots of all equations. 

\smallskip

Consider that two curves can intersect at many points in space. 
Two hypersurfaces in three-dimensional space may not intersect or may have multiple \coloremph{curves of intersection}.
}

\end{itemize}

\end{frame}


\begin{frame}{Conditions for Existence of Solution}

\begin{itemize}
\item \coloremph{Intermediate value theorem} for univariate problems:

\lgcond{
If for $a<b$, $\mathop{sign}(f(a))\neq \mathop{sign}(f(b))$ and $f$ is continuous, then $[a,b]$ is a \coloremph{bracket} that contains a root,
\[\exists x^*{\in}[a,b],\quad f(x^*)=0.\]
}
\item A function has a unique \coloremph{fixed point} $\B g(\B x^*)= \B x^*$ in a given closed domain if it is \coloremph{contractive} and contained in that domain, 
\[||\B g(\B x) - \B g (\B z)|| \leq \gamma ||\B x - \B z||\]

\lgcond{
\begin{itemize}
\item Contained implies that in the domain $S$, for any $\B x \in S$,  $\B g( \B x) \in S$, while contractive implies that the function is Lipschitz continuous in $S$.
\sitem When solving for a root of $\B f$, can define various fixed point functions $\B g$, so that their solution $\B g(\B x^*) = \B x^*$ provides a root of $\B f$, $\B f(\B x^*) =\B 0$, the simplest being $\B g(\B x)=\B f(\B x) + \B x$.
\end{itemize}
}
\end{itemize}

\end{frame}


%\begin{frame}{Conditions for Uniqueness of Solution}
%
%\begin{itemize}
%\item  The solution is unique within a neighborhood, if for root $\B x^*$, \(\B{J}_{\B{f}}(\B{x}^*)\) is nonsingular:
%\lgcond{
%\[ \B J_{\B{f}} (\B{x}) = {\footnotesize \begin{bmatrix}
%     \frac{df_1}{dx_1}(\B{x}) & \cdots & \frac{df_1}{dx_n}(\B{x})\\
%     \vdots &  & \vdots \\
%     \frac{df_m}{dx_1}(\B{x}) & \cdots & \frac{df_m}{dx_n}(\B{x})
%   \end{bmatrix}   }\]
%If $\B{J}_{\B{f}}(\B{x}^*)$ is singular, $\exists \B s \neq \B 0$ so that $\B{J}_{\B{f}}(\B{x}^*)\B s=\B 0$, which means a linear approximation cannot distinguish the solution from a nearby point, $\B x^* + \alpha \B s$, which may or may not be another root.
%
%}
%\lgcond{}
%\end{itemize}
%
%\end{frame}

\section{Multiple Roots and Conditioning}

\begin{frame}{Conditioning of Nonlinear Equations}

\begin{itemize}
\item Generally, we take interest in the absolute rather than relative conditioning of solving $\B f(\B x)= \B 0$:

\lgcond{
\begin{itemize}
\item The sensitivity of solving a nonlinear equation is the ratio of magnitudes of the perturbation to the root and perturbation to the function values.
\sitem It makes sense to consider absolute perturbations to $\B f$, since any nonzero perturbation to function values is infinite in magnitude relative to $\B f(\B x^*)=\B 0$.
\end{itemize}
}

\item The \coloremph{absolute condition number} of finding a root $x^*$ of $f$ is $1/|f'(x^*)|$ and for a root  $\B x^*$ of $\B f$ it is  $||\B{J}_{\B f}^{-1}(\B x^*)||$:

\lgcond{
\begin{itemize}
\item  If we change the value of $f$ by at most $\delta f$ at any point in the function while maintaining continuity, the root will shift by at most $|\delta f|/|f'(x^*)|$ assuming $|\delta f|$ is sufficiently small.
\sitem This relationship is the converse of conditioning in function evalution, where a perturbation to input $x$, results in a perturbation of at most $\kappa_\text{abs}(f)=|f'(x)|$ larger to the function value.
\end{itemize}
}

\end{itemize}

\end{frame}


\begin{frame}{Multiple Roots and Degeneracy}

\begin{itemize}
\item If $x^*$ is a root of $f$ with \coloremph{multiplicity} $m$, its $m-1$ derivatives are also zero at $x^*$, 
\[f(x^*)=f'(x^*)=f''(x^*) = \cdots = f^{(m-1)}(x^*)=0.\]

\lgcond{
Proof: for some function $t^{(0)}(x)$, we have that 
\begin{align*}
f(x) &= (x-x^*)^mt^{(0)}(x), \\
f'(x) &= m(x-x^*)^{m-1}t^{(0)}(x) + (x-x^*)^mt^{(0)'}(x) \\
&\equiv (x-x^*)^{m-1} t^{(1)}(x), \\
f^{(m-1)}(x) &=  (x-x^*)t^{(m-1)}(x),
\end{align*}
where $t^{(i)} = (m-i+1)t^{(i-1)}(x) - (x-x^*)t^{(i-1)'}(x)$.
}

\item Increased multiplicity affects conditioning and convergence:

\lgcond{
\begin{itemize}
\item When a root $x^*$ not \coloremph{simple}, i.e. $m>1$, then $f'(x^*)=0$, so the problem of finding that root is ill-posed as $1/|f'(x^*)| = \infty$.
\sitem In practice, this means we have multiple roots at the same $x^*$ which are impossible to distinguish and may need to reformulate problem/algorithms.
\end{itemize}
}

\end{itemize}

\end{frame}
\section{Algorithms for Single-Variate Problems}

\subsection{Bisection}

\begin{frame}{Bisection Algorithm}

\urcornerlinkdemo{05-nonlinear-equations}{Bisection Method}

\begin{itemize}
\item Assume we know the desired root exists in a bracket $[a,b]$ and $\mathop{sign}(f(a))\neq\mathop{sign}(f(b))$:

\lgcond{
\begin{itemize}
\item if $f$ is continuous, by the intermediate value theorem, the bracket contains a root
\sitem can proceed to narrow interval to find a root
\sitem one caveat is that multiple roots may exist in $[a,b]$ 
\sitem another caveat is that we've imposed a restrictive condition, it can be difficult to find two points where the function has opposite sign
\end{itemize}
}

\item Bisection subdivides the interval by a factor of two at each step by considering $f(c_k)$ at $c_k=(a_k+b_k)/2$:

\lgcond{
\[[a_{k+1},b_{k+1}] = \begin{cases} [c_k,b_k] & : \mathop{sign}(f(a_k))=\mathop{sign}(f(c_k))   \\  [a_k, c_k] & : \mathop{sign}(f(b_k))=\mathop{sign}(f(c_k)) \end{cases}\]
}

\end{itemize}

\end{frame}

\subsection{Convergence Rates}

\begin{frame}{Rates of Convergence}

\urcornerlinkdemo{05-nonlinear-equations}{Rates of Convergence}

\begin{itemize}
\item Let $\B{x}_k$ be the $k$th iterate and $\B{e}_k=\B{x}_k=\B{x}^*$ be the error, bisection obtains \coloremph{linear convergence}, $\lim_{k\to \infty} ||\B{e}_k||/||\B{e}_{k-1}||\leq C$ where $C<1$:

\lgcond{
In bisection, working with the natural error bound given by bracket size,
\[e_k = b_k-a_k = \frac{1}{2}(b_{k-1}-a_{k-1}) = \frac{1}{2}e_{k-1},\]
so bisection achieves linear convergence with $C=1/2$.
With linear convergence, error $e_k\leq \epsilon$ is achieved after $O(\log_C(1/\epsilon))$ steps.
}

\item $r$th order convergence implies that $||\B{e}_k||/||\B{e}_{k-1}||^r\leq C$

\lgcond{
\begin{itemize}
\item Convergence of order $r>1$ (\coloremph{superlinear}) implies that the number of digits of correctness increases by a factor of $r$ at each step.
\sitem For $r>1$, error $e_k\leq \epsilon$ is achieved after $O(\log_r(\log(1/\epsilon)))$ steps.
\sitem Having achieved superlinear convergence ($r>1$), methods differ only by constant factors in complexity.
\end{itemize}
}

\end{itemize}

\end{frame}

\begin{frame}{Convergence of Fixed Point Iteration}

\begin{itemize}
\item Fixed point iteration: $x_{k+1}=g(x_k)$ is locally linearly convergent if for
      $x^* =g(x^*)$, we have $|g'(x^*)|< 1$:

\lgcond{
Note that,
\[e_{k+1} = x_{k+1}-x^* = g(x_k) - g(x^*). \]
By taking the Taylor expansion of $g$ at $x^*$ to represent $g(x_k)$, we can observe
\begin{align*}
e_{k+1}&= g'(x^*)(x_k-x^*)+O((x_k-x^*)^2)  \\
        &= g'(x^*)e_k+O((x_k-x^*)^2) ,
%&= g'(\theta_k)e_k, \quad \theta_k\in[x_k,x^*]
\end{align*}
where the asymptotic term decreases as $x_k$ approaches $x^*$.
}


\item It is quadratically convergent if $g'(x^*)=0$:

\lgcond{
Taking the same Taylor expansion, the leading term is now zero and we obtain 
%Taylor's theorem allows us to show quadratic convergence,
\begin{align*}
e_{k+1}&= g''(x^*)(x_k-x^*)^2/2+O((x_k-x^*)^3) \\
&= g''(x^*)e_k^2/2+O((x_k-x^*)^3) .
%e_{k+1} &= x_{k+1}-x^* = g(x_k) - g(x^*) \\
%&= g''(\zeta_k)(x_k-x^*)^2/2 \\
%&= g''(\zeta_k)|e_k|^2/2, \quad \zeta_k\in[x_k,x^*]
\end{align*}
%where the asymptotic term decreases as $x_k$ approaches $x^*$.
}

\end{itemize}

\end{frame}

\subsection{Newton's Algorithm}

\begin{frame}{Newton's Method}

\dblurcornerlinkdemo{05-nonlinear-equations}{Newton's Method}{05-nonlinear-equations}{Convergence of Newton's Method}

\begin{itemize}
\item Newton's method is derived from a \coloremph{Taylor series} expansion of $f$ at $x_k$:

\lgcond{
\[f(x) = \underbrace{f(x_k) + f'(x_k)(x-x_k)}_{\text{secant line approximation} } + (1/2!)f''(x_k)(x-x_k)^2 + \cdots\]
}

\item Newton's method is \coloremph{quadratically convergent} if started sufficiently close to $x^*$ so long as $f'(x^*)\neq 0$:

\lgcond{
\begin{itemize}
\item Newton's method corresponds to the fixed point function $g(x)=x-f(x)/f'(x)$.
\mitem It achieves quadratic convergence since $g'(x^*)=0$,
\[g'(x^*)= 1-\underbrace{f'(x^*)/f'(x^*)}_{1} + \underbrace{f(x^*)}_{0}f''(x^*)/f'(x^*)^2.\]
%\[f(x^{*})-f(x_{k+1}) \leq (1/2)f''(x_k)(x-x_k)^2 + \cdots = (1/2)f''(\xi_k)||e_k||^2, \quad \xi_k\in[x_k,x^*]\]
\end{itemize}
}

\end{itemize}

\end{frame}

\subsection{Quasi-Newton Algorithms}

\begin{frame}{Secant Method}

\dblurcornerlinkdemo{05-nonlinear-equations}{Secant Method}{05-nonlinear-equations}{Convergence of the Secant Method}

\begin{itemize}
\item The \coloremph{Secant method} approximates $f'(x_k) \approx \frac{f(x_k)-f(x_{k-1})}{x_k-x_{k-1}}$:

\lgcond{
\begin{itemize}
\item Usually, this method is the cheapest approximation possible, since function values $f(x_k)$ and $f(x_{k-1})$ are already available.
\sitem Approximation quality depends on magnitude $f(x_k)-f(x_{k-1})$ and $x_k-x_{k-1}$.
\sitem If the two points are far apart, the derivative approximation may be bad locally, while if they are very close we have to take care in handling cancellation.
\sitem A well-chosen finite-difference step at each $x_k$ provides a more robust approximation, but requires another function evaluation.
\end{itemize}

}

\item The convergence of the Secant method is \coloremph{superlinear} but not quadratic:

\lgcond{
\begin{itemize}
\item The error will now depend on the previous two errors, since we are using the previous two points.
\sitem In simplified form, the error at the $k$th iteration satisfies \(e_k\leq e_{k-1}e_{k-2}.\)
\sitem Note that $\log(e_k)=\log(e_{k-1})+\log(e_{k-2})$ is the Fibonacci sequence, which grows at a rate of $r=(1+\sqrt{5})/2$.
\sitem Thus the (negative) exponent of the error increases by roughly a factor of $r$ at each step, i.e. the convergence rate is $r$.
\end{itemize}
}

\end{itemize}

\end{frame}

\begin{frame}{Nonlinear Tangential Interpolants}

\begin{itemize}
\item Secant method uses a linear interpolant based on points $f(x_k)$, $f(x_{k-1})$, could use more points and higher-order interpolant:

\mdcond{
Have points $(x_0,f(x_0)),\ldots,(x_{k},f(x_k))$ can fit polynomial to $p(x_i)=f(x_i)$ for some subset of points $x_i\in S\subseteq \{x_0,\ldots x_k\}$.
}

\item Quadratic interpolation (Muller's method) achieves convergence rate $r\approx 1.84$:

\mdcond{
Quadratic interpolation requires three points $x_{k-2}$, $x_{k-1}$, and $x_{k}$.
}

\item Inverse quadratic interpolation resolves the problem of nonexistence/nonuniqueness of roots of polynomial interpolants:

\lgcond{
\begin{itemize}
\item Interpolate quadratic polynomial $q$ so that $q(f(x_i))=x_i$ for $i\in\{k,k-1,k-2\}$.
\sitem Approximate root simply computed by evaluating interpolant at zero $x_{k+1}=q(0)$.
\end{itemize}
}


\end{itemize}

\end{frame}
\section{Global Convergence}

\begin{frame}{Achieving Global Convergence}

\begin{itemize}
\item Hybrid bisection/Newton methods:

\lgcond{
Given a bracket (interval), can proceed with bisection until bracket is small then switch to Newton.
Alternatively, can attempt Newton, check if it stays within bracket (\coloremph{safeguard}) and proceed with change only if it does.
}

\item Bounded (damped) step-size:

\lgcond{
Newton's method gives us a direction. Decreasing the step size in that direction trades off convergence rate for reliability.
We will study how step sizes can be chosen in more detail in the context of optimization.
}

\end{itemize}

\end{frame}

%\begin{frame}{Multivariate Fixed-Point and Newton Iteration}
%
%\begin{itemize}
%\item Fixed-point iteration achieves local convergence so long as $|\lambda_\text{max}(\B{J}_{\B{g}}(\B x^*))|< 1$:
%
%\lgcond{}
%
%\item Newton's method corresponds to the fixed-point iteration $\B g(\B x) = \B x - \B J^{-1}_{\B f}(\B x) \B f(\B x)$
%
%\lgcond{}
%
%\end{itemize}
%
%\end{frame}
%
%\begin{frame}{Cost of Multivariate Newton Iteration}
%
%\begin{itemize}
%\item What is the cost of computing $\B J_{\B f}(\B x)$?
%
%\lgcond{}
%
%\item What is the cost of solving $\B J_{\B f}(\B x)\B x_k = \B f (\B{x})$?
%
%\lgcond{}
%
%\end{itemize}
%
%\end{frame}
%
%
%\begin{frame}{Secant Updating Methods}
%
%\begin{itemize}
%\item Rather than compute $\B J_{\B f}(\B x_k)$ for each $\B{x}_k$, find $\B{B}_{k+1} = \B B_{k}+\B u \B v^T$ so tas to solve or approximate secant equation
%\[\B{B}_{k+1}(\B{x}_{k+1}-\B{x}_k)=\B{f}(\B{x}_{k+1}-\B{x}_{k})\]
%
%\lgcond{}
%
%\item What is the cost of solving $\B J_{\B f}(\B x)\B x_k = \B f (\B{x})$?
%
%\lgcond{}
%
%\end{itemize}
%
%\end{frame}
%
%\begin{frame}{Global Convergence in the Multivariate Case}
%
%\begin{itemize}
%\item Hybrid methods and step-size dampening:
%
%\lgcond{}
%
%\item Methods based on trust regions:
%
%\lgcond{}
%
%\end{itemize}
%
%\end{frame}
%
%
%
%
%
%
%
%
%
%

%

\section{Systems of Nonlinear Equations}

\subsection{Jacobian}

\begin{frame}{Systems of Nonlinear Equations}

\begin{itemize}
\item
Given $\B f(\B x) = \begin{bmatrix} f_1(\B x) & \cdots & f_m(\B x) \end{bmatrix}^T$ for $\B x \in \mathbb{R}^n$, seek $\B x^*$ so that $\B f(\B x^*)=\B 0$

\lgcond{
\begin{itemize}
\item $\B{x}^*$ must simultaneously set to zero all components of $\B f$: $f_1(\B x^*)=\cdots = f_m(\B x^*)=0$.
\sitem We focus on the case of $m=n$, so that the number of equations matches the number of unknowns.
\end{itemize}
}

\item
At a particular point $\B x$, the  \coloremph{Jacobian} of $\B f$, describes how $\B f$ changes in a given direction of change in $\B x$,
\[ \B J_{\B{f}} (\B{x}) = \begin{bmatrix}
     \frac{df_1}{dx_1}(\B{x}) & \cdots & \frac{df_1}{dx_n}(\B{x})\\
     \vdots &  & \vdots \\
     \frac{df_m}{dx_1}(\B{x}) & \cdots & \frac{df_m}{dx_n}(\B{x})
   \end{bmatrix}   \]

\lgcond{
Our local linear approximation is given by
\[\B f(\B x + \B{\delta x})\approx \B f(\B x) + \B J_{\B{f}} (\B{x})\B{\delta x},\]
note that when $m=1$ the Jacobian corresponds to the gradient of $\B f$.
}


\end{itemize}

\end{frame}

\subsection{Newton's Algorithm}

\urcornerlinkdemo{05-nonlinear-equations}{Newton's method in n dimensions}

\begin{frame}{Multivariate Newton Iteration}

\begin{itemize}
\item Fixed-point iteration $\B{x}_{k+1}=\B{g}(\B{x}_k)$ achieves local convergence so long as $|\lambda_\text{max}(\B{J}_{\B{g}}(\B x^*))|< 1$ and quadratic convergence if $\B J_{\B g} = \B O$:

\lgcond{
%Given starting point $\B{x}_0$ close enough to $\B{x}^\star$, we will have $|\lambda_\text{max}(\B{J}_{\B{g}}(\B x_k))|<1$.
Taking a multivariate Taylor expansion of $\B g(\B x_k)$ at center $\B x^{*}$ we get
\begin{align*}
\B e_k &= \B x^* - \B x_{k+1} \\
       &= \B g(\B x^*) - \B g(\B x_k) \\
       &= \underbrace{\B J_{\B g}(\B x^{*})}_{\B O} (\B x^* - \B x_{k})  \\
&\ + \frac{1}{2}\begin{bmatrix} \B H_{\B g}^{(1)}(\B x^*)\cdot  (\B x^* - \B x_{k}) & \cdots & \B H_{\B g}^{(n)}(\B x^*)\cdot (\B x^* - \B x_{k})\end{bmatrix}(\B x^* - \B x_{k})\\
||\B e_{k+1}|| &= O(\max_i||\B H_{\B g}^{(i)}(\B x^*)||\cdot ||\underbrace{\B x^*- \B x_{k}}_{\B e_{k}}||^2) \\
       &= O( ||\B e_k||^2) 
%\B g(\B x^*) &= \B x^* - \B J^{-1}_{\B f}(\B x^*) \B f(\B x^*) \\
%\B{J_g}(\B x^*) &= \B I - \B J^{-1}_{\B f}(\B x^*)  \B J_{\B f}(\B x^*) - \sum_{i} \underbrace{f_i(\B x^*)}_{0}\B H_{\B f}^{(i)}(\B x^*) \\
\end{align*}
where $\B{H_g}^{(i)}$ is the $i$th component of the derivative of $\B J_{\B g}(\B x^*)$.
}
\lgcond{}

\end{itemize}

\end{frame}
%
%\begin{frame}{Convergence of Newton Iteration}
%
%\begin{itemize}
%\item Newton's method achieves quadratic local convergence if $||\B J_{\B f}^{-1}(\B x)||$ is bounded near $\B x^*$:
%
%\lgcond{
%\begin{align*}
%\B e_k & = \B x_{k} - \B x^* = \B g(\B x_{k-1}) - \B x^* \\
%       &= \B x_{k-1} - \B J_{\B{f}}^{-1} (\B{x}_{k-1}) \B f(\B x_{k-1}) - \B x^* \\
%       &= \B J_{\B{f}}^{-1} (\B{x}_{k-1}) (\B f(\B x_{k-1}) -\B J_{\B{f}}(\B{x}_{k-1})(\B x^* -\B x_{k-1}))   \\
%||\B e_k|| &\leq ||\B J_{\B{f}}^{-1} (\B{x}_{k-1})|| (||\B f(\B x_{k-1})|| + ||\B J_{\B{f}}(\B{x}_{k-1})||\cdot ||\B x^* -\B x_{k-1}||)   \\
%       &= ||\B J_{\B{f}}^{-1} (\B{x}_{k-1}) ||O(||\B x^* -\B x_{k-1}||^2)   \\
%       &= ||\B J_{\B{f}}^{-1} (\B{x}_{k-1}) ||O(||\B e_{k-1}||^2)
%\end{align*}
%}
%\lgcond{}
%
%\end{itemize}
%
%\end{frame}


\begin{frame}{Multidimensional Newton's Method}

\begin{itemize}

\item Newton's method corresponds to the fixed-point iteration 
\[\B g(\B x) = \B x - \B J^{-1}_{\B f}(\B x) \B f(\B x)\]

\lgcond{
\begin{itemize}
\item Note that generally Newton's method iteration has a fixed-point $\B{\bar{x}}$ whenever $\B f(\B{\bar{x}})=\B 0$, i.e. we have found a root of $\B f$, namely $\B{x}^*=\B{\bar{x}}$.

\sitem Generally, we require that $\B J_{\B f}(\B x^*)$ is nonsingular, otherwise we can find nonzero solutions $\B y$ to  $\B J_{\B f}(\B x^*)\B y = \B f(\B x^*)=\B 0$.

%\sitem When $m\neq n$ or $\B J_{\B f}(\B x^*)$ is singular, can instead use pseudoinverse or perform regularization.
\end{itemize}
}

\item Quadratic convergence is achieved when the Jacobian of a fixed-point iteration is zero at the solution, which is true for Newton's method:
\lgcond{
\begin{align*}
\B g(\B x^*) &= \B x^* - \B J^{-1}_{\B f}(\B x^*) \B f(\B x^*) \\
\B{J_g}(\B x^*) &= \B I - \B J^{-1}_{\B f}(\B x^*)  \B J_{\B f}(\B x^*) - \sum_{i} \underbrace{f_i(\B x^*)}_{0}\B H_{\B f}^{(i)}(\B x^*) \\
&= \B I - \B I - \B O = \B O
\end{align*}
where $\B{H_f}^{(i)}$ is the $i$th component of the derivative of $\B J^{-1}_{\B f}(\B x^*)$.
}
\end{itemize}

\end{frame}

\subsection{Quasi-Newton Algorithms}

\begin{frame}{Estimating the Jacobian using Finite Differences}

\begin{itemize}
\item To obtain $\B J_{\B f}(\B x_k)$ at iteration $k$, can use finite differences:

\lgcond{
\begin{itemize}
\item If $\B J_{\B f}(x)\in\mathbb{R}^{m\times 1}$ (single-variate but vector-valued $\B f$), we can estimate
\[\B J_{\B f}(x_k) \approx (1/h)(\B f(x_k+h)- \B f(x_k)).\]

\item
More generally, the $i$th column of $\B{j}_i$ of the Jacobian $\B J_{\B f}(\B x_k)$ can be estimated by
\[\B{j}_i\approx (1/h)(\B f(\B x_k+h \B e_i )- \B f(\B x_k)).\]
\end{itemize}
}

\item $n+1$ function evaluations are needed: $\B f(\B x)$ and $\B f(\B x+h\B e_i), \forall i\in\{1,\ldots,n\}$, which correspond to $m(n+1)$ scalar function evaluations if $\B J_{\B f}(\B x_k)\in\mathbb{R}^{m\times n}$.
\lgcond{}

\end{itemize}

\end{frame}


\begin{frame}{Cost of Multivariate Newton Iteration}

\begin{itemize}

\item What is the cost of solving $\B J_{\B f}(\B x_k)\B s_{k} = \B f (\B{x}_k)$?

\lgcond{$O(n^3)$}

\item What is the cost of Newton's iteration overall?

\lgcond{For $k$ steps, $O(n^3k+kn^2\gamma_\text{func-eval})$.}
\end{itemize}

\end{frame}


\begin{frame}{Quasi-Newton Methods}

\urcornerlinkinclass{inclass-broyden}{Broyden's Method}

In solving a nonlinear equation, seek approximate Jacobian $\B J_{\B f}(\B x_k)$ for each $\B{x}_k$
\begin{itemize}
\item Find $\B{B}_{k+1} = \B B_{k}+\B{\delta B}_k \approx \B J_{\B f}(\B x_{k+1})$, so as to approximate \coloremph{secant equation}
\[\B{B}_{k+1}(\underbrace{\B{x}_{k+1}-\B{x}_k}_{\B{\delta x}})=\underbrace{\B{f}(\B{x}_{k+1})-\B f(\B{x}_{k})}_{\B{\delta f}}\]

\lgcond{
Generally, the secant equation is underdetermined, as we usually need $n$ finite-difference formulas to determine $\B J_{\B f}(\B x_k)$, so
the secant updating methods find only approximate $\B{B}_{k+1}$, usually as a modification of $\B{B}_{k}$.
}

\item \coloremph{Broyden's method} solves the secant equation and minimizes $||\B{\delta B}_{k}||_F$:
\[\B{\delta B}_{k} = \frac{\B{\delta f} - \B{B}_k\B{\delta x}}{||\B{\delta x}||^2}\B{\delta x}^T\]

\lgcond{
Note that $\B{\delta B}_{k}$ is rank-1.
Consequently, we can use the Sherman-Morrison formula to update $\B{B}^{-1}_{k+1}$ with $O(n^2)$ work.
Various other variants exist.
}

\end{itemize}

\end{frame}

\begin{frame}{Safeguarding Methods}

\begin{itemize}
\item Can dampen step-size to improve reliability of Newton or Broyden iteration:

\mdcond{
\[\B x_{k+1} = \B x_k + \alpha_k \B s_k \quad \text{where} \quad \alpha_k\leq 1\]
can pick $\alpha_k$ so to ensure $||\B f (\B x_{k+1})|| < ||\B f (\B x_{k})||$ or by doing a line-search to minimize $||\B f(\B x_k + \alpha_k \B s_k)||.$
}

\item \coloremph{Trust region methods} provide general step-size control:

\mdcond{
Establish/maintain/update region within which step is expected to be accurate.
Pick each step to stay within trust region while minimizing $||\B f (\B x_{k+1})||$.
Observe that the Newton-like generally seek to progress to a minima of $||\B f (\B x_{k+1})||$, and indeed much of the theory of these methods targets optimization.
}
\lgcond{}

\end{itemize}

\end{frame}





%\begin{frame}{Secant Updating Methods}
%
%\begin{itemize}
%\item Rather than compute $\B J_{\B f}(\B x_k)$ for each $\B{x}_k$, find $\B{B}_{k+1} = \B B_{k}+\B{\delta B}_k \approx \B J_{\B f}(\B x_{k+1})$, so as to solve or approximate secant equation
%\[\B{B}_{k+1}(\underbrace{\B{x}_{k+1}-\B{x}_k}_{\B{\delta x}})=\underbrace{\B{f}(\B{x}_{k+1})-\B f(\B{x}_{k})}_{\B{\delta f}}\]
%
%\lgcond{
%}
%
%\item Broyden's method is given by minimizing with minimal $||\B{\delta B}_{k}||_F$:
%\[\B{\delta B}_{k} = \frac{\B{\delta f} - \B{B}_k\B{\delta x}}{||\B{\delta x}||^2}\B{\delta x}^T\]
%
%\lgcond{
%}
%
%\end{itemize}
%
%\end{frame}
%
%\begin{frame}{Trust Region Methods}
%
%\begin{itemize}
%\item Can dampen step-size to improve reliability of Newton's iteration:
%
%\mdcond{}
%
%\item \coloremph{Trust region methods} provide general step-size control:
%
%\mdcond{}
%\lgcond{}
%
%\end{itemize}
%
%\end{frame}
%
%
%
%
%
%





%\end{document}\end{document}
